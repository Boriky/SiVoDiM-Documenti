\section{Introduzione}

\subsection{Scopo del documento}
Il presente documento ha lo scopo di presentare in modo dettagliato le funzionalità che il prodotto finale offrirà all'utente. Tali funzionalità sono nate in seguito ad un attento studio sui requisiti, sorti durante i diversi incontri tra i membri del gruppo e durante l'incontro con il proponente\G. 

\subsubsection{Scopo del Prodotto}
L'obiettivo del progetto è di sperimentare e rendere disponibili su dispositivi 
mobili nuove funzionalità di sintesi vocale (TTS\G), come la possibilità di 
applicare effetti alle voci digitali o sintetizzare e utilizzare la voce degli 
utenti.
Ciò sarà possibile realizzando due applicazioni per i sistemi Android\G.
\begin{itemize}
\item La prima deve permettere all'utente di interfacciarsi direttamente con il sistema operativo per configurare, salvare, modificare nuove voci campionate;
\item La seconda invece permette la creazione, il salvataggio e la condivisione di veri e propri sceneggiati.
\end{itemize}
Entrambe le applicazioni devono sfruttare altre due componenti, realizzate sempre all'interno del progetto:
\begin{itemize}
\item Un Engine\G\ che permetta di interfacciarsi tramite una connessione al motore di sintesi prodotto dal proponente\G;
\item Una libreria di funzionalità, utile ad un riuso futuro.
\end{itemize} 

\subsubsection{Glossario}
Al fine di aumentare la comprensione del testo ed evitare eventuali ambiguità, 
viene fornito un glossario (\textit{Glossario v1.0.0}) contenente le 
definizioni degli acronimi e dei termini tecnici utilizzati nel documento. Ogni 
vocabolo contenuto nel glossario è contrassegnato dal pedice "\G ".

\subsection{Riferimenti}

\subsubsection{Normativi}
\begin{itemize}
\item \textit{Norme di Progetto v1.0}.
\item \textbf{Capitolato C6} – SiVoDiM: Sintesi Vocale per Dispositivi Mobili\\ 
\url{http://www.math.unipd.it/~tullio/IS-1/2015/Progetto/C6.pdf}.
\item \textbf{Verbali esterni}:
\begin{itemize}
\item Verbale dell'incontro con il proponente\G\ in data 09/03/2016.
\end{itemize}
\end{itemize}

\subsubsection{Informativi}
\begin{itemize}
\item \textbf{Slide dell’insegnamento Ingegneria del Software modulo A}:
\begin{itemize}
\item Ingegneria dei requisiti;
\item Diagrammi dei casi d'uso.\\
\url{http://www.math.unipd.it/~tullio/IS-1/2015/}
\end{itemize}
\item Software Engineering - Ian Sommerville - 9th Edition 2010 – Chapter 4: Requirements engineering;
\item IEEE 830-1998: \url{ https://en.wikipedia.org/wiki/Software_requirements_specification}
\end{itemize}
