2\section{Casi d'uso}
L’analisi del capitolato, l’incontro con il MIVOQ s.r.l. e la discussione tra gli \textit{Analisti} ha
portato alla definizione dei casi d’uso seguenti.\\
Ogni caso d’uso ha un codice univoco gerarchico, nella forma:\\
\begin{center}
UC[codice univoco del padre].[codice progressivo di livello]
\end{center}
Il codice progressivo può includere diversi livelli di gerarchia separati da un punto.

\subsection{Caso d'uso UC1: Scenario di alto livello}

\begin{figure}[htbp]
\centering
\includegraphics[scale=0.5]{UseCase_17_03_2016/immagini/uc_1_scenario_alto_livello.png}
\captionsetup{labelfont=bf}
\caption{Caso d'uso UC1}
\end{figure}

\begin{itemize}
\item \textbf{Attore}: Utente;
\item \textbf{Scopo e descrizione}: l'utente ha avviato correttamente l'applicazione per creare sceneggiati e questa è pronta all'uso. Possono essere effettuate diverse operazioni: l'utente può creare un nuovo sceneggiato oppure modificarne uno già iniziato; può salvare il suo lavoro, esportarlo in formato audio e condividerlo.
\item \textbf{Precondizione}: il programma è stato avviato ed è pronto all'uso;
\item \textbf{Flusso principale degli eventi}:
\begin{enumerate}
\item L'utente può creare un nuovo sceneggiato o lavorare su uno già iniziato (UC1.1 - UC1.8);
\item L'utente può esportare lo sceneggiato creato (UC1.3);
\item L'utente può salvare lo sceneggiato creato (UC1.6);
\item L'utente può condividere il file audio dello sceneggiato creato (UC1.2).
\end{enumerate}
\item \textbf{Estensioni}: la condivisione sarà disponibile solo dopo l'esportazione di un file in formato audio (UC1.2).
\item \textbf{Postcondizione}: il sistema ha ottenuto le informazioni sulle operazioni che
l'utente desidera eseguire.
\end{itemize}

\subsection{Caso d'uso UC1.1: Creazione sceneggiato}

\begin{figure}[htbp]
\centering
\includegraphics[scale=0.5]{UseCase_17_03_2016/immagini/uc_1_1_creazione_sceneggiato.png}
\captionsetup{labelfont=bf}
\caption{Caso d'uso UC1.1}
\end{figure}

\begin{itemize}
\item \textbf{Attore}: Utente;
\item \textbf{Scopo e descrizione}: l'utente ha scelto di creare un nuovo sceneggiato;
\item \textbf{Precondizione}: il sistema è pronto a creare un nuovo file;
\item \textbf{Flusso principale degli eventi}:
\begin{enumerate}
\item L'utente può assegnare un titolo al nuovo sceneggiato (UC1.1.6);
\item L'utente può creare i personaggi dello sceneggiato (UC1.1.1);
\item L'utente può scrivere le battute dello sceneggiato (UC1.1.2);
\item L'utente può associare un sentimento ad una battuta (UC1.1.3);
\end{enumerate}
\item \textbf{Scenari alternativi}: viene annullata la creazione del nuovo file se esiste già un altro sceneggiato salvato con lo stesso nome di quello appena creato; 
\item \textbf{Postcondizione}: Il sistema ha creato il nuovo file.
\end{itemize}

\subsection{Caso d'uso UC1.1.1: Creazione personaggi}

\begin{figure}[htbp]
\centering
\includegraphics[scale=0.5]{UseCase_17_03_2016/immagini/uc_1_1_1_creazione_personaggi.png}
\captionsetup{labelfont=bf}
\caption{Caso d'uso UC1.1.1}
\end{figure}

\begin{itemize}
\item \textbf{Attore}: Utente;
\item \textbf{Scopo e descrizione}: l'utente può creare un nuovo personaggio per il suo sceneggiato associandoci un nome, un avatar e una voce;
\item \textbf{Precondizione}: il sistema è pronto a creare un nuovo personaggio;
\item \textbf{Flusso principale degli eventi}:
\begin{enumerate}
\item L'utente assegna un nome al nuovo personaggio (UC1.1.1.1);
\item L'utente assegna un avatar al nuovo personaggio (UC1.1.1.3);
\item L'utente assegna una voce al nuovo personaggio (UC1.1.1.2);
\end{enumerate} 
\item \textbf{Scenari alternativi}: la creazione può fallire nel caso in cui esista già un personaggio nel corrente sceneggiato con lo stesso nome di quello dato a quello nuovo;
\item \textbf{Postcondizione}: il sistema crea un nuovo personaggio.
\end{itemize}

\subsection{Caso d'uso UC1.1.2: Assegnazione voce}

\begin{figure}[htbp]
\centering
\includegraphics[scale=0.5]{UseCase_17_03_2016/immagini/uc_1_1_1_2_assegnazione_voce.png}
\captionsetup{labelfont=bf}
\caption{Caso d'uso UC1.1.2}
\end{figure}

\begin{itemize}
\item \textbf{Attore}: Utente;
\item \textbf{Scopo e descrizione}: l'utente assegna al personaggio creato una voce tra quelle disponibili; 
\item \textbf{Precondizione}: il sistema è pronto a ricevere la selezione di una voce;
\item \textbf{Flusso principale degli eventi}:
\begin{enumerate}
\item L'utente può navigare tra le voci disponibili (UC1.1.1.2.1);
\item L'utente seleziona una voce (UC1.1.1.2.1).
\end{enumerate}
\item \textbf{Postcondizione}: il sistema associa la voce selezionata al personaggio.
\end{itemize}

\subsection{Caso d'uso UC1.1.2: Scrittura battute}

\begin{figure}[htbp]
\centering
\includegraphics[scale=0.5]{UseCase_17_03_2016/immagini/uc_1_1_2_scrittura_battute.png}
\captionsetup{labelfont=bf}
\caption{Caso d'uso UC1.1.2}
\end{figure}

\begin{itemize}
\item \textbf{Attore}: Utente;
\item \textbf{Scopo e descrizione}: l'utente può digitare il testo di una nuova battuta e assegnarla ad un personaggio;
\item \textbf{Precondizione}: il sistema è pronto a ricevere il testo per la creazione di una nuova battuta;
\item \textbf{Flusso principale degli eventi}:
\begin{enumerate}
\item L'utente scrive il testo della nuova battuta (UC1.1.2.3);
\item L'utente naviga tra i personaggi disponibili (UC1.1.2.1);
\item L'utente seleziona il personaggio a cui associare la battuta (UC1.1.2.2).
\end{enumerate} 
\item \textbf{Postcondizione}: \'E stata creata una nuova battuta;
\end{itemize}

\subsection{Caso d'uso UC1.1.3: Associazione battuta-sentimento}

\begin{figure}[htbp]
\centering
\includegraphics[scale=0.5]{UseCase_17_03_2016/immagini/uc_1_1_3_associazione_battuta_sentimento.png}
\captionsetup{labelfont=bf}
\caption{Caso d'uso UC1.1.3}
\end{figure}

\begin{itemize}
\item \textbf{Attore}: Utente;
\item \textbf{Scopo e descrizione}: l'utente, dopo aver creato una battuta, può decidere di associarle una sentimento navigando tra quelli disponibili;
\item \textbf{Precondizione}: una battuta è stata creata;
\item \textbf{Flusso principale degli eventi}:
\begin{enumerate}
\item L'utente seleziona la battuta (UC1.1.3.1);
\item L'utente naviga tra i sentimenti disponibili (UC1.1.3.2);
\item L'utente seleziona il sentimento (UC1.1.3.3).
\end{enumerate}
\item \textbf{Postcondizione}: è stato assegnato un sentimento alla battuta selezionata.
\end{itemize}

\subsection{Caso d'uso UC1.1.4: Modifica battuta}

\begin{figure}[htbp]
\centering
\includegraphics[scale=0.5]{UseCase_17_03_2016/immagini/uc_1_1_4_modifica_battuta.png}
\captionsetup{labelfont=bf}
\caption{Caso d'uso UC1.1.4}
\end{figure}

\begin{itemize}
\item \textbf{Attore}: Utente;
\item \textbf{Scopo e descrizione}: l'utente può modificare una battuta già creata cambiandone il testo o il sentimento associato o il personaggio; 
\item \textbf{Precondizione}: la battuta è stata creata correttamente;
\item \textbf{Flusso principale degli eventi}:
\begin{enumerate}
\item L'utente seleziona la battuta da modificare (UC1.1.4.1);
\item L'utente può scegliere se cambiarne il testo, il sentimento associato o il relativo personaggio (UC1.1.4.2 - UC1.1.4.3 - UC1.1.4.4).
\end{enumerate}
\item \textbf{Postcondizione}: la modifica è avvenuta correttamente.
\end{itemize}

\subsection{Caso d'uso UC1.2: Condivisione}

\begin{figure}[htbp]
\centering
\includegraphics[scale=0.5]{UseCase_17_03_2016/immagini/uc_1_2_condivisione.png}
\captionsetup{labelfont=bf}
\caption{Caso d'uso UC1}
\end{figure}

\begin{itemize}
\item \textbf{Attori}: Utente, Sistema;
\item \textbf{Scopo e descrizione}: L'utente può condividere il file di esportazione relativo allo sceneggiato creato scegliendo uno tra gli strumenti di condivisione reperiti dal sistema;
\item \textbf{Precondizione}: l'esportazione dell'audio è andata a buon fine;
\item \textbf{Flusso principale degli eventi}:
\begin{enumerate}
\item Il sistema reperisce la lista degli strumenti di condivisione presenti nel dispositivo (UC1.2.3);
\item L'utente può navigare tra gli elementi della suddetta lista (UC1.2.1);
\item L'utente scegli uno tra gli elementi della lista (UC1.2.2).
\end{enumerate}
\item \textbf{Scenari alternativi}: il dispositivo non possiede alcuno strumento di condivisione;  
\item \textbf{Postcondizione}: l'audio verrà condiviso correttamente; 
\end{itemize}

\subsection{Caso d'uso UC1.3: Esportazione file audio}

\begin{figure}[htbp]
\centering
\includegraphics[scale=0.5]{UseCase_17_03_2016/immagini/uc_1_3_esportazione_audio_finale.png}
\captionsetup{labelfont=bf}
\caption{Caso d'uso UC1.3}
\end{figure}

\begin{itemize}
\item \textbf{Attore}: Utente;
\item \textbf{Scopo e descrizione}: l'utente digita il nome del file da generare e ne sceglie il formato di esportazione, 
\item \textbf{Precondizione}: il sistema è pronto ad esportare uno sceneggiato correttamente formato;
\item \textbf{Flusso principale degli eventi}:
\begin{enumerate}
\item L'utente digita il nome del file di esportazione (UC1.3.2);
\item L'utente seleziona il formato desiderato (UC1.3.1).
\end{enumerate}
\item \textbf{Scenari alternativi}: l'utente dcide di salvare il file con il nome di default; 
\item \textbf{Postcondizione}: viene effettuata l'esportazione nel formato desiderato.  
\end{itemize}

\subsection{Caso d'uso UC1.4: Salvataggio file}

\begin{figure}[htbp]
\centering
\includegraphics[scale=0.5]{UseCase_17_03_2016/immagini/uc_1_4_salvataggio_file.png}
\captionsetup{labelfont=bf}
\caption{Caso d'uso UC1.4}
\end{figure}

\begin{itemize}
\item \textbf{Attori}: Utente, Sistema;
\item \textbf{Scopo e descrizione}: l'utente può salvare le modifiche apportate ad uno sceneggiato; il Sistema provvederà al slavataggio effettivo reperendo la cartella di destinazione impostata di default;
\item \textbf{Precondizione}: il sistema è pronto ad effettuare il salvataggio;
\item \textbf{Flusso principale degli eventi}:
\begin{enumerate}
\item Il sistema reperisce la cartella di default (UC1.4.1);
\item Il sistema effettua il salvataggio (UC1.4.2).
\end{enumerate}
\item \textbf{Scenari alternativi}: il sistema avvertirà l'utente in caso il salvataggio non andasse a buon fine a causa di mancanza di memoria disponibile;  
\item \textbf{Postcondizione}: il salvataggio viene effettuato correttamente. 
\end{itemize}

\subsection{Caso d'uso UC1.7: Apertura file}

\begin{figure}[htbp]
\centering
\includegraphics[scale=0.5]{UseCase_17_03_2016/immagini/uc_1_7_apertura_file.png}
\captionsetup{labelfont=bf}
\caption{Caso d'uso UC1.4}
\end{figure}

\begin{itemize}
\item \textbf{Attori}: Utente;
\item \textbf{Scopo e descrizione}: l'utente, navigando nel filesystem, può selezionare un file da aprire;
\item \textbf{Precondizione}: il programma è avviato;
\item \textbf{Flusso principale degli eventi}:
\begin{enumerate}
\item L'utente naviga nel filesystem (UC1.7.0.1);
\item L'utente seleziona il file da aprire (UC.1.7.0.2);
\item L'utente conferma l'apertura del file selezionato (UC1.7.0.3).
\end{enumerate}
\item \textbf{Scenari alternativi}: il file selezionato non è conforme alla tipologia di file necessario;
\item \textbf{Postcondizione}: il file viene aperto correttamente;
\end{itemize}

\subsection{Caso d'uso UC1.8: Modifica sceneggiato}

\begin{figure}[htbp]
\centering
\includegraphics[scale=0.5]{UseCase_17_03_2016/immagini/uc_1_8_modifica_sceneggiato.png}
\captionsetup{labelfont=bf}
\caption{Caso d'uso UC1.8}
\end{figure}

\begin{itemize}
\item \textbf{Attori}: Utente;
\item \textbf{Scopo e descrizione}: l'utente può aprire uno sceneggiato già creato per modificarlo;
\item \textbf{Precondizione}: esiste almeno uno sceneggiato creato in precedenza;
\item \textbf{Flusso principale degli eventi}:
\begin{enumerate}
\item L'utente naviga nella lista degli sceneggiati già creati;
\item L'utente apre lo sceneggiato da modificare (UC1.7.2);
\item L'utente apporta le modifiche necessarie (UC1.1.4);
\end{enumerate}
\item \textbf{Scenari alternativi}: non esiste alcuno sceneggiato modificabile
\item \textbf{Postcondizione}: lo sceneggiato viene modificato correttamente.
\end{itemize}

\subsection{Caso d'uso UC2: Configurazione alto livello}

\begin{figure}[htbp]
\centering
\includegraphics[scale=0.5]{UseCase_17_03_2016/immagini/uc_2_configurazione_alto_livello.png}
\captionsetup{labelfont=bf}
\caption{Caso d'uso UC2}
\end{figure}

\begin{itemize}
\item \textbf{Attori}: Utente;
\item \textbf{Scopo e descrizione}: dopo la corretta esecuzione della applicazione di configurazione, l'utente potrà creare dei nuovi preset, modificando parametri ed effetti; inoltre potrà selezionare una nuova voce da impostare per il sistema e anche campionare la sua voce;
\item \textbf{Precondizione}: l'applicazione è stata avviata correttamente;
\item \textbf{Flusso principale degli eventi}:
\begin{enumerate}
\item L'utente può creare un nuovo preset o modificarne uno già creato (UC2.1 - UC2.5);
\item L'utente può testare il preset creato (UC2.4); 
\item L'utente può selezionare una voce per il sistema (UC2.2);
\item L'utente può campionare la sua voce (UC2.3);
\end{enumerate}
\item \textbf{Postcondizione}: l'applicazione si aspetta direttive dall'utente.
\end{itemize}


\subsection{Caso d'uso UC2.1: Creazione preset}

\begin{figure}[htbp]
\centering
\includegraphics[scale=0.5]{UseCase_17_03_2016/immagini/uc_2_1_creazione_preset.png}
\captionsetup{labelfont=bf}
\caption{Caso d'uso UC2.1}
\end{figure}

\begin{itemize}
\item \textbf{Attori}: Utente, Modulo di sistema;
\item \textbf{Scopo e descrizione}: Dopo che il modulo di sistema ha reperito gli effetti disponibili, l'utente potrà navigare tra questi ultimi, selezionarli e associarli a voci; potrà in fine salvare i cambiamenti apportati;
\item \textbf{Precondizione}: il sistema ha reperito gli effetti disponibili;
\item \textbf{Flusso principale degli eventi}:
\begin{enumerate}
\item Il modulo reperisce gli effetti (UC3.2);
\item L'utente può navigare tra gli effetti disponibili reperiti (UC2.1.1);
\item L'utente può settare dei parametri (UC2.1.2);
\item L'utente può salvare il preset creato.
\end{enumerate}
\item \textbf{Postcondizione}: è stato salvato un preset creato dall'utente.
\end{itemize}

\subsection{Caso d'uso UC2.2: Selezione voce per il sistema}

\begin{figure}[htbp]
\centering
\includegraphics[scale=0.5]{UseCase_17_03_2016/immagini/uc_2_2_selezione_voce_sistema.png}
\captionsetup{labelfont=bf}
\caption{Caso d'uso UC2.2}
\end{figure}

\begin{itemize}
\item \textbf{Attori}: Utente;
\item \textbf{Scopo e descrizione}: l'utente può impostare una delle voci disponibili come voce di default per il sistema;
\item \textbf{Precondizione}: esistono voci disponibili;
\item \textbf{Postcondizione}: la voce è stata associata correttamente.
\end{itemize}

\subsection{Caso d'uso UC2.3: Campionamento voce}

\begin{figure}[htbp]
\centering
\includegraphics[scale=0.5]{UseCase_17_03_2016/immagini/uc_2_3_campionamento_voce.png}
\captionsetup{labelfont=bf}
\caption{Caso d'uso UC2.3}
\end{figure}

\begin{itemize}
\item \textbf{Attori}: Utente, Applicazione;
\item \textbf{Scopo e descrizione}: l'utente può campionare la sua voce  eseguendo una procedura di lettura e registrazione di frasi che verranno inviate al server dall'applicazione;
\item \textbf{Precondizione}: il sistema è pronto per il campionamento;
\item \textbf{Flusso principale degli eventi}:
\begin{enumerate}
\item L'utente visualizza la frase da leggere (UC2.3.1);
\item L'utente legge la frase che viene registrata (UC2.3.2);
\item L'applicazione invia la registrazione al server per il campionamento, per poi ricominciare dal punto 1 mostrando all'utente una nuova frase da leggere (UC2.3.3);
\end{enumerate}
\item \textbf{Postcondizione}: il campionamento è andato a buon fine.
\end{itemize}

\subsection{Caso d'uso UC2.4: Test preset}

\begin{figure}[htbp]
\centering
\includegraphics[scale=0.5]{UseCase_17_03_2016/immagini/uc_2_4_test_preset.png}
\captionsetup{labelfont=bf}
\caption{Caso d'uso UC2.4}
\end{figure}

\begin{itemize}
\item \textbf{Attori}: Utente;
\item \textbf{Scopo e descrizione}: l'utente può ascoltare il preset creato;
\item \textbf{Precondizione}: il preset è stato creato correttamente;
\item \textbf{Postcondizione}: il preset viene ascoltato.
\end{itemize}

\subsection{Caso d'uso UC2.5: Modifica preset}

\begin{figure}[htbp]
\centering
\includegraphics[scale=0.5]{UseCase_17_03_2016/immagini/uc_2_5_modifica_preset.png}
\captionsetup{labelfont=bf}
\caption{Caso d'uso UC2.5}
\end{figure}

\begin{itemize}
\item \textbf{Attori}: Utente;
\item \textbf{Scopo e descrizione}: l'utente può aprire un preset già creato e modificarlo;
\item \textbf{Precondizione}: viene selezionato un preset correttamente creato; 
\item \textbf{Flusso principale degli eventi}:
\begin{enumerate}
\item L'utente seleziona il preset da modificare (UC2.5.1);
\item L'utente modifica i parametri del preset (UC2.5.2);
\item L'utente salva le modifiche apportate (UC2.5.3).
\end{enumerate}
\item \textbf{Postcondizione}: il preset viene modificato e salvato correttamente.
\end{itemize}

\subsection{Caso d'uso UC3: Modulo di sistema alto livello}

\begin{figure}[htbp]
\centering
\includegraphics[scale=0.5]{UseCase_17_03_2016/immagini/uc_3_modulo_sistema_alto_livello.png}
\captionsetup{labelfont=bf}
\caption{Caso d'uso UC3}
\end{figure}

\begin{itemize}
\item \textbf{Attori}: Modulo di sistema;
\item \textbf{Scopo e descrizione}: il modulo di sistema può reperire gli effetti e le voci disponibili comunicando con il server;
\item \textbf{Precondizione}: il modulo è pronto ad operare;
\item \textbf{Postcondizione}: il modulo svolge quanto richiesto;
\end{itemize}

\subsection{Caso d'uso UC3.3: Comunicazione server}

\begin{figure}[htbp]
\centering
\includegraphics[scale=0.5]{UseCase_17_03_2016/immagini/uc_3_3_comunicazione_server.png}
\captionsetup{labelfont=bf}
\caption{Caso d'uso UC3.3}
\end{figure}

\begin{itemize}
\item \textbf{Attori}: Modulo di sistema, Utente;
\item \textbf{Scopo e descrizione}: il modulo di sistema crea una richiesta http, la invia al server e riceve da quest'ultimo una risposta;
\item \textbf{Precondizione}: il modulo è pronto a comunicare con il server;
\item \textbf{Flusso principale degli eventi}:
\begin{enumerate}
\item Il modulo crea una richiesta http (UC3.1.1);
\item Il modulo invia una richiesta http creata (UC3.1.2);
\item Il modulo riceve una risposta dal server (UC3.1.3).
\end{enumerate}
\item \textbf{Estensioni}: nel caso in cui non fosse possibile comunicare con il server a causa di assenza di connessione, il sistema avverte l'utente dell'errore;  
\item \textbf{Postcondizione}: la comunicazione va a buon fine;
\end{itemize}

\subsection{Caso d'uso UC3.5: Selezione preset}

\begin{figure}[htbp]
\centering
\includegraphics[scale=0.5]{UseCase_17_03_2016/immagini/uc_3_5_selezione_preset.png}
\captionsetup{labelfont=bf}
\caption{Caso d'uso UC3.5}
\end{figure}

\begin{itemize}
\item \textbf{Attori}: Modulo di sistema;
\item \textbf{Scopo e descrizione}: il modulo di sistema salva il preset selezionato su file di impostazioni;
\item \textbf{Precondizione}: esiste almeno un preset disponibile;
\item \textbf{Postcondizione}: il preset è stato selezionato e impostato.
\end{itemize}