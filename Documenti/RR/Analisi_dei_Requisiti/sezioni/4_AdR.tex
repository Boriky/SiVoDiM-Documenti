\section{Requisiti}
\textbf{introduzione}..

\subsection{Requisiti funzionali}
\begin{center}
\def\arraystretch{1.6}
\bgroup
\begin{longtable}{| p{2.5cm} | p{3cm} | p{5.25cm} | p{2cm} |}
\hline
\textbf{Requisito} & \textbf{Tipologia} & \textbf{Descrizione} & \textbf{Fonti}\\ \hline \hline 

R0F1 & Funzionale \newline Obbligatorio &
Il programma deve permettere la creazione di sceneggiati & 
Capitolato \newline UC1 \newline UC1.1\\ \hline

R0F1.6 & Funzionale \newline Obbligatorio & L'utente deve poter assegnare un titolo allo sceneggiato & UC1.1.6 \\ \hline

R1F1.6.1 & Funzionale \newline Desiderabile & L'utente deve poter modificare il titolo ad uno sceneggiato & UC1.1.6 \newline UC1.8 \\ \hline 

R0F1.1 & Funzionale \newline Obbligatorio & L'utente deve poter creare un nuovo personaggio per lo sceneggiato & UC1.1.1 \\ \hline

R0F1.1.1 & Funzionale \newline Obbligatorio & L'utente deve poter assegnare un nome al personaggio creato & UC1.1.1.1 \\ \hline

R1F1.1.2 & Funzionale \newline Desiderabile & L'utente deve poter modificare il nome associato ad un personaggio & UC1.1.1.1 \newline UC1.8 \\ \hline 

R0F1.1.3 & Funzionale \newline Obbligatorio & L'utente deve poter assegnare un avatar al personaggio creato & UC1.1.1.3 \\ \hline

R1F1.1.4 & Funzionale \newline Desiderabile & L'utente deve poter cambiare l'avatar al personaggio creato & UC1.1.1.3 \\ \hline

R0F1.1.1.3 & Funzionale \newline Obbligatorio & L'utente deve poter aprire un'immagine & UC1.7.1 \newline UC1.1.1.3 \\ \hline


\caption{Requisiti funzionali}
\end{longtable}

\egroup
\end{center}