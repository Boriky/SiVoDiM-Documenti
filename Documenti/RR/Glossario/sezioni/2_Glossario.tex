\section{Definizioni}
\begin{description} \item[Android]  $\\$ $\\$ Sistema operativo per dispositivi mobili sviluppato da Google Inc. e basato su kernel Linux. Sito ufficiale: \url{https://www.android.com/intl/it_it/}. \\  \item[API]  $\\$ $\\$ \textit{Application Programming Interface} - Insieme di processi, protocolli e strumenti per sviluppare \textit{software}. La finalità è ottenere un'astrazione a più alto livello, di solito tra l'\textit{hardware} e il programmatore o tra \textit{software} a basso e alto livello, semplificando così il lavoro di programmazione. Le API permettono infatti di evitare di riscrivere ogni volta tutte le funzioni necessarie al programma rientrando quindi nel più vasto concetto di riuso del codice. 
 \\  \item[Astah]  $\\$ $\\$ Uno strumento \textit{software} di modellazione di 
 grafici UML\G\ sviluppato dall'azienda giapponese Change Vision. Sito 
 ufficiale: \url{http://astah.net/}. \\  \item[Avatar]  $\\$ $\\$ 
 L'\textit{avatar} è un'immagine scelta per rappresentare un'identità, spesso 
 usata in comunità virtuali, forum e giochi online per identificare ciascun 
 utente. Nel progetto il termine si riferisce ad un'immagine che è possibile 
 associare a ogni personaggio dello sceneggiato. \\  \newpage \item[Baseline]  
 $\\$ $\\$ Contestualmente alle \textit{milestone}\G, è un punto di avanzamento 
 certo che non consente di retrocedere, ma solo di avanzare. Viene inoltre 
 utilizzato, come modello di confronto, per misurare l'avanzamento del progetto 
 nelle diverse fasi di lavoro. \\  \item[Best-practice]  $\\$ $\\$ Modo di 
 operare che per esperienza e per studio abbia dimostrato garantire i migliori 
 risultati in circostanze note e specifiche. Viene usata per descrivere il 
 processo di sviluppo e segue uno standard utilizzabile da più organizzazioni. 
 Una best-practice è inoltre una \textit{feature}\G\ contenuta negli standard 
 ISO-9000 e ISO-14001. \\  \item[Beta testing]  $\\$ $\\$ Si riferisce ad una 
 fase di collaudo del \textit{software} non ancora pubblicato, con lo scopo di 
 trovare errori (BUG\G). È l'ultima fase di test prima del lancio del prodotto. 
 La versione beta del \textit{software} dovrebbe essere destinata ad un gruppo 
 limitato di \textit{tester} esterno al \textit{team} di programmatori. \\  
 \item[BOM]  $\\$ $\\$ Il \textit{Byte Order Mark} (BOM) è una piccola sequenza 
 di \textit{byte} che viene posizionata all'inizio di un flusso di dati di puro 
 testo per indicarne il tipo di codifica \textit{Unicode}. Quando si è noto che 
 un file o altra sequenza di dati è di testo e non binario, il BOM permette di 
 identificare subito se il testo è in formato Unicode e, in caso affermativo, 
 il tipo esatto di codifica. Ciò è utile quando non si conosce a priori la 
 codifica utilizzata; se invece questa è sempre nota, i byte del BOM possono 
 risultare inutili o addirittura dannosi. \\  \item[Branch]  $\\$ $\\$ 
 Duplicazione di un oggetto sottoposto a versionamento, così che delle 
 modifiche possano avvenire parallelamente su più copie. \\  \item[BUG]  $\\$ 
 $\\$ Identifica un errore nella scrittura o nella progettazione di un 
 programma \textit{software}. Un bug di un programma è un errore o guasto che 
 porta al malfunzionamento di esso (per esempio producendo un risultato 
 inatteso o errato). \\  \item[Business Management]  $\\$ $\\$ Attività legate 
 alla gestione di un'azienda, come il controllo, la direzione, l'organizzazione 
 e la pianificazione. \\  \newpage \item[Caching]  $\\$ $\\$ Tecnica che 
 utilizza un'area di memoria adibita a memorizzare temporaneamente dati recenti 
 con lo scopo di migliorare le prestazioni.  \\  \item[Capitolato]  $\\$ $\\$ 
 Atto allegato a un contratto d'appalto che intercorre tra il cliente ed una 
 ditta appaltatrice in cui vengono indicate modalità, costi e tempi di 
 realizzazione dell'opera oggetto del contratto. \\  \item[Ciclo di Deming]  
 $\\$ $\\$ Un modello studiato per il miglioramento continuo della qualità. Per 
 una spiegazione più approfondita si veda la sezione 4.2 Ciclo di Deming del 
 \textit{Piano di Qualifica}. \\  \item[Ciclo di vita]  $\\$ $\\$ Riferito al 
 \textit{software}, è una metodologia di sviluppo o un modello di processo che 
 scompone l'attività in un insieme di stati che il prodotto assume dal 
 concepimento al ritiro. \\  \item[Cloud]  $\\$ $\\$ Un modo di erogazione di 
 risorse informatiche, come archiviazione, elaborazione o trasmissione di dati 
 attraverso internet, caratterizzato dalla disponibilità su richiesta 
 attraverso internet a partire da un insieme di risorse preesistenti e 
 configurabili. \\  \item[Commit]  $\\$ $\\$ Un comando Git\G\ che si effettua 
 quando si copiano le modifiche fatte su \textit{file} locali nella 
 \textit{directory} del \textit{repository}\G. \\  \item[Crash]  $\\$ $\\$ Il 
 blocco o la terminazione improvvisa, non richiesta e inaspettata di un 
 programma in esecuzione, oppure il blocco completo dell'intero computer. 
 Tipicamente il \textit{crash} avviene per colpa di errori di progettazione o 
 programmazione presenti nel \textit{software}, oppure per un malfunzionamento 
 dell'\textit{hardware} che lo esegue. \\  \newpage \item[Database]  $\\$ $\\$ 
 Il termine \textit{database}, in italiano base di dati o banca dati (a volte 
 abbreviato con la sigla DB), indica un insieme organizzato di dati. Le 
 informazioni contenute in un \textit{database} sono strutturate e collegate 
 tra loro secondo un particolare modello logico scelto dal progettista del 
 \textit{database} (per esempio: relazionale, gerarchico, reticolare, a 
 oggetti). Gli utenti si interfacciano con i \textit{database} attraverso 
 \textit{query} (ossia interrogazioni) e grazie a particolari applicazioni 
 \textit{software} dedicate (DBMS). \\  \item[DB]  $\\$ $\\$ Si veda la voce: 
 \textit{Database}\G. \\  \item[Debian]  $\\$ $\\$ Un sistema operativo 
 multi-architettura per computer, composto interamente da \textit{software} 
 libero e basato su Linux\G. Sito ufficiale: 
 \url{https://www.debian.org/index.it.html}. \\  \item[Design Pattern]  $\\$ 
 $\\$ Si tratta di una descrizione o modello logico da applicare per la 
 risoluzione di un problema che può presentarsi in diverse situazioni durante 
 le fasi di progettazione e sviluppo del \textit{software}, ancor prima della 
 definizione dell'algoritmo risolutivo della parte computazionale. \\  
 \item[DMG]  $\\$ $\\$ I file con l'estensione .dmg sono immagini di disco che 
 si trovano in sistemi operativi OS X\G\ e che consentono di installare 
 programmi. \\  \item[Driver]  $\\$ $\\$ Una componente attiva e fittizia per 
 pilotare un test. \\  \item[Dropbox]  $\\$ $\\$ Un software di memorizzazione 
 in cloud\G, multipiattaforma\G, che offre un servizio di 	\textit{file 
 hosting}\G\ e sincronizzazione automatica di \textit{file} tramite web. \\  
 \item[DSL]  $\\$ $\\$ \textit{Domain specific language} - Nello sviluppo 
 \textit{software} è un linguaggio di programmazione o un linguaggio di 
 specifica dedicato a particolari problemi di un dominio, a una particolare 
 tecnica di rappresentazione e/o a una particolare soluzione tecnica. \\  
 \newpage \item[EmotionML]  $\\$ $\\$ \textit{Emotion Markup Language} - 
 Linguaggio di \textit{markup}\G\ creato per rappresentare testi  con delle 
 emozioni associate. Viene applicato in 3 diverse aree: annotazione manuale di 
 dati, riconoscimento di stati emotivi nel comportamento degli utenti e nella 
 generazione di comportamenti emotivi da parte dei sistemi. \\  
 \item[Encoding]  $\\$ $\\$ Consiste in un codice che associa un insieme di 
 caratteri (tipicamente rappresentazioni di grafemi così come appaiono in un 
 alfabeto utilizzato per comunicare in una lingua naturale) ad un insieme di 
 altri oggetti, come numeri (specialmente nell'informatica) o pulsazioni 
 elettriche, con lo scopo di facilitare la memorizzazione di un testo in un 
 computer o la sua trasmissione attraverso una rete di telecomunicazioni. \\  
 \item[Engine]  $\\$ $\\$ In informatica è usato per dare un modello mentale 
 delle componenti \textit{software} più complesse rispetto ai moduli ordinari. 
 \\  \item[Escaping]  $\\$ $\\$ Procedura nella quale vengono inseriti in un 
 testo i caratteri di \textit{escape} necessari. Ogni carattere di 
 \textit{escape} invoca un'interpretazione alternativa di una sotto-sequenza di 
 caratteri. \\  \newpage \item[FA-TTS]  $\\$ $\\$ Si veda la voce Flexible and 
 Adaptive Text-To-Speech\G. \\  \item[Fallback]  $\\$ $\\$ Scelta secondaria 
 che viene effettuata quando la scelta principale non è disponibile. \\  
 \item[Feature]  $\\$ $\\$ Caratteristica, peculiarità, qualità. In questo 
 contesto è una funzionalità messa a disposizione da un prodotto software. \\  
 \item[Filesystem]  $\\$ $\\$ Un meccanismo con il quale i \textit{file} sono 
 posizionati e organizzati o su un dispositivo di archiviazione o su una 
 memoria di massa, come un disco rigido. Più formalmente, un 
 \textit{filesystem} è l'insieme dei tipi di dati astratti necessari per la 
 memorizzazione (scrittura), l'organizzazione gerarchica, la manipolazione, la 
 navigazione, l'accesso e la lettura dei dati. Di fatto, alcuni 
 \textit{filesystem} non interagiscono direttamente con i dispositivi di 
 archiviazione. I \textit{filesystem} possono essere rappresentati sia 
 graficamente tramite \textit{file browser} sia testualmente tramite terminale 
 di testo. Nella rappresentazione grafica (GUI) è generalmente utilizzata la 
 metafora delle cartelle che contengono documenti (i \textit{file}) ed altre 
 sottocartelle. \\  \item[Flexible and Adaptive Text-To-Speech]  $\\$ $\\$ 
 Trattasi di un server TTS\G\ che consente la sintetizzazione di voci a partire 
 da un testo dato in input. La tecnologia utilizzata permette la manipolazione 
 di diversi parametri acustici e linguistici così da ottenere una voce 
 sintetizzata adatta ad ogni specifica situazione.\\ Fonte: 
 \url{http://lab.mediafi.org/discover-flexibleandadaptivetexttospeech-try.html}.
  \\  \item[Framework]  $\\$ $\\$ Un'architettura logica di supporto su cui un 
 \textit{software} può essere progettato e realizzato, spesso facilitandone lo 
 sviluppo. La sua funzione è quella di creare una infrastruttura generale, 
 lasciando al programmatore il contenuto vero e proprio dell'applicazione. Lo 
 scopo di un \textit{framework} è infatti quello di risparmiare allo 
 sviluppatore la riscrittura di codice già scritto in precedenza per compiti 
 simili. \\  \newpage \item[Git]  $\\$ $\\$ Un sistema \textit{software} di 
 controllo di versione distribuito, con un'enfasi sulla velocità, 
 l'integrazione dei dati e il supporto a flussi di lavoro distribuiti e non 
 lineari, creato da Linus Torvalds. Sito ufficiale: \url{https://git-scm.com/}. 
 \\  \item[GitHub]  $\\$ $\\$ Un servizio web di \textit{hosting} per lo 
 sviluppo di progetti \textit{software}, che usa il sistema di controllo di 
 versione Git\G. \\  \item[GPS]  $\\$ $\\$ Il Global Positioning System 
 (Sistema di posizionamento globale) è un sistema di posizionamento e 
 navigazione satellitare civile. \\  \item[Gulpease]  $\\$ $\\$ Un indice di 
 leggibilità di un testo, calibrato sulla lingua italiana, definito nel 1988 
 nell'ambito delle ricerche del GULP (Gruppo Universitario Linguistico 
 Pedagogico) presso il Seminario di Scienze dell'Educazione dell'Università  
 degli studi di Roma \textit{La Sapienza}. Utilizza la lunghezza delle parole 
 in lettere anziché in sillabe, semplificandone il calcolo automatico. \\  
 \newpage \item[Hashtag]  $\\$ $\\$ Un tipo di etichetta o 	\textit{tag}\G\ 
 per metadati utilizzato su alcuni servizi di rete e \textit{social network}\G\ 
 come aggregatore tematico. \\  \item[Hosting]  $\\$ $\\$ Un servizio di rete 
 che consiste nell'allocare su un \textit{server} le pagine \textit{web} di un 
 sito o un'applicazione, rendendolo così accessibile dalla rete Internet ai 
 suoi utenti. \\  \item[HTML5]  $\\$ $\\$ Un linguaggio di \textit{markup}\G\ 
 per la strutturazione delle pagine, web pubblicato come W3C Recommendation da 
 ottobre 2014. Sito ufficiale: \url{https://www.w3.org/TR/html5/}. \\  
 \item[HTTP]  $\\$ $\\$ L'\textit{HyperText Transfer Protocol} (protocollo di 
 trasferimento di un ipertesto) è un protocollo a livello applicativo usato 
 come principale sistema per la trasmissione di informazioni sul \textit{web}. 
 Le specifiche del protocollo sono gestite dal 
 W3C\G(\url{https://www.w3.org/Protocols/}). \\  \newpage \item[IEC]  $\\$ $\\$ 
 \textit{International Electrotechnical Commission} - Un'organizzazione 
 internazionale per la definizione di \textit{standard} in materia di 
 elettronica, elettricità e tecnologie correlate. \\  \item[Indoor]  $\\$ $\\$ 
 Termine inglese utilizzato per indicare eventi che si svolgono all'interno di 
 edifici o di strutture. \\ \item[iOS]  $\\$ $\\$ Un sistema operativo 
 sviluppato da Apple per dispositivi iPhone, iPod touch e iPad. Sito ufficiale: 
 \url{http://www.apple.com/it/ios/}. \\ \item[IPS]  $\\$ $\\$ \textit{Indoor 
 positioning system} - Un sistema per localizzare oggetti o persone all'interno 
 di edifici utilizzando onde radio, campi magnetici, segnali acustici o altre 
 informazioni raccolte da dispositivi mobili. \\  \item[ISO]  $\\$ $\\$ 
 \textit{International Organization for Standardization} - La più importante 
 organizzazione a livello mondiale per la definizione di norme tecniche. Sito 
 ufficiale: \url{http://www.iso.org/iso/home.html}. \\  \newpage \item[JAR]  
 $\\$ $\\$ L'estensione JAR \text{(Java Archive)} indica un archivio dati 
 compresso usato per distribuire raccolte di classi Java. \\  \item[Json]  $\\$ 
 $\\$ Un formato adatto all'interscambio di dati fra applicazioni 
 \textit{client-server} basato sul linguaggio JavaScript. \\  \newpage 
 \item[Key-value]  $\\$ $\\$ Paradigma di memorizzazione di dati in un database 
 non relazionale, che prevede l'associazione di chiavi a vettori di dati 
 organizzati.  \\  \newpage \item[LaTeX]  $\\$ $\\$ Un linguaggio di 
 \textit{markup}\G\ usato per la preparazione di testi basato sul programma di 
 composizione tipografica TEX. Sito ufficiale: 
 \url{https://www.latex-project.org/}. \\  \item[Libreria]  $\\$ $\\$ Insieme 
 di funzioni o strutture dati predefinite e predisposte per essere collegate ad 
 un programma software. \\  \item[Linguaggio di markup]  $\\$ $\\$ In generale 
 un linguaggio di \textit{markup} è un insieme di regole che descrivono i 
 meccanismi di rappresentazione (strutturali, semantici o presentazionali) di 
 un testo che, utilizzando convenzioni standardizzate, sono utilizzabili su più 
 supporti. \\  \item[Linux Mint]  $\\$ $\\$ Una distribuzione GNU/Linux per PC, 
 focalizzata sulla semplicità di installazione e facilità di utilizzo. Sito 
 ufficiale: \url{https://www.linuxmint.com/}. \\  \item[Logger]  $\\$ $\\$ Un 
 componente non intrusivo di registrazione dei dati di esecuzione per 
 un'analisi dei risultati. \\  \item[Login]  $\\$ $\\$ Il \textit{login} è un 
 termine utilizzato per indicare la procedura di accesso ad un sistema 
 informatico o ad un'applicazione informatica. \\  \newpage \item[MB]  $\\$ 
 $\\$ MegaByte - unità di misura delle informazioni o delle quantità di dati. 
 \\  \item[Microsoft PowerPoint]  $\\$ $\\$ Programma di presentazione prodotto 
 da Microsoft, fa parte della \textit{suite} di \textit{software} di 
 produttività personale Microsoft Office. Sito ufficiale: 
 \url{https://office.live.com/start/PowerPoint.aspx}. \\  \item[Microsoft 
 Project 2016]  $\\$ $\\$ Un \textit{software} di pianificazione per assistere 
 i \textit{Project Manager} nella pianificazione, nell'assegnazione delle 
 risorse, nella verifica del rispetto dei tempi, nella gestione dei 
 \textit{budget} e nell'analisi dei carichi di lavoro. \\  \item[Milestone]  
 $\\$ $\\$ Indica importanti punti di controllo intermedi nello svolgimento del 
 progetto. \\  \item[Modello di ciclo di vita]  $\\$ $\\$ Un insieme di 
 processi che scompongono la creazione di un prodotto. Viene reso indipendente 
 da metodi e strumenti di sviluppo, e ne precede la loro selezione. \\  
 \item[MongoDB]  $\\$ $\\$ Un database\G\ di tipo NoSQL\G\ orientato ai 
 documenti. Si allontana dalla struttura tradizionale basata su tabelle dei 
 database relazionali in favore di documenti in stile JSON\G\ con schema 
 dinamico, rendendo l'integrazione di dati di alcuni tipi di applicazioni più 
 facile e veloce. Sito ufficiale: \url{https://www.mongodb.org/}. \\  
 \item[Monospace]  $\\$ $\\$ Un tipo di carattere \textit{Unicode} monospazio, 
 che ricorda quello delle macchine da scrivere. \\  \item[MP3]  $\\$ $\\$ 
 \textit{Moving Picture Expert Group-1/2 Audio Layer 3} - Un algoritmo di 
 compressione audio in grado di ridurre drasticamente la quantità di dati 
 richiesti per memorizzare un suono, rimanendo comunque una riproduzione 
 accettabilmente fedele del \textit{file} originale non compresso. \\  
 \item[Multipiattaforma]  $\\$ $\\$ Un dispositivo \textit{software} o 
 \textit{hardware} che funziona su più di un sistema o piattaforma. \\  
 \item[MySQL]  $\\$ $\\$ Un sistema di gestione di \textit{database} basato sul 
 modello relazionale, composto da un terminale a riga di comando e un server. 
 MySQL è un software libero sviluppato da Oracle Corporation. Sito ufficiale: 
 \url{https://www.mysql.it/}. \\  \newpage \item[NoSQL]  $\\$ $\\$ Si riferisce 
 a quei linguaggi di progettazione di \textit{database}\G\ che non utilizzano 
 il modello relazionale per la rappresentazione dei dati. \\  \item[Notebooks]  
 $\\$ $\\$ Funzione di blocco note di Teamwork\G\ per gli appunti. \\  \newpage 
 \item[OAuth]  $\\$ $\\$ Protocollo che permette l'autorizzazione di API di 
 sicurezza con un metodo standard e semplice sia per applicazioni portatili che 
 per PC fisso e web. Per gli sviluppatori di applicazioni è un metodo per 
 pubblicare e interagire con dati protetti. OAuth garantisce ai service 
 provider l'accesso da parte di terzi ai dati degli utenti proteggendo 
 contemporaneamente le loro credenziali. Sito ufficiale: 
 \url{http://oauth.net/}. \\  \item[Open-source]  $\\$ $\\$ Indica un 
 \textit{software} di cui gli autori rendono pubblico il codice sorgente, 
 favorendone il libero studio e permettendo a programmatori indipendenti di 
 apportarvi modifiche ed estensioni. \\  \item[OS X]  $\\$ $\\$ Il sistema 
 operativo sviluppato da Apple per i computer Macintosh. \\  \newpage 
 \item[Pattern]  $\\$ $\\$ Indica una regolarità che si riscontra all'interno 
 di un insieme di oggetti osservati. \\  \item[PDCA]  $\\$ $\\$ \textit{Plan Do 
 Check Act} - Si veda la voce Ciclo di Deming \G. \\  \item[PHP]  $\\$ $\\$ Un 
 linguaggio di \textit{scripting} interpretato, originariamente concepito per 
 la programmazione di pagine web dinamiche, sviluppato da Php Group. Sito 
 ufficiale: \url{http://php.net/}. \\  \item[PNG]  $\\$ $\\$ \textit{Portable 
 Network Graphics} (PNG) è un formato di file per memorizzare immagini, capace 
 di immagazzinarle in modo \textit{lossless}, ossia senza perdere alcuna 
 informazione. Inoltre è più efficiente con immagini non fotorealistiche (che 
 contengono troppi dettagli per essere compresse in poco spazio). \\  
 \item[Preset]  $\\$ $\\$ Insieme di configurazioni automaticamente assegnate 
 ad un determinato componente \textit{software}. \\  \item[Prodotto 
 videoludico]  $\\$ $\\$ Un gioco gestito da un dispositivo elettronico che 
 consente di interagire con le immagini di uno schermo. \\  
 \item[Programmazione]  $\\$ $\\$ Indica l'insieme di attività e tecniche che 
 una o più persone specializzate svolgono per creare un programma, scrivendone 
 il relativo codice sorgente in un certo linguaggio di programmazione. \\  
 \item[Project Management]  $\\$ $\\$ Si intende l'insieme di attività svolte 
 tipicamente da una figura dedicata e specializzata detta \textit{Project 
 Manager} volte all'analisi, progettazione, pianificazione e realizzazione 
 degli obiettivi di un progetto, gestendolo in tutte le sue caratteristiche e 
 fasi evolutive, nel rispetto di precisi vincoli (tempi, costi, risorse, scopi, 
 qualità). \\  \newpage \item[Screen reader]  $\\$ $\\$ Un'applicazione che 
 identifica e interpreta il testo mostrato sullo schermo di un computer, 
 presentandolo tramite sintesi vocale. \\  \item[Skype]  $\\$ $\\$ 
 \textit{Software} per la messaggistica istantanea che unisce le chat ad un 
 sistema di  telefonate. Sito ufficiale: \url{http://www.skype.com/it/}. \\  
 \item[Slider]  $\\$ $\\$ Componente grafico con il quale l'utente può 
 impostare un valore muovendo un indicatore. \\  \item[Smartphone]  $\\$ $\\$ 
 Telefono cellulare con capacità di calcolo, di memoria e di connessione dati 
 molto più avanzate rispetto ai normali telefoni cellulari, basato su un 
 sistema operativo per dispositivi mobili. \\  \item[Social network]  $\\$ $\\$ 
 Una rete sociale che consente ad un gruppo di individui di restare connessi 
 tra loro attraverso diversi legami sociali. \\  \item[SPICE]  $\\$ $\\$ 
 \textit{Software Process Improvement and Capability Determination} - Un 
 insieme di standard tecnici per lo sviluppo di \textit{software} e il relativo 
 \textit{business managment\G}. \\  \item[SSML]  $\\$ $\\$ Acronimo di 
 \textit{Speech Synthesis Markup Language}. \'E un linguaggio di 
 \textit{markup\G} raccomandato dal \textit{W3C\G} per assistere la sintesi 
 vocale su applicazioni di qualsiasi tipo. Permette il controllo del volume, 
 della pronuncia, del rate e altri aspetti sulla maggior parte delle 
 piattaforme di sintesi presenti sul mercato. Maggiori informazioni si possono 
 trovare su \url{https://www.w3.org/TR/speech-synthesis/}  \\  \item[Stub]  
 $\\$ $\\$ \'E una componente passiva e fittizia utilizzata in fase di 
 \textit{testing} per simulare una parte del sistema. \\  \newpage \item[Tag]  
 $\\$ $\\$ Una parola chiave o un termine associato ad un'informazione che 
 descrive l'oggetto. \\  \item[Task]  $\\$ $\\$ Una suddivisione delle 
 attività, un compito che può essere svolto da una sola persona. \\  
 \item[Teamwork]  $\\$ $\\$ Teamwork è un servizio web per la gestione di 
 progetti e lavori di gruppo. Sito ufficiale: \url{https://www.teamwork.com/}. 
 \\  \item[Telegram]  $\\$ $\\$ Un servizio di messaggistica istantanea 
 multipiattaforma\G\ che mette a disposizione degli utenti varie funzionalità 
 come stabilire comunicazioni cifrate, scambiare messaggi vocali o inviare file 
 di qualsiasi dimensione fino a 1.500 MB\G. Sito ufficiale: 
 \url{https://telegram.org/}. \\  \item[Template]  $\\$ $\\$ Un documento nel 
 quale viene definita una struttura generica o \textit{standard}, che viene 
 utilizzato per la definizione di vari documenti con lo stesso formato. \\  
 \item[TeXstudio]  $\\$ $\\$ Un \textit{software} di scrittura per la creazione 
 di documenti LaTeX\G\ che mette a disposizione strumenti di supporto, come il 
 controllo dell'ortografia. Sito ufficiale \url{http://www.texstudio.org/}. \\  
 \item[Text-To-Speech]  $\\$ $\\$ Si tratta della sintesi vocale, ossia la 
 tecnica di convertire un testo in parlato, attraverso una voce artificiale. 
 \\  \item[Ticketing]  $\\$ $\\$ Sistema di assegnazione dei \textit{task}\G. 
 \\  \item[TTS]  $\\$ $\\$ Si veda la voce: Text-To-Speech\G. \\  \newpage 
 \item[Ubuntu]  $\\$ $\\$ Una distribuzione GNU/Linux basata su Debian, 
 focalizzata sulla facilità di utilizzo. Sito ufficiale: 
 \url{http://www.ubuntu-it.org/}. \\  \item[UC]  $\\$ $\\$ \textit{Use Case} - 
 Il caso d'uso in informatica è una tecnica usata nei processi di Ingegneria 
 del \textit{software} per effettuare in maniera esaustiva e non ambigua la 
 raccolta dei requisiti al fine di produrre \textit{software} di qualità. \\  
 \item[UML]  $\\$ $\\$ \textit{Unifying Model Language} - Famiglia di notazioni 
 grafiche che servono a supportare la descrizione e la progettazione dei 
 sistemi \textit{software}. Linguaggio visuale con sintassi e semantica 
 indipendente dai linguaggi di sviluppo e programmazione. Supporta l'intero 
 ciclo di vita del \textit{software}. Sito ufficiale: 
 \url{http://www.uml.org/}. \\  \item[Underscore]  $\\$ $\\$ È il carattere 
 trattino basso \_ . \\  \item[Upload]  $\\$ $\\$ Processo di invio o 
 trasmissione di un file da un \textit{client} ad un sistema remoto. \\  
 \item[UTF-8]  $\\$ $\\$ Una codifica dei caratteri \textit{Unicode} in 
 sequenze di lunghezza variabile di \textit{byte.} \\  \newpage \item[W3C]  
 $\\$ $\\$ Il \textit{World Wide Web Consortium}, anche conosciuto come W3C, è 
 un'organizzazione non governativa internazionale che ha come scopo quello di 
 sviluppare tutte le potenzialità del \textit{World Wide Web}. Al fine di 
 riuscire nel proprio intento, la principale attività svolta dal W3C consiste 
 nello stabilire standard tecnici per il \textit{World Wide Web} inerenti sia i 
 linguaggi di \textit{markup}\G\ che i protocolli di comunicazione (link al 
 sito ufficiale: \url{https://www.w3.org/}). \\  \item[WAV]  $\\$ $\\$ 
 Contrazione di WAVEform audio file format (formato audio per la forma d'onda) 
 è un formato audio di codifica digitale sviluppato da Microsoft e IBM.
 \\  \item[WBS]  $\\$ $\\$ \textit{Work Breakdown Structure} - Struttura di scomposizione del lavoro che elenca tutte le attività di un progetto. Viene usato nel \textit{Project ManagementG} per l'organizzazione delle attività. \\  \item[Windows]  $\\$ $\\$ Una famiglia di sistemi operativi dedicati ai PC, ai server e agli \textit{smartphone}. Sito ufficiale: \url{https://www.microsoft.com/it-it/windows}. \\  \item[Windows Phone]  $\\$ $\\$ Una famiglia di sistemi operativi creati da Microsoft per \textit{Smartphone}. Sito ufficiale: \url{https://www.microsoft.com/it-it/windows/phones}. \\ \end{description}