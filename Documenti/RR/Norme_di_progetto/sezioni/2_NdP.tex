\section{Comunicazione}
\subsection{Esterna}
Il Project Manager\G\ sarà la persona preposta a mantenere i contatti con 
individui esterni al gruppo; per rappresentare il gruppo è stato creato il 
seguente indirizzo di posta elettronica:
\begin{center}
	starklabs.swe@gmail.com
\end{center}
Tutti i componenti del gruppo possono accedere alla casella, tuttavia solo 
il Project Manager\G\ si incaricherà di inviare le comunicazioni con questo 
indirizzo 
e-mail. 
Tutte le e-mail ricevute alla casella sopra indicata verranno automaticamente 
inoltrate ai membri del gruppo.

\subsection{Interna}
Si utilizza Telegram\G\ per una comunicazione informale all'interno del 
gruppo. Inoltre Telegram\G\ fornisce il vantaggio di essere un'applicazione 
multi-piattaforma e disponibile anche in versione desktop/web. 

\subsection{Composizione e-mail}
In questo paragrafo viene descritta la struttura per un messaggio e-mail per 
una comunicazione esterna.

\subsubsection{Destinatario}
\begin{itemize}
	\item \textbf{Esterno}: i destinatari possono essere il Proponente, Giulio 
	Paci e l'azienda Mivoq, il Prof. Tullio Vardanega o il Prof. Riccardo 
	Cardin.
\end{itemize}

\subsection{Mittente}
\begin{itemize}
	\item \textbf{Esterno}: l'unico indirizzo utilizzabile è 
	starklabs.swe@gmail.com e deve essere usato solamente dal Project Manager.
\end{itemize}

\subsubsection{Oggetto}
L'oggetto deve contenere [UNIPD-TTS\G] e dovranno essere indirizzate 
all'attenzione del referente Giulio Paci, così come è stato specificato nel 
capitolato dell'azienda Proponente. Nel caso il messaggio sia una risposta è 
necessario aggiungere la particella “Re:” all'inizio dell'oggetto per 
distinguere il livello di risposta; se si dovesse trattare di un inoltro si 
deve usare la particella “I:”. L'oggetto non va mai cambiato.

\subsection{Corpo}
In caso di risposta da parte dell'azienda Mivoq o del fornitore
risulta utile la citazione della frase a cui si intende rispondere. Il modello 
per citare correttamente deve seguire le seguenti regole. Devono essere 
presenti data e ora della mail a cui si risponde, il nome del mittente, il suo 
indirizzo e-mail tra parentesi angolari, ad esempio: <starklabs.swe@gmail.com>, 
la dicitura “ha scritto:” e infine il testo con una parentesi angolare chiusa 
prima, “>testo di prova”. Se dovessero essere presenti alcune parti con uno o 
più destinatari specifici, il nome dovrà essere indicato all'inizio del 
paragrafo tramite la dicitura: \textit{@destinatario}.

\subsection{Allegati}
Qualora vi sia la necessità è data la possibilità di allegare alcuni file al 
messaggio e-mail. Possono essere usati per allegare il verbale di incontri con 
proponente e committente o i punti più importanti citati in una comunicazione 
Telegram\G.
