\section{Riunioni}
\subsection{Frequenza}
Le riunioni del gruppo di lavoro avranno una frequenza settimanale.

\subsection{Convocazione riunione}

\subsubsection {Interna}
Il Responsabile di Progetto ha il compito di convocare le riunioni generali, a 
cui dovranno partecipare tutti i membri del gruppo.
Su decisione del Responsabile di Progetto le riunioni possono coinvolgere anche 
solo specifici componenti del gruppo, a seconda del ruolo che si ritiene più 
utile in una data fase del progetto. Al termine di ogni riunione viene redatto 
un verbale.
Il responsabile deve convocare l'assemblea, con almeno un giorno di preavviso, 
attraverso l'invio di una mail a starklabs.swe@gmail.com contente:
\begin{itemize}
\item \textbf{Oggetto}: convocazione riunione n. X, dove X indica il numero 
crescente di 
riunioni effettuate.
\item \textbf{Corpo}: 
\begin{itemize}
	\item \textbf{Data}: data e ora prevista;
	\item \textbf{Luogo}: luogo previsto;
	\item \textbf{Tipo}: ordinaria/straordinaria;
	\item \textbf{Ordine del giorno}: elenco numerato delle voci da esaminare.
\end{itemize}
\end{itemize}

Ogni componente del gruppo deve rispondere al messaggio in modo più celere 
possibile, confermando la presenza o giustificando un'eventuale assenza. Il 
Responsabile di Progetto, in mancata risposta di uno o più membri nel tempo 
utile, ha il compito di contattarli telefonicamente. Una volta ricevute le 
risposte e verificata l'assenza o presenza dei membri richiesti, il 
Responsabile di Progetto ha la possibilità di decidere se confermare, o 
posticipare la riunione per permettere la presenza di tutti i membri convocati; 
tutte le eventuali modifiche dovranno essere notificate tramite e-mail. 

\subsubsection {Esterna}
Vengono seguite le stesse dinamiche esposte per la comunicazione delle riunioni 
interne, per quanto sia auspicabile una riunione plenaria, eventuali assenze 
dei componenti del gruppo non causeranno posticipazioni o spostamenti delle 
date di incontro, dovendo ovviamente considerare gli impegni dell'azienda 
Proponente.

\subsection{Svolgimento riunione}
\subsubsection{Esterna}
Ad ogni riunione verrà scelto, tra i membri presenti, un segretario che avrà il 
compito di annotare ogni argomento trattato e controllare che venga seguito 
l'ordine del giorno. Inoltre sarà suo compito redigere il verbale 
dell'assemblea ed inviarlo ai restanti componenti del gruppo.

\subsection{Verbale}
\subsubsection{Riunione interna}
Il verbale di riunione interna si presenta in forma di documento interno 
informale, per fissare i punti principali trattati e le soluzioni proposte.
Verrà redatto dal segretario della riunione, ruolo scelto a rotazione tra i 
presenti.
Redatto come documento testuale utilizzando la funzione Notebooks\G\ di 
TeamWork\G\, permettendo così la sua condivisione, tra tutti i membri del 
gruppo, di un documento sempre aggiornato all'ultima modifica.

\subsubsection{Riunione esterna}
In caso di riunione con il committente od il proponente, il verbale è un 
documento che assume carattere ufficiale, e quindi redatto secondo uno schema 
specifico.
Per agevolare la scrittura di tale documento viene utilizzato un template\G\ 
\LaTeX, per definire la struttura e organizzare i contenuti. Tale documento 
dovrà essere inviato come allegato in risposta all'e-mail di convocazione 
dell'assemblea e al Proponente Giulio Paci.