\section{Processi organizzativi}
\subsection{Gestione dei processi}
\subsubsection{Attività}
\paragraph{Gestione delle comunicazioni}
\subparagraph{Comunicazione interna}
Si utilizza Telegram\G\ per una comunicazione informale all'interno del 
gruppo, che inoltre fornisce il vantaggio di essere un'applicazione 
multi-piattaforma e disponibile anche in versione desktop/web. 

\subparagraph{Comunicazione esterna}
Il Project Manager\G\ sarà la persona preposta a mantenere i contatti con 
individui esterni al gruppo; per rappresentare il gruppo è stato creato il 
seguente indirizzo di posta elettronica:
\begin{center}
	starklabs.swe@gmail.com
\end{center}
Tutti i componenti del gruppo possono accedere alla casella, tuttavia solo 
il Project Manager\G\ si incaricherà di inviare le comunicazioni con questo 
indirizzo 
e-mail. 
Tutte le e-mail ricevute alla casella sopra indicata verranno automaticamente 
inoltrate ai membri del gruppo.

\subparagraph{Composizione email}
\begin{itemize}
\item \textbf{Destinatario:}
\begin{itemize}
\item \textbf{Esterno}: i destinatari possono essere il Proponente, Giulio 
	Paci e l'azienda Mivoq, il Prof. Tullio Vardanega o il Prof. Riccardo 
	Cardin.
\end{itemize}
\item \textbf{Mittente:}
\begin{itemize}
\item \textbf{Esterno:} l'unico indirizzo utilizzabile è 
	starklabs.swe@gmail.com e deve essere usato solamente dal Project Manager.
\end{itemize}

\item \textbf{Oggetto:} l'oggetto deve contenere [UNIPD-TTS\G] e dovranno essere indirizzate 
all'attenzione del referente Giulio Paci, così come è stato specificato nel 
capitolato dell'azienda Proponente. Nel caso il messaggio sia una risposta è 
necessario aggiungere la particella “Re:” all'inizio dell'oggetto per 
distinguere il livello di risposta; se si dovesse trattare di un inoltro si 
deve usare la particella “I:”. L'oggetto non va mai cambiato.

\item \textbf{Corpo:} in caso di risposta da parte dell'azienda Mivoq o del fornitore
risulta utile la citazione della frase a cui si intende rispondere. Il modello 
per citare correttamente deve seguire le seguenti regole. Devono essere 
presenti data e ora della mail a cui si risponde, il nome del mittente, il suo 
indirizzo e-mail tra parentesi angolari, ad esempio: <starklabs.swe@gmail.com>, 
la dicitura “ha scritto:” e infine il testo con una parentesi angolare chiusa 
prima, “>testo di prova”. Se dovessero essere presenti alcune parti con uno o 
più destinatari specifici, il nome dovrà essere indicato all'inizio del 
paragrafo tramite la dicitura: \textit{@destinatario}.

\item \textbf{Allegati:} qualora vi sia la necessità è data la possibilità di allegare alcuni file al 
messaggio e-mail. Possono essere usati per allegare il verbale di incontri con 
proponente e committente o i punti più importanti citati in una comunicazione 
Telegram\G.

\end{itemize}


\paragraph{Gestione delle riunioni}
\subparagraph{Riunioni interne}
\begin{itemize}
\item \textbf{Frequenza:} le riunioni del gruppo di lavoro avranno una frequenza settimanale; 

\item \textbf{Convocazione:} Il Responsabile di Progetto ha il compito di convocare le riunioni generali, a 
cui dovranno partecipare tutti i membri del gruppo.
Su decisione del Responsabile di Progetto le riunioni possono coinvolgere anche 
solo specifici componenti del gruppo, a seconda del ruolo che si ritiene più 
utile in una data fase del progetto. Al termine di ogni riunione viene redatto 
un verbale.
Il responsabile deve convocare l'assemblea, con almeno un giorno di preavviso, 
attraverso l'invio di una mail a starklabs.swe@gmail.com contente:
\begin{itemize}
\item \textbf{Oggetto}: convocazione riunione n. X, dove X indica il numero 
crescente di 
riunioni effettuate.
\item \textbf{Corpo}: 
\begin{itemize}
	\item \textbf{Data}: data e ora prevista;
	\item \textbf{Luogo}: luogo previsto;
	\item \textbf{Tipo}: ordinaria/straordinaria;
	\item \textbf{Ordine del giorno}: elenco numerato delle voci da esaminare.
\end{itemize}
\end{itemize}

Ogni componente del gruppo deve rispondere al messaggio in modo più celere 
possibile, confermando la presenza o giustificando un'eventuale assenza. Il 
Responsabile di Progetto, in mancata risposta di uno o più membri nel tempo 
utile, ha il compito di contattarli telefonicamente. Una volta ricevute le 
risposte e verificata l'assenza o presenza dei membri richiesti, il 
Responsabile di Progetto ha la possibilità di decidere se confermare, o 
posticipare la riunione per permettere la presenza di tutti i membri convocati; 
tutte le eventuali modifiche dovranno essere notificate tramite e-mail. 

\item \textbf{Verbale:} il verbale di riunione interna si presenta in forma di documento interno 
informale, per fissare i punti principali trattati e le soluzioni proposte. Verrà redatto come documento testuale utilizzando la funzione Notebooks\G\ di 
TeamWork\G\, permettendo così la sua condivisione, tra tutti i membri del 
gruppo, di un documento sempre aggiornato all'ultima modifica, dal segretario della riunione, ruolo scelto a rotazione tra i 
presenti. Sarà inoltre compito del segretario annotare ogni argomento trattato e controllare che venga seguito 
l'ordine del giorno.

\end{itemize}

\subparagraph{Riunioni esterne}
\begin{itemize}
\item \textbf{Convocazione}: vengono seguite le stesse dinamiche esposte per la comunicazione delle riunioni 
interne, per quanto sia auspicabile una riunione plenaria, eventuali assenze 
dei componenti del gruppo non causeranno posticipazioni o spostamenti delle 
date di incontro, dovendo ovviamente considerare gli impegni dell'azienda 
Proponente.

\item \textbf{Verbale}: in caso di riunione con il committente od il proponente, il verbale è un 
documento che assume carattere ufficiale, e quindi redatto secondo uno schema 
specifico.
Per agevolare la scrittura di tale documento viene utilizzato un template\G\ 
\LaTeX, per definire la struttura e organizzare i contenuti. Tale documento 
dovrà essere redatto e inviato come allegato in risposta all'e-mail di convocazione 
dell'assemblea e al Proponente Giulio Paci dal segretario scelto tra i membri presenti. 
\end{itemize}


\paragraph{Gestione del sistema dei task\G} Il sistema scelto per la gestione dei task\G\ è Teamwork\G, un'applicazione web di project management. Le viste presenti sono:
\begin{itemize}
\item \textbf{Dashboard}: dove vengono visualizzati i progetti attivi e le ultime notizie relative ad essi;

\item \textbf{Everything}: che consente di visualizzare i task, le \textit{milestone}\G, i file e filtrarli per data;

\item \textbf{Project}: permette di visualizzare la lista di tutti i propri progetti suddivisi per categoria e ne permette l'accesso;

\item \textbf{Calendar}: mostra un calendario per la gestione degli impegni e delle scadenza;

\item \textbf{Statuses}: consente di verificare gli stati dei propri collaboratori al progetto;

\item \textbf{People}: permette di visualizzare l'elenco dei singoli elementi del gruppo di lavoro e di accedere al loro profilo.    
\end{itemize}

Le funzionalità principali e più importanti si hanno una volta avuto l'accesso al progetto desiderato e sono le seguenti:
\begin{itemize}
\item Aggiunta di nuovi task\G, ed eventualmente di \textit{sub-task}\G, da associare ad uno o più membri del \textit{team}\G;

\item Assegnazione a ciascun \textit{task}\G\ di una data di inizio e di termine di fattibilità;

\item Aggiunta di nuove \textit{milestone}\G\ e relativi dettagli come responsabile, descrizione e data di scadenza;

\item \textit{Upload}\G\ di file potenzialmente utili al gruppo di lavoro;

\item Utilizzo di un blocco di note.
 
\end{itemize}

\paragraph{Gestione delle \textit{milestone}\G} 
Il \textit{Responsabile di Progetto} dovrà pianificare i punti di controllo che il \textit{team}\G deve raggiungere e assicurarsi che ogni \textit{task}\G\ necessario al suo soddisfacimento venga terminato entro la data stabilita.

\paragraph{Gestione dei \textit{task}\G} 
Sarà compito del \textit{Responsabile di Progetto} individuare ogni singolo task\G\ e, al seguito di un' accurata valutazione, assegnarlo al membro del gruppo più adatto. Dovrà inoltre assegnare una data di inizio e di scadenza di fattibilità. Tutte queste attività possono essere facilmente effettuate mediante l'interfaccia grafica di Teamwork\G.

\paragraph{Gestione dello svolgimento dei \textit{task}\G}
Ogni membro del gruppo di lavoro è tenuto ad accettare il \textit{task}\G\ assegnatigli dal \textit{Responsabile di Progetto} e fare quanto possibile per portarlo a termine entro la data di scadenza. Nel caso in cui l'assegnatario non fosse in grado di adempire al suo compito dovrà renderlo noto al \textit{Responsabile di Progetto} entro 24 ore dall'assegnazione del \textit{task}\G, altrimenti quest'ultimo verrà considerato come accettato; solo dopo un'accurata valutazione delle motivazioni riportate, il \textit{Responsabile di Progetto} dovrà provvedere a trovare un nuovo destinatario del task.

\subsubsection{Procedure}
\paragraph{Generazione di una milestone}
Il \textit{Responsabile di Progetto} dovrà eseguire i seguenti passi per generare una \textit{milestone}\G\ dopo aver avuto accesso al progetto a Teamwork\G:
\begin{itemize}
\item Cliccare sul pulsante \textit{"Add a milestone"};
\item Definire il titolo della \textit{milestone}\G;
\item Definire la data di scadenza;
\item Assegnare un responsabile.
\end{itemize}

\paragraph{Assegnazione di un task}
Il \textit{Responsabile di Progetto} dovrà eseguire i seguenti passi, riassunti in Figura 1, per generare una \textit{task}\G\ dopo aver avuto accesso al progetto a Teamwork\G:
\begin{itemize}
\item Cliccare sul pulsante \textit{"Add a task"};
\item Definire il titolo del \textit{task}\G;
\item Definire la data di inizio;
\item Definire la data di scadenza;
\item Assegnarlo ad uno o più membri del gruppo.
\end{itemize}

\begin{figure}[htbp]
\centering
\includegraphics[scale=0.5]{png/Procedura_di assegnazione_di_un_task.png}
\captionsetup{labelfont=bf}
\caption{Diagramma di attività - Procedura di assegnazione di un task}
\end{figure}

\paragraph{Svolgimento di un task}
Il membro assegnatario del \textit{task}\G, ricevuta la notifica e non avendo alcun impedimento, dovrà procedere secondo le seguenti direttive che vengono anche riportate in figura 2:
\begin{itemize}
\item Se il \textit{task}\G\ ricevuto ha una scadenza più immediata rispetto a quello su cui sta lavorando, dovrà sospendere lo svolgimento di quest'ultimo, metterlo in coda e dedicarsi al \textit{task}\G\ appena notificato;
\item Se, dopo aver iniziato lo svolgimento del \textit{task}\G, si riceve la notifica di uno nuovo con scadenza più immediata si procederà come riportato nel punto precedente;
\item Se si dovesse superare la data di scadenza prevista, si dovrà impostare il tag "Delay" dal sistema di Teamwork\G.\\
Questa situazione si può verificare se:
\begin{itemize}
\item Il tempo assegnato dal \textit{Responsabile di progetto} non era sufficiente al completamento del \textit{task}\G;
\item Il \textit{task}\G\ in ritardo sta alle dipendenze di un altro non ancora terminato;
\item L'assegnatario ha rallentamenti esterni non resi noti al \textit{Responsabile di Progetto};
\item Il committente non ha a disposizione tutte le conoscenze necessarie al corretto svolgimento del \textit{task}\G.\\
\'E compito del \textit{Responsabile di Progetto} fare in modo che i primi due casi non si verifichino.
\end{itemize}
\item Al completamento del lavoro l'assegnatario dovrà spuntare il \textit{task}\G\ dalla lista presente su Teamwork\G;
\item Proseguire con lo svolgimento dei \textit{task}\G\ rimanenti seguendo la procedura dal suo inizio.
\end{itemize}

\begin{figure}[htbp]
\centering
\includegraphics[scale=0.5]{png/Procedura_di svolgimento_di_un_task.png}
\captionsetup{labelfont=bf}
\caption{Diagramma di attività - Procedura di svolgimento di un task}

\end{figure}

\paragraph{Rilevazione dei rischi}
Sarà compito del \textit{Responsabile di Progetto} individuare i rischi trovati nel \textit{Piano di Progetto v1.0.0}.\\
Questa attività necessita di un continuo monitoraggio, in quanto è plausibile che insorgano nuovi rischi in seguito a quelli rilevati nella fase preliminare. In tal caso il \textit{Responsabile di Progetto} dovrà agire come segue:
\begin{itemize}
\item Registrare il resoconto effettivo dei rischi nel \textit{Piano di Progetto v1.0.0};
\item Pianificare per gestire i nuovo rischi;
\item Aggiornare le metodologie per far fronte alla nuova pianificazione;
\item Monitorare i nuovo rischi riscontrati durante lo sviluppo del progetto. 
\end{itemize}

\paragraph{Ruoli di Progetto}
Ogni componente del gruppo \textit{Stark Labs} dovrà ricoprire almeno una volta ciascuno dei ruoli necessari allo sviluppo del progetto.\\
Vengono ora presentate le diverse cariche, delineando per ciascuna le mansioni e le responsabilità.

\subparagraph{Responsabile di Progetto} Il \textit{Responsabile di Progetto} rappresenta il \textit{team}\G\ e il progetto verso il committente e il proponente; accentra le responsabilità di scelta e approvazione.\\
Detiene le seguenti responsabilità:
\begin{itemize}
\item Pianificazione e coordinamento delle attività;
\item Gestione e controllo delle risorse;
\item Analisi e gestione dei rischi;
\item Approvazione dei documenti;
\item assicurarsi che tutte le attività svolte siano conformi alle \textit{Norme di Progetto
v1.0.0} e rispettino la pianificazione effettuata nel \textit{Piano di Progetto v1.0.0} .
\end{itemize}  

\subparagraph{Amministratore di Progetto} L' \textit{Amministratore di Progetto} deve svolgere i seguenti compiti:
\begin{itemize}
\item Assicurarsi che tutte le risorse siano presenti e operanti; 
\item Deve garantire un'infrastruttura funzionale;
\item Fornire procedure che servono a garantire la qualità del prodotto uscente da un
determinato compito.
\end{itemize}

\subparagraph{Analista} L' \textit{Analista} deve:
\begin{itemize}
\item Tradurre il bisogno del cliente in una specifica utile a trovare una soluzione;
\item Comprendere la complessità del problema;
\item Capire il dominio nel quale lavora il cliente;
\item Analizzare il dominio applicativo e le specifiche per poi produrre i documenti di analisi.
\end{itemize}
\subsubsection{Norme}

\subparagraph{Progettista} Il \textit{Progettista} ha il compito di:
\begin{itemize}
\item Individuare la tecnologia più idonea per risolvere il problema indicato dall' \textit{Analista};
\item Descrivere il funzionamento interno del sistema a diversi livelli di dettaglio;
\item Produrre una soluzione comprensibile e attuabile. 
\end{itemize}

\subparagraph{Programmatore} Il \textit{Programmatore} ha responsabilità sulle attività di codifica perciò deve:
\begin{itemize}
\item Scrivere codice documentato, versionato e manutenibile;
\item Implementare le soluzioni descritte dal \textit{Progettista};
\item Implementare i test sul codice prodotto. 
\end{itemize}

\subparagraph{Verificatore} Il \textit{Verificatore} è il responsabile delle attività di verifica, ha quindi i seguenti compiti:
\begin{itemize}
\item Controllare che vengano rispettate le norme di progetto;
\item Assicurarsi la conformità di ogni stadio del ciclo di vita del prodotto.
\end{itemize}

\subsubsection{Strumenti}

\paragraph{Teamwork} Teamwork\G\ è l'applicazione web scelta per la gestione dei \textit{task}\G; permette anche di gestire un calendario dove inserire note o fissare appuntamenti e o traguardi importanti.
\paragraph{Astah} Astah\G\ è l’applicativo scelto per la creazione di grafici UML\G. La versione adottata è quella Professional, resa disponibile gratuitamente per un utilizzo da parte di studenti.
\paragraph{Telegram} Si utilizza Telegram\G\ per una comunicazione informale all'interno del 
gruppo. Inoltre Telegram\G\ fornisce il vantaggio di essere un'applicazione 
multi-piattaforma e disponibile anche in versione desktop/web.
\paragraph{Microsoft Office Power Poin} Power Point\G\ è l'applicativo utilizzato per creare presentazioni.

\subsection{Gestione delle infrastrutture}

\paragraph{Attività} 
\subparagraph{Gestione del repository} Il gruppo ha deciso di utilizzare un repository\G\ utile a svolgere funzioni diverse, ma necessarie, allo sviluppo del sistema finale. Una volta iscritto ciascun membro avrà la possibilità di creare il suo \textit{branch}\G\ personale contenente una copia dei file originali del \textit{branch}\G\ master in modo da poterci lavorare in locale.

\subparagraph{Gestione del messaggio di commit} Per mantenere l'ambiente di lavoro il meno ambiguo possibile, è stato deciso di adottare un formato standard per andare a scrivere il messaggio della commit\G.

\paragraph{Procedure} 
\subparagraph{Installazione di Git} La procedura di installazione varia a seconda del sistema operativo utilizzato.\\
Per i sistemi Linux\G\ occorre seguire la seguente procedura:
\begin{itemize}
\item Aprire il terminale;
\item Dare il comando sudo apt-get update;
\item Dare il comando apt-get install git;
\end{itemize}
Per sistemi OS X\G:
\begin{itemize}
\item Recarsi nella sezione dedicata ai \textit{download} https://git-scm.com/download/mac;
\item Scaricare il file in formato DMG;
\item Aprire il file appena scaricato;
\item Lanciare l'installazione cliccando su git.pkg.
\end{itemize}
Infine per i sistemi Windows\G\ fare quanto segue:
\begin{itemize}
\item Accedere al sito ufficiale https://git-for-windows.github.io/
\item Scaricare l'eseguibile;
\item Lanciare l'eseguibile;
\item Seguire la procedura riportata dalla finestra di dialogo.
\end{itemize}

\subparagraph{Creazione di una cartella locale di repository} Seguire la seguente procedura:
\begin{itemize}
\item Creare una nuova cartella;
\item Aprire il terminale;
\item Collocarsi all'interno della cartella appena create;
\item Eseguire il comando git init.
\item Dare il comando git clone <indirizzo>, sostituendo con <indirizzo> l'URL del progetto su GitHub\G.
\end{itemize}

\subparagraph{Creazione del branch personale} Per creare il \textit{branch} personale occorre seguire i seguenti passi:
\begin{itemize}
\item Muoversi nella cartella di repository\G.
\item Accedere alla cartella di progetto;
\item Eseguire il comando git branch <nome>, sostituendo a <nome> il nome del nuovo \textit{branch}\G\ da creare.
\end{itemize}

\subsubsection{Norme}
\paragraph{Repository}
\subparagraph{Nomi dei file in SiVoDim} I file e le cartelle presenti
nel repository\G\ devono essere conformi al seguente formalismo tratto dallo Standard
ISO\G\ 9660:1999 (Level 2):
\begin{itemize}
\item I caratteri usati sono solo quelli minuscoli a-z, 0-9, l’underscore  e il punto;
\item Non sono ammessi caratteri accentati;
\item I nomi non possono includere spazi o finire con un punto (.);
\item I nomi non devono contenere più di un punto (.) ad eccezione di quelli che fanno
riferimento ad una specifica versione;
\item I nomi non devono essere più lunghi di 21 caratteri esclusi i 3 destinati all’estensione.
\end{itemize}

\subparagraph{Struttura di SiVoDim}  Le cartelle nel repository\G\ verranno organizzate nel seguente modo a partire dalla root:
\begin{itemize}
\item \textbf{Documenti} nella quale sono presenti le seguenti cartelle corrispondenti alle varie fasi del processo di sviluppo:
\begin{itemize}
\item \textbf{RR}: contenente i documenti e i file necessari alla revisione dei requisiti;
\item \textbf{RP}: contenente i documenti e i file necessari alla revisione di progettazione;
\item \textbf{RQ}: contenente i documenti e i file necessari alla revisione di qualifica;
\item \textbf{RA}: contenente i documenti e i file necessari alla revisione dei accettazione;
\end{itemize}
\end{itemize}

\subparagraph{Messaggio di commit} Il messaggio di commit\G\ dovrà
essere conforme alla seguente notazione:
\\
Desc:\\
Data:\\
Note:\\
\\
dove:
\begin{itemize}
\item \textbf{Desc} fornisce una descrizione esaustiva dell’attività svolta;
\item \textbf{Data} fornisce la data in cui si è apportata la modifica;
\item \textbf{Note} aiuta a specificare lo stato del lavoro, nello specifico si adotteranno le seguenti notazioni:
\begin{itemize}
\item [C] se il lavoro è stato completato;
\item [NC] se il lavoro non è stato completato;
\item [V] se il lavoro necessita de verifica;
\end{itemize}
\end{itemize}

\subsubsection{Strumenti}
\paragraph{Git} Git\G\ è il sistema di controllo di versione utilizzato per il \textit{repository}\G\ del \textit{team}\G.

\paragraph{GitHub} GitHub\G\ è il servizio web di hosting adottato per tenere una
copia del \textit{repository}\G\ del progetto.

\paragraph{Dropbox} Dropbox\G\ è lo strumento di cloud\G\ che si è scelto di utilizzare
per gestire file che non necessitano di essere sottoposti a controllo di versione.

\paragraph{Sistema Operativo} I membri del gruppo operano su tre diversi sistemi operativi:
\begin{itemize}
\item Ubuntu\G;
\item Windows10\G;
\item Mac OS;
\end{itemize} 

\subsection{Formazione dei membri del gruppo}
I membri del gruppo, per soddisfare le richieste assegnate dal \textit{Responsabile di Progetto}
al quale non sanno fare fronte con le conoscenze attuali in loro possesso, dovranno
documentarsi adeguatamente durante ore esterne a quelle di lavoro, non imputabili
perciò come costi al proponente\G .

