\section{Documentazione}
Questa sezione descrive tutte le convenzioni concordate e adottate da Stark 
Labs riguardo a stesura, verifica e approvazione della
documentazione da produrre.

\subsection{Template}
Per agevolare e uniformare la redazione dei documenti è stato creato un 
template\G\ \LaTeX\ contenente tutte le impostazioni stilistiche e 
grafiche utilizzate.

\subsection{Struttura del documento}
\subsubsection{Prima pagina}
Ogni documento redatto in \LaTeX è caratterizzato da una prima pagina che 
contiene le seguenti informazioni che identificano il documento:
\begin{itemize}
	\item nome del progetto;
	\item logo del gruppo;
	\item titolo del documento;
	\item versione di fattibilità
	\item nome e Cognome dei redattori del documento;
	\item nome e Cognome dei verificatori del documento;
	\item nome e Cognome dei responsabili del documento;
	\item destinazione d'uso del documento;
	\item sommario del documento.
\end{itemize}

\subsubsection{Registro delle modifiche}
La seconda pagina di ogni documento prodotto contiene il registro delle modifiche apportate
al documento. Il registro è strutturato a tabella e ogni riga contiene:
\begin{itemize}
	\item un sommario sintetico delle modifiche svolte;
	\item nome e cognome dell'autore;
	\item data della modifica;
	\item versione del documento dopo la modifica.
\end{itemize}
La tabella è ordinata per data in ordine decrescente, in modo che la prima riga 
corrisponda all'ultima versione del documento.

\subsubsection{Indici}
In ogni documento è presente un indice delle sezioni, un indice delle figure e 
un indice delle tabelle.

\subsubsection{Formattazione generale delle pagine}
L'intestazione di ogni pagina contiene:
\begin{itemize}
	\item logo del gruppo;
	\item sezione corrente del documento.
\end{itemize}
A piè di pagina invece è presente:
\begin{itemize}
	\item nome del documento;
	\item versione del documento;
	\item pagina corrente nel formato N di T, dove N è il numero di pagina 
	corrente e T è il numero di pagine totali.
\end{itemize}

\subsection{Norme tipografiche}
Questa sezione contiene le norme che riguardano l'ortografia, la tipografia e l'utilizzo di uno stile uniforme per tutti i documenti.
\subsubsection{Stile del testo}
\begin{itemize}
	\item \textbf{corsivo}: va utilizzato per indicare termini in lingua inglese, citazioni, nomi di documenti;
	\item \textbf{grassetto}: va usato per evidenziare parole significative. Viene applicato a titoli di sezioni e sottosezioni;
	\item \textbf{maiuscolo}: viene utilizzato unicamente per scrivere acronimi e macro \LaTeX\ presenti nei documenti;
	\item \textbf{monospace}\G: serve per formattare il testo contenente porzioni di codice, percorsi dei file, comandi e indirizzi web;
	\item \LaTeX: viene usato il comando \textbackslash LaTeX per ogni occorrenza del termine \LaTeX.
\end{itemize}

\subsubsection{Punteggiatura}
\begin{itemize}
	\item \textbf{spaziatura}: lo spazio non può mai precedere un carattere di punteggiatura; 
	\item \textbf{maiuscolo}: vanno utilizzate lettere maiuscole per riferirsi ai ruoli di progetto, alle fasi di lavoro, al nome del team, del progetto e dei documenti.
	\item \textbf{numerazione}: viene utilizzato è lo standard internazionale SI/ISO 31-0 per indicare quantità e unità di misura;
	\item \textbf{monospace}\G: serve per formattare il testo contenente porzioni di codice, percorsi dei file, comandi e indirizzi web.
\end{itemize}

\subsubsection{Formati}
\begin{itemize}
	\item \textbf{data}:
	\item \textbf{ora}:
	\item \textbf{percorsi}:
	\item \textbf{nomi dei file}\G:
	\item \textbf{nomi propri}:
	\item \textbf{nome del gruppo}:
	\item \textbf{nome proponente}:
	\item \textbf{nome committente}:
	\item \textbf{nome del progetto}:
	\item \textbf{glossario}:
\end{itemize}

\subsubsection{Composizione del testo}
\begin{itemize}
	\item \textbf{elenchi puntati}: l'ultima voce deve terminare con un punto, mentre le altre con un punto e virgola. La prima lettera di ogni punto va scritta in minuscolo e la prima parola va in grassetto se seguita da una descrizione della stessa.
\end{itemize}

\subsubsection{Sigle}
\begin{itemize}
	\item \textbf{AdR}: Analisi dei Requisti;
	\item \textbf{GL}: Glossario;
	\item \textbf{NdP}: Norme di Progetto;
	\item \textbf{PdP}: Piano di Progetto;
	\item \textbf{PdQ}: Piano di Qualifica;
	\item \textbf{SdF}: Studio di Fattibilità
	\item \textbf{ST}: Specifica Tecnica;
	\item \textbf{RA}: Revisione di Accettazione;
	\item \textbf{RP}: Revisione di Progettazione;
	\item \textbf{RQ}: Revisione di Qualifica;
	\item \textbf{RR}: Revisione dei Requisiti.
\end{itemize}

\subsection{Componenti grafiche}
\subsubsection{Tabelle}
Ogni tabella va accompagnata da una didascalia e deve avere un numero identificativo incrementale per essere tracciabile.

\subsubsection{Immagini}
Tutte le immagini incluse nei documenti devono essere salvate in formato Portable Network Graphics (PNG\G).


\subsection{Classificazione dei documenti}
\subsubsection{Documenti formali}
Un documento è formale quando è stato approvato da un Responsabile di Progetto. Solo allora il documento può essere distribuito. Nel caso di modifiche, il documento deve essere sottoposto a nuova approvazione. Il processo di validazione è descritto nel Piano di Qualifica.

\subsubsection{Documenti informali}
Un documento è informale quando non ha ancora ricevuto l'approvazione del Responsabile di Progetto. Il suo utilizzo deve essere limitato ai componenti interni del gruppo.

\subsubsection{Glossario}
Il glossario è sviluppato su un documento a sé stante che contiene tutte le definizioni dei termini, in ordine alfabetico, che potrebbero risultare ambigui.

\subsection{Versionamento}
Ogni documento prodotto deve essere corredato dal numero di versione. Il formato adottato è il seguente:
\begin{center}
	vX.Y.Z
\end{center}
tale che:
\begin{itemize}
	\item \textbf{X}: indice di versione principale. Tale valore viene incrementato ad ogni approvazione del documento e ne indica la versione di rilascio.
	\item \textbf{Y}: indice di modifica parziale. Tale valore viene incrementato ad ogni verifica del documento.
	\item \textbf{Z}: indice di modifica minore. Tale valore viene incrementato ad ogni modifica del documento.
\end{itemize}

