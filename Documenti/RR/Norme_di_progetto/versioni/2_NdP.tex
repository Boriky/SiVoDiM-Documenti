\section{Processi primari}
In questa sezione vengono descritti i processi di fornitura e di sviluppo attuati dal gruppo \GRUPPO. Il processo di acquisizione spetta a Proponente\G\ e Committente\G\  del capitolato scelto, mentre il processo di manutenzione non può essere eseguito per vincoli dati dal tempo disponibile. Il prodotto \textit{software} e la documentazione fornita con esso devono garantire la possibilità futura di essere sottoposti alle attività del processo di manutenzione.
\subsection{Fornitura}
\subsubsection{Attività}
\paragraph{Accettazione}
\subparagraph{Discussione e scelta del capitolato}
Il \textit{Responsabile di Progetto} ha il compito di organizzare gli incontri per permettere ai componenti del gruppo di discutere sui capitolati disponibili.
Le valutazioni che hanno portato a prendere questa decisione vanno documentate nello \textit{Studio di Fattibilità}.
\subparagraph{Studio di fattibilità}
Per realizzare il documento devono essere presi in considerazione i seguenti punti, adattati ai capitolati disponibili:
\begin{itemize}
\item{Valutazione generale del capitolato;}
\item{Valutazione dei fattori di rischio.}
\end{itemize}

Per il capitolato scelto devono essere analizzati anche i seguenti punti:
\begin{itemize}
	\item{Studio del dominio applicativo;}
	\item{Studio del dominio tecnologico;}
	\item{Analisi di mercato;}
	\item{Analisi delle potenziali criticità.}
\end{itemize}

\paragraph{Preparazione della risposta}
\subparagraph{Definizione e preparazione della proposta}
I membri del gruppo  \GRUPPO\ devono redigere i seguenti documenti:
\begin{itemize}
	\item{\textit{Norme di Progetto v1.0.0};}
	\item{\textit{Studio di Fattibilità v1.0.0};}
	\item{\textit{Analisi dei Requisiti v1.0.0};}
	\item{\textit{Piano di Progetto v1.0.0};}
	\item{\textit{Piano di Qualifica v1.0.0}.}
\end{itemize}
In allegato viene consegnata anche la \textit{Lettera di Presentazione}. Si veda la sezione 3.1.4.2 Versionamento per maggiori dettagli sul numero di versione indicato nei documenti.

\paragraph{Pianificazione}
\subparagraph{Scelta del modello di ciclo di vita}
Il \textit{Responsabile di Progetto} ha il compito di scegliere il modello di ciclo di vita\G\ adatto per lo sviluppo del prodotto richiesto, a meno che non venga fornita un'indicazione dal Proponente\G.
\subparagraph{Sviluppo e documentazione del Piano di Progetto}
Il \textit{Responsabile di Progetto} ha il compito di delineari i lavori che i membri del gruppo devono eseguire. Inoltre deve calcolare costi e tempistiche per le attività da svolgere. Tali pianificazioni sono trascritte nel documento \textit{Piano di Progetto v1.0.0}.

\subsection{Sviluppo}
Il processo di sviluppo contiene le attività che definiscono la realizzazione del prodotto \textit{software} da parte del gruppo \GRUPPO.

\subsubsection{Attività}
\paragraph{Analisi dei requisiti}
Dopo aver redatto lo \textit{Studio di Fattibilità}, gli \textit{Analisti} devono produrre il documento \textit{Analisi dei Requisiti v1.0.0}. L'obiettivo è produrre dei requisiti a partire dalle informazioni acquisite dal gruppo.
Le risorse utilizzabili per questo scopo provengono dal capitolato d'appalto e dagli incontri con il Proponente\G\ e con il Committente\G.

\subsubsection{Norme}
\paragraph{Classificazione requisiti}
I requisiti prodotti devono essere classificati a seconda del tipo e dell'importanza, rispettando la seguente notazione:
\begin{center}
	R[Importanza][Tipo][Codice]
\end{center}

\begin{itemize}
	\item\textbf{Importanza}: i valori che può assumere sono:
	
	\begin{itemize}
		\item[-] 0: requisito obbligatorio;
		\item[-] 1: requisito desiderabile;
		\item[-] 2: requisito opzionale.
	\end{itemize}
	
	\item\textbf{Tipo}: i valori che può assumere sono:
		\begin{itemize}
			\item[-] F: requisito funzionale;
			\item[-] P: requisito prestazionale;
			\item[-] Q: requisito di qualità;
			\item[-] V: requisito di vincolo.
		\end{itemize}
	
	\item\textbf{Codice} : è il codice gerarchico e univoco del vincolo espresso nella forma X.X.X dove X è un valore numerico.
\end{itemize}
Ogni requisito inoltre deve contenere le seguenti informazioni:
	
\begin{itemize}
	\item\textbf{Descrizione}: descrizione del requisito con la minore ambiguità possibile;
	\item\textbf{Fonte}: la scelta può ricadere tra:
	\begin{itemize}
		\item[-] \textbf{Capitolato}: requisito ottenuto dalle specifiche del capitolato;
		\item[-] \textbf{Interno}: requisito elaborato dagli \textit{Analisti} nel corso di un'analisi più approfondita del problema;
		\item[-] \textbf{Caso d'uso}: requisito ottenuto da uno o più casi d'uso. Deve essere quindi specificato il codice del caso d'uso a cui ci si riferisce;
		\item[-] \textbf{Verbale}: requisito ottenuto da un'incontro con il Proponente\G\ o da riunioni interne tra i membri del gruppo.	
	\end{itemize}
\end{itemize}

\paragraph{Classificazione casi d'uso}
I casi d'uso devono essere suddivisi in ordine gerarchico secondo il seguente schema:
\begin{center}
	UC[Codice]
\end{center}

	\begin{itemize}
		\item \textbf{Codice}: è il codice gerarchico e univoco che serve a identificare ogni caso d'uso.
	\end{itemize}
Per ogni caso d'uso devono essere presenti anche le seguenti informazioni:
	\begin{itemize}
		\item \textbf{Titolo}: è necessario fornire un titolo riassuntivo dell'operazione che il caso d'uso intende modellare;
		\item \textbf{Descrizione}: è necessario fornire una breve descrizione con la minore ambiguità possibile;
		\item \textbf{Attori principali}: elenco degli attori principali coinvolti nel caso d'uso;
		\item \textbf{Attori secondari}: elenco degli attori secondari coinvolti nel caso d'uso;
		\item \textbf{Scenari principali}: descrizione dei possibili scenari principali;
		\item \textbf{Scenari alternativi}: descrizione dei possibili scenari secondari;
		\item \textbf{Pre-condizioni}: una pre-condizione è una condizione sempre vera all'inizio del caso d'uso;
		\item \textbf{Flusso degli eventi}: ordine di esecuzione dei casi d'uso figli;
		\item \textbf{Inclusioni}: spiegazione di tutte le inclusioni se presenti;
		\item \textbf{Estensioni}: spiegazione di tutte le estensioni se presenti;
		\item \textbf{Generalizzazioni}: spiegazione di tutte le generalizzazioni se presenti;
		\item \textbf{Post-condizioni}: una post-condizione è una condizione sempre vera alla fine dell'esecuzione del caso d'uso.
	\end{itemize}
Alcune fra le precedenti informazioni potrebbero essere assenti nel caso non fossero utilizzate.

\paragraph{Codifica file}
Tutti i \textit{file} creati dal gruppo contenenti sia codice che documentazione devono essere codificati tramite UTF-8\G\ senza BOM\G.\\
Eventuali cambiamenti di codifica devono essere approvati dal \textit{Responsabile di Progetto} .
\paragraph{Nomi e norme di codifica}
Questa sezione verrà redatta nel dettaglio durante le fasi successive.
Vengono introdotte alcune norme che andranno applicate per lo sviluppo del codice, indipendentemente dal linguaggio di programmazione che verrà adottato. \\
Le norme introdotte sono le seguenti:

\begin{itemize}
	\item In ogni \textit{file} deve essere presente un'intestazione contente le seguenti informazioni: 
	\begin{itemize}
		\item[-] Percorso e nome del \textit{file};
		\item[-] Cognome e nome dell'autore;
		\item[-] Data di creazione;
		\item[-] Indirizzo email dell’autore;
		\item[-] Per ogni modifica effettuata devono essere specificati: la versione successiva generata dall'avanzamento, l'autore, la data e una breve descrizione.	
	\end{itemize}
	\item I nomi delle variabili devono essere chiari, descrittivi e in inglese;
	\item I commenti vanno scritti in italiano.
\end{itemize}

\paragraph{Ricorsione}
Quando possibile, la ricorsione va evitata. Per ogni funzione ricorsiva è d'obbligo fornire una prova di terminazione. Risulta inoltre necessario valutare il costo in termini di occupazione della memoria. Nel caso in cui l’utilizzo di memoria risulti eccessivo, la ricorsione deve essere rimossa.
\subsubsection{Strumenti}
\paragraph{Requirements Tool}
COMPLETARE CON SPIEGAZIONE DELL'APPLICAZIONE PER MEMORIZZARE I REQUISITI
\paragraph{PhpStorm ?}
COMPLETARE DESCRIVENDO L'IDE UTILIZZATO PER REALIZZARE L'APPLICAZIONE PRECEDENTE.

\newpage
