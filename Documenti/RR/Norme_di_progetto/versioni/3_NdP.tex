\section{Processi di supporto}
\subsection{Documentazione}
\subsubsection{Attività}
\paragraph{Documentazione} 
Il gruppo di lavoro \GRUPPO\ si impegna a registrare tutte le informazioni acquisite nel corso del ciclo di vita\G\ del \textit{software} all'interno di una serie definita di documenti che verranno descritti in questa sezione. Nello specifico, vengono spiegati gli standard da rispettare per la stesura dei documenti e i passaggi necessari per renderli formalmente corretti.

\subsubsection{Procedure}
\paragraph{Gestione dei documenti}
La stesura di un nuovo documento viene decisa dal \textit{Responsabile di Progetto}. Tutta la documentazione deve essere creata attraverso l'uso di un \textit{template}\G\ \LaTeX\ disponibile nel \textit{repository}\G\ GitHub\G\ del gruppo, al fine di mantenerne uniforme la struttura e lo stile.

\paragraph{Creazione di un nuovo documento}
Nel \textit{repository}\G\ è presente un \textit{file} generico denonimato \textit{new\_doc.tex} contenente un \textit{template} adattabile ad ogni nuovo documento. Il \textit{file} deve essere incluso nella cartella contenente il \textit{template} \LaTeX\, che è messo a disposizione a ogni membro del gruppo. Ogni redattore deve possedere tutti i \textit{file} previsti dal \textit{template} al fine di creare correttamente un nuovo documento.

\paragraph{Avanzamento di un documento}
Le procedure adottate per sviluppare un documento sono le seguenti:
\begin{itemize}
	\item Il \textit{Responsabile di Progetto} si preoccupa di assegnare la stesura del documento a uno o più redattori, a seconda delle complessità dello stesso. L'assegnazione è gestita attraverso \textit{task}\G\ organizzati sul servizio \textit{online} Teamwork\G; 
	\item A stesura completata, ogni assegnatario ha il compito di segnalare il \textit{task} come completato;
	\item Il \textit{Verificatore} riceve in automatico un'email che segnala il completamento del documento;
	\item Nel caso si riscontrassero errori durante la fase di verifica, il \textit{Verificatore} deve occuparsi di creare un nuovo \textit{task} da assegnare al redattore del documento;
	\item Nel caso di documento corretto, il \textit{Verificatore} deve completare il task assegnatogli. Un'email viene quindi automaticamente recapitata al \textit{Responsabile di Progetto}.
	\item Il \textit{Responsabile di Progetto} deve infine approvare il documento. In caso di mancata approvazione, il Verificatore dovrà creare un nuovo \textit{task} indirizzato al redattore, che si occuperà di correggere gli errori in base alle segnalazioni emesse. 
\end{itemize}

\subsubsection{Gestione del glossario}
Il popolamento del \textit{Glossario v1.0.0} è un attività che coinvolge redattori e \textit{Verificatori}.
\begin{itemize}
	\item{\textbf{Individuazione di un nuovo termine}}: se durante la stesura del documento il redattore identifica un nuovo termine che ritiene debba essere inserito nel \textit{Glossario v1.0.0}, è tenuto a trascriverlo all'interno di un apposito documento (Glossario Provvisorio) presente nella sezione \textit{Notebooks} di Teamwork\G.
	\item{\textbf{Inserimento del termine nei documenti}}: un \textit{Verificatore} deve occuparsi dell'approvazione di un dato termine presente in Glossario Provvisorio di Teamwork e dell'inserimento dello stesso all'interno del \textit{Glossario v1.0.0}. Una volta inserito, il termine deve essere rimosso dal documento di Teamwork.
\end{itemize}

\subsubsection{Norme}
\paragraph{Progettazione e sviluppo dei documenti}
Ogni documento deve rispettare rigidamente la seguente serie di norme.

\paragraph{Versionamento}
Ogni documento prodotto deve essere corredato dal numero di versione. Il formato adottato è
il seguente:
\begin{center}
	vX.Y.Z
\end{center}
tale che:
\begin{itemize}
	\item{\textbf{X}}: indice di versione principale. Tale valore viene incrementato ad ogni approvazione del documento e ne indica la versione di rilascio;
	\item{\textbf{Y}}: indice di modifica parziale. Tale valore viene incrementato ad ogni verifica del documento;
	\item{\textbf{Z}}: indice di modifica minore. Tale valore viene incrementato ad ogni cambiamento che avviene del documento.
\end{itemize}

\paragraph{Template}
La creazione dei documenti avviene attraverso l'utilizzo di un \textit{template}\G\ sviluppato con \LaTeX\ e la cui struttura \textbf{non} deve essere modificata a meno di direttive imposte dal \textit{Responsabile di Progetto}. Il \textit{template} funge da supporto per la stesura organizzata e sistematica dei documenti e, grazie ad esso, ogni componente del documento ha una precisa impostazione che non può essere modificata o equivocata dai redattori.

\paragraph{Struttura dei documenti}
L'organizzazione dei documenti è la seguente: 
\begin{itemize}
	\item Una cartella generale in cui sono contenuti il \textit{template}\G\ \LaTeX\ e le varie cartelle specifiche di ogni documento;
	\item Nella cartella \textit{Template} sono contenuti i \textit{file} di configurazione e strutturazione del \textit{template}. Il contenuto della cartella non deve essere modificato, previa autorizzazione da parte del \textit{Responsabile di Progetto};
	\item Nelle cartella specifica di ogni documento è presente un \textit{file} di tipo nome\_documento.tex tramite cui si determina la struttura dello specifico documento accompagnato da una sotto cartella "sezioni" che contiene le varie sezioni nelle quali sono scritti i contenuti veri e propri.
\end{itemize}

\subparagraph{Prima pagina}
\label{sec:primaPagina}
Nella prima pagina sono presenti tutte le informazioni generali relative al documento:
\begin{itemize}
	\item Nome del progetto;
	\item Logo del gruppo;
	\item Nome del documento;
	\item Versione del documento;
	\item Membri del gruppo che hanno lavorato come redattori, \textit{Verificatori} e \textit{Responsabili} per la stesura del documento. I nomi vanno scritti nel formato Nome Cognome;
	\item Specifica dell'uso del documento (interno o esterno);
	\item Lista di distribuzione del documento. I nomi vanno scritti nel formato Nome Cognome;
	\item Breve descrizione che identifica lo scopo del documento.
\end{itemize}
\subparagraph{Registro delle modifiche}
Nella seconda pagina si trova una tabella con il registro delle modifiche effettuate al documento. Esso è indispensabile per un corretto tracciamento delle varie fasi che si sono percorse lungo la sua stesura. Ogni riga corrisponde a una modifica dove vengono segnalati:
\begin{itemize}
	\item Descrizione dell'azione compiuta sul documento;
	\item Nome e cognome dell'autore della modifica;
	\item Data della modifica;
	\item Versione del documento a seguito della modifica.
\end{itemize}

\subparagraph{Indici}
In terza pagina è presente l'indice che tiene traccia delle varie sezioni in cui è stato suddiviso il documento. La profondità dell'indice arriva a cinque livelli: gli argomenti trattati sono suddivisi in sezioni, sottosezioni, sotto-sottosezioni, paragrafi e sottoparagrafi. L'unico documento in cui non è presente l'indice è il \textit{Glossario v1.0.0}.
Se presenti sono contenute immagini e/o tabelle, è possibile trovare automaticamente il relativo indice associato.
\subparagraph{Formattazione di una pagina}
Tutte le pagine dei documenti seguono una precisa formattazione imposta dal \textit{template}\G :
\begin{itemize}
	\item{\textbf{Intestazione}}: contiene sulla sinistra il logo del gruppo e sulla destra il nome della sezione in cui è contenuta la pagina;
	\item{\textbf{Contenuto}}: contiene il contenuto effettivo della pagina le cui norme tipografiche sono descritte in \hyperref[sec:normeTipografiche]{3.1.4.5.6 Norme Tipografiche};
	\item{\textbf{Piè di pagina}}: contiene sulla sinistra il nome del documento accompagnato dal numero della versione e sulla destra il numero della pagina, scritto nel formato "X di Y", con X numero di pagina corrente e Y numero delle pagine totali.
\end{itemize}
La suddetta struttura si ripete per ogni pagina ad eccezione della prima, il cui formato è descritto in \hyperref[sec:primaPagina]{3.1.4.4.1 Prima Pagina}.

\paragraph{Suddivisione dei documenti}
\subparagraph{Norme di Progetto}
Il documento ha lo scopo di definire le linee guida per le varie attività di sviluppo. Al suo interno sono raccolte le norme, le procedure e gli strumenti che il gruppo adotterà nel corso della realizzazione del progetto. Il documento è destinato a uso interno.
\subparagraph{Studio di Fattibilità}
Il documento ha lo scopo di descrivere le considerazioni elaborate dal gruppo per l'accettazione del progetto che si è deciso di prendere in carico, con valutazione di rischi, costi e benefici calcolati sulla base di una prima analisi del capitolato. Al suo interno vengono motivate le scelte che hanno spinto il gruppo all'esclusione degli altri progetti. Il documento è destinato a uso interno.
\subparagraph{Analisi dei Requisiti}
Il documento ha lo scopo di identificare e descrivere i requisiti, i vincoli e gli obiettivi necessari allo sviluppo del progetto. Al suo interno sono contenuti i casi d'uso e i requisiti utili alla realizzazione del progetto, accompagnati da diagrammi e grafici di interazione fra utenti e sistema. Il documento è destinato a uso esterno.
\subparagraph{Piano di Progetto}
Il documento ha lo scopo di pianificare lo svolgimento del progetto. Al suo interno sono fissate le risorse disponibili, la suddivisione e il calendario delle attività, e gli obiettivi necessari per valutare in modo corretto il grado di avanzamento dello sviluppo. Il documento è destinato a uso esterno.
\subparagraph{Piano di Qualifica}
Il documento ha lo scopo di spiegare le strategie applicate al progetto per ottenere gli obiettivi di qualità. Al suo interno sono presenti le attività di verifica e pianificazione con i relativi test da sviluppare. Il documento è destinato a uso esterno.
\subparagraph{Specifica Tecnica}
Il documento ha lo scopo di descrivere l'architettura logica del progetto, senza fissare i dettagli implementativi, ma definendo linee e strategie di realizzazione, al fine di stabilitre cause ed effetti e avere una visione complessiva della soluzione. Al suo interno è contenuta una prima progettazione ad alto livello del sistema da sviluppare. In esso vengono specificati i \textit{design pattern}\G\ utilizzati. Il documento è destinato a uso esterno. 
\subparagraph{Definizione di Prodotto}
Il documento ha lo scopo di descrivere nel dettaglio l'architettura del prodotto da sviluppare. Il suo contenuto viene utilizzato dai \textit{Programmatori} per sviluppare il \textit{software}. Il documento è destinato a uso esterno.
\subparagraph{Glossario}
Il documento ha lo scopo di raccogliere, in ordine alfabetico, tutti i termini ambigui presenti nei documenti accompagnati da una loro definizione. Il documento è destinato a uso esterno.

\paragraph{Norme tipografiche}
\label{sec:normeTipografiche}
Al fine di rendere omogenea e coesa la stesura dei documenti, il contenuto deve rispettare le seguenti norme tipografiche.
\subparagraph{Stile del testo}
\begin{itemize}
\item \textbf{Corsivo}: va utilizzato tassativamente per indicare termini in lingua inglese, citazioni, nomi di documenti interni, ruoli dei membri del gruppo, nome del \textit{team} ed eventualmente (ma con moderazione) per parole che si ritiene debbano essere messe in risalto rispetto al resto del testo; 
\item \textbf{Grassetto}: va usato per evidenziare parole significative di estrema importanza. È importante che non se ne abusi. Viene applicato automaticamente a titoli di sezioni, sottosezioni e paragrafi; 
\item \textbf{Maiuscolo}: viene utilizzato unicamente per scrivere acronimi e macro \LaTeX; 
\item \textbf{Monospace}\G: serve per formattare il testo contenente porzioni di codice, percorsi dei \textit{file}, comandi e indirizzi \textit{web};
\item \LaTeX: viene usato il comando \textbackslash LaTeX per ogni occorrenza del termine \LaTeX.
\item \textbf{Font}: nel \textit{template} è impostato Gillius, un \textit{font} professionale di tipo \textit{sans-serif} il cui scopo è garantire maggiore leggibilità dei documenti su schermo.  
\end{itemize}
\subparagraph{Punteggiatura}
\begin{itemize}
\item \textbf{Spaziatura}: lo spazio non può mai precedere un carattere di punteggiatura; 
\item \textbf{Maiuscolo}: vanno utilizzate lettere maiuscole per riferirsi ai ruoli di progetto, alle fasi di lavoro e ai seguenti nomi: del \textit{team}, del progetto e dei documenti. 
\item \textbf{Numerazione}: viene utilizzato lo standard internazionale SI/ISO 31-0 per indicare quantità e unità di misura.
\end{itemize} 
\subparagraph{Composizione del testo}
\begin{itemize}
	\item {\textbf{Elenchi puntati}}: l’ultima voce deve terminare con un punto, mentre le altre con un punto e virgola. La prima lettera di ogni punto va scritta in maiuscolo e la prima parola va in grassetto se seguita da una descrizione della stessa.
	elenchi numerati... non numerati...?
	\item \textbf{Pedice G}: il pedice \G\ è utilizzato al solo scopo di indicare termini potenzialmente ambigui contenuti nel documento \textit{Glossario v1.0.0}.
\end{itemize}

\subparagraph{Formati} Elenco dei formati rispettati dai documenti.
\begin{itemize}
	\item \textbf{Data}: il formato utilizzato è dd/mm/yyyy. Per esempio, 04/06/1996 indica il 4 giugno 1996;
	\item \textbf{Ora}: si utilizza il formato internazionale previsto dalla norma ISO 8601 del tipo [hh]:[mm]:[ss] ove [hh] indica l'ora, [mm] i minuti, [ss] i secondi, espressi con due cifre. Ad esempio, 18:35:26 indica le ore 18, 35 minuti e 26 secondi; 04:09:01 indica le ore 4, 9 minuti e 1 secondo;
	\item \textbf{Percorsi}: viene utilizzato il separatore \textit{slash} (/) per indicare il percorso di un file. Per esempio, cartella1/cartella2/file\_esempio.txt;
	\item \textbf{Nomi di persona}: espressi nel formato Nome Cognome.
\end{itemize}

\subparagraph{Sigle} Elenco delle sigle che possono apparire nel corso dei documenti.
\begin{itemize}
	\item \textbf{AdR}: Analisi dei Requisti; 
	\item \textbf{GL}: Glossario; 
	\item \textbf{NdP}: Norme di Progetto; 
	\item \textbf{PdP}: Piano di Progetto;
	\item \textbf{PdQ}: Piano di Qualifica; 
	\item \textbf{SdF}: Studio di Fattibilità;
	\item \textbf{ST}: Specifica Tecnica; 
	\item \textbf{RA}: Revisione di Accettazione;
	\item \textbf{RP}: Revisione di Progettazione;
	\item \textbf{RQ}: Revisione di Qualifica;
	\item \textbf{RR}: Revisione dei Requisiti.
\end{itemize}

\paragraph{Componenti grafiche}
\subparagraph{Immagini}
Tutte le immagini incluse nei documenti devono essere salvate in formato \textit{Portable Network Graphics} (PNG\G).
\subparagraph{Diagrammi}

\subsection{Processo di verifica e validazione}
Il processo di verifica consiste nel controllare che il materiale prodotto al raggiungimento delle \textit{milestone}\G\ sia conforme agli obiettivi prefissati. Pertanto, è necessario verificare che non siano stati prodotti errori. Il prodotto è validato se il risultato ottenuto è consistente e conforme alle attese. Una corretta applicazione del processo di verifica genera un aumento del rapporto fra efficienza ed efficacia, riducendo il tempo impiegato nel percorso di analisi.
\subsubsection{Attività}
\paragraph{Analisi statica}
L'attività di analisi statica è una tecnica di verifica applicabile sia a documenti che a codice sorgente che va ad analizzare il solo testo del \textit{file} senza mandarlo in esecuzione. Tale tecnica viene utilizzata durante l'intero sviluppo del progetto e si pone come obiettivo il ritrovamento di eventuali anomalie. I metodi di controllo sono i seguenti:
\begin{itemize}
	\item \textbf{Walktrough}: si ricercano all'interno di testo o codice tutte le possibili anomalie; l'attività è eseguita da un umano. L'analisi si basa sulla lettura di tutto il contenuto del \textit{file}. In seguito al ritrovamento di anomalie, le si analizza con il redattore del documento (o \textit{Programmatore} nel caso di codice) e si indaga per raggiungere una soluzione del problema. Tale tecnica risulta utile durante le prime fasi di sviluppo del progetto, in quanto manca, ai componenti, una visione complessiva del documento o del codice che si sta scrivendo. Permette così ai \textit{Verificatori}, dopo aver svolto le prime correzioni, di preparare una \textit{lista di controllo} con gli errori più frequenti in modo da migliorare l'efficienza delle verifiche future. A questo punto è possibile implementare un'analisi mirata e più efficiente attraverso il metodo dell'\textit{Inspection}.
	\item \textbf{Inspection}: si ricercano all'interno di un testo errori specifici; l'attività può essere eseguita sia da un umano che, nel caso di anomalie nella sintassi, da uno script. Il metodo focalizza la ricerca su errori presupposti identificati dalla \textit{lista di controllo}.
\end{itemize}
\paragraph{Analisi dinamica}
L'attività di analisi dinamica è una tecnica di verifica applicabile solamente al \textit{software}. Tale tecnica può essere utilizzata per analizzare l'intero \textit{software} o una porzione limitata dello stesso. L'attività consiste nell'esecuzione di \textit{test} automatici realizzati dal \textit{team}. Le verifiche devono essere effettuate su un insieme finito di casi, con valori di ingresso, uno stato iniziale e un esito decidibile. Tutti i \textit{test} producono risultati automatici che inviano notifiche sulla tipologia di problema individuato. Ogni \textit{test} è ripetibile, ossia applicabile durante l'intero ciclo di vita\G\ del \textit{software}. Le caratteristiche da rispettare sono le seguenti: 
\begin{itemize}
	\item \textbf{Ambiente}: è necessario riportare l'ambiente sia \textit{software} che \textit{hardware} in cui il sistema esegue il \textit{test}. Deve essere specificato lo stato iniziale del sistema; 
	\item \textbf{Specifica}: è necessario riportare i dati in ingresso e in uscita per verificare l'esito del \textit{test}, ossia se il codice analizzato è conforme alle aspettative;
	\item \textbf{Procedure}: è possibile specificare ulteriori istruzioni per l'esecuzione dei \textit{test}. Inoltre possono essere riportate istruzioni sulle corretta lettura dei risultati.
\end{itemize}
\paragraph{Gestione anomalie}
Se si dovessero riscontrare anomalie o discordanze normative durante le attività di verifica, il \textit{Verificatore} ha il dovere di notificarle all'assegnatario del \textit{task}\G\  

\paragraph{Tracciamento}
L'attività di tracciamento svolta dai \textit{Verificatori} consiste nella catalogazione di tutti i casi d'uso e i requisiti che ne derivano ed evidenziare la corrispondenza fra di essi in modo da avere ben chiaro i processi di derivazione.

\subsubsection{Procedure}
\paragraph{Procedure di assegnazione delle anomalie}

\subsubsection{Norme}
\paragraph{Priorità risoluzione anomalie}

\subsubsection{Strumenti}
\paragraph{Correzione ortografica}
\paragraph{Calcolo indice Gulpease}
\paragraph{Database online}