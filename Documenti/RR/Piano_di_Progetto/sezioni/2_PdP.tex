\section{Introduzione}

\subsection{Scopo del documento}
Questo documento ha lo scopo di descrivere l'organizzazione che il gruppo \GRUPPO\ si è dato per portare a termine il progetto \PROGETTO. Viene pianificata, sulla base delle scadenze e delle risorse disponibili, la dislocazione temporale con cui suddividere in modo proficuo le attività da svolgere. Più precisamente, viene indicato:
\begin{itemize}
\item Un insieme di fattori di rischio che sono stati identificati per poter essere arginati;
\item Il tempo previsto per ogni attività;
\item Una stima del costo in termini di risorse;
\item Un bilancio sull'utilizzo totale delle risorse.
\end{itemize}  

\subsection{Scopo del prodotto}
L'obiettivo del progetto è di sperimentare e rendere disponibili su dispositivi mobili nuove funzionalità di sintesi vocale (TTS\G), come la possibilità di applicare effetti alle voci digitali o di poter sintetizzare e utilizzare la voce degli utenti. L'applicazione specifica, realizzata per ambiente Android\G\, deve gestire la strutturazione, scrittura e lettura di sceneggiati, che possono essere salvati in formato audio ed essere facilmente condivisibili su \textit{Social Network}\G\ ed altro.

\subsection{Glossario}
Al fine di aumentare la comprensione del testo ed evitare eventuali ambiguità, viene fornito un glossario (Glossario v1.0) contenente le definizioni degli acronimi e dei termini tecnici utilizzati nel documento. Ogni vocabolo che ha un riferimento contenuto nel glossario è contrassegnato dal pedice “\G “.

\subsection{Riferimenti}

\subsubsection{Normativi}
\begin{itemize}
\item \textbf{Norme di Progetto}: Norme di Progetto v1.0.0;
\item \textbf{Regole del progetto didattico}: \url{http://www.math.unipd.it/~tullio/IS-1/2015/Dispense/PD01.pdf};
\item \textbf{Regolamento Organigramma}: \url{http://www.math.unipd.it/~tullio/IS-1/2015/Progetto/PD01b.html}.
\end{itemize}

\subsubsection{Informativi}
\begin{itemize}
\item \textbf{Glossario}: Glossario v1.0.0;
\item \textbf{Studio Fattibilità}: Studio di Fattibilità v1.0.0.
\item \textbf{SWEBOK Guide\G}: The Guide to the Software Engineering Body of Knowledge V3 \\ \url{https://www.computer.org/web/swebok/}
\end{itemize}
 

\subsection{Scadenze}
Di seguito sono riportate le scadenze che il gruppo \GRUPPO\ si impegna a rispettare. La fase di pianificazione delle attività si basa su tali date.
\begin{itemize}
\item \textbf{Revisione dei Requisiti (RR)}: 18/04/2016
\item \textbf{Revisione dei Progettazione (RP)}: 23/05/2016
\item \textbf{Revisione di Qualifica (RQ)}: 17/06/2016
\item \textbf{Revisione di Accettazione (RA)}:  11/07/2016
\end{itemize}

\subsection{Ciclo di vita}
Per lo sviluppo del prodotto è stato scelto di seguire il modello di ciclo di vita XXX...
\textbf{motivare la scelta}.