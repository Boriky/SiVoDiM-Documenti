\section{Pianificazione}
I diagrammi delle attività presenti in questa sezione sono stati rappresentati 
tramite l'uso di diagrammi di Gantt.

\subsection{Analisi dei Requisiti}
\textbf{Periodo:} 04/03/2016 - 31/03/2016\\
Questa fase prevedere la scelta e l'implementazione degli strumenti necessari 
per definire il \textit{repository}\G\ utilizzato dal \textit{team}\G, le 
norme di base per sviluppare una documentazione quanto più possibile omogenea e 
coerente, una pianificazione che guida lo sviluppo del progetto. Termina con 
la scadenza di consegna dell'offerta, cioè con la consegna della 
\textit{Revisione dei Requisiti}.\\\\
I ruoli attivi in questa fase sono:

\begin{itemize}
	\item Responsabile;
	\item Amministratore;
	\item Analista;
	\item Progettista;
	\item Verificatore.
\end{itemize}
Le attività da svolgersi in questa fase sono:
\begin{itemize}
	\item \textbf{Norme di Progetto}: attività svolta dall'Amministratore di 
	Progetto. 
	Concordati con il \textit{team}\G\ gli strumenti da utilizzare, si procede 
	alla stesura di un serie di norme, che dovranno essere rispettate dai 
	membri del team per tutta la durata del progetto. Le norme sono interne al 
	\textit{team}\G\ e non legate al capitolato SiVoDiM;
	\item \textbf{Studio di fattibilità}: è compito degli Analisti valutare 
	ogni capitolato e redarre uno Studio di fattibilità, dal quale si delinea 
	chiaramente quale capitolato è stato scelto e la motivazione che ha guidato 
	a tale scelta.
	\item \textbf{Piano di Progetto}: qui vengono pianificate le attività, 
	risorse e costi della gestione del \textit{team}\G\. Poi riportate in modo 
	strutturato nel \textit{Piano di Progetto v1.0.0};
	\item \textbf{Analisi dei Requisiti}: gli Analisti hanno l'incarico di 
	ricercare i requisiti e di redigere il documento Analisi dei Requisiti 
	v1.0.0, con Diagrammi dei Casi D'Uso (UC\G). Questa attività è poco legata 
	al proponente\G, data la natura vaga del Capitolato\G\ fornito dall'azienda 
	MIVOQ s.r.l.;
	\item \textbf{Piano di Qualifica}: descrizione su strategie di verifica e 
	validazione adottate. Documentate poi nel \textit{Piano di Progetto v1.0.0};
	\item \textbf{Glossario}: attività parallela alla stesura di tutti i 
	documenti sopracitati. Il \textit{Glossario v1.0.0} verrà aggiornato in 
	modo incrementale fino al completamento della documentazione;
	\item \textbf{Verbale incontri}: per analizzare i requisiti presentati nel 
	Capitolato \G\ vengono organizzati incontri, in seguito documentati con 
	verbali formali.
\end{itemize}
\includegraphics[width= 16cm]{AR.png}

\subsection{Analisi di Dettaglio}
\textbf{Periodo:} 01/04/2016 - 18/04/2016\\
Questa attività ha termina con la presentazione in data 18/04/2016 dell'Analisi 
dei Requisiti\\\\
I ruoli attivi in questa fase sono:

\begin{itemize}
	\item Responsabile;
	\item Amministratore;
	\item Analista;
	\item Progettista;
	\item Verificatore.
\end{itemize}
Questa fase ha lo scopo di integrare e consolidare i requisiti ottenuti 
precedentemente.

\includegraphics[width= 16cm]{AD.png}

\subsection{Progettazione Architetturale}
\textbf{Periodo:} 19/04/2016 - 23/05/2016\\

\subsection{Progettazione di Dettaglio e Codifica}
\textbf{Periodo:} 24/05/2016 - 17/06/2016\\

\subsection{Validazione}
\textbf{Periodo:} 18/06/2016 - 11/07/2016\\