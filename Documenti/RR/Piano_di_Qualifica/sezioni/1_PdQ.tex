\section{Introduzione}

\subsection{Scopo del documento}
Questo documento ha lo scopo di definire le strategie che il gruppo di lavoro 
ha deciso di adottare per perseguire obiettivi qualitativi da applicare al 
proprio progetto. A tale scopo è necessario un continuo processo di verifica, 
in modo da correggere eventuali anomalie o incongruenze sulle attività svolte 
in maniera tempestiva e senza spreco di risorse.

\subsubsection{Scopo del Prodotto}
L'obiettivo del progetto è di sperimentare e rendere disponibili su dispositivi 
mobili nuove funzionalità di sintesi vocale (TTS\G), come la possibilità di 
applicare effetti alle voci digitali o sintetizzare e utilizzare la voce degli 
utenti.
Ciò sarà possibile realizzando due applicazioni per i sistemi Android\G.
\begin{itemize}
	\item La prima deve permettere all'utente di interfacciarsi direttamente 
	con il sistema operativo per configurare, salvare, modificare nuove voci 
	campionate;
	\item La seconda invece permette la creazione, il salvataggio e la 
	condivisione di veri e propri sceneggiati.
\end{itemize}
Entrambe le applicazioni devono sfruttare altre due componenti, realizzate 
sempre all'interno del progetto:
\begin{itemize}
	\item Un Engine\G\ che permetta di interfacciarsi tramite una connessione 
	al motore di sintesi prodotto dal proponente\G;
	\item Una libreria di funzionalità.
\end{itemize} 

\subsection{Glossario}
Al fine di aumentare la comprensione del testo ed evitare eventuali ambiguità, 
viene fornito un glossario (\textit{Glossario v1.0.0}) contenente le 
definizioni degli acronimi e dei termini tecnici utilizzati nel documento. Ogni 
vocabolo contenuto nel glossario è contrassegnato dal pedice “\G “.

\subsection{Riferimenti}

\newpage