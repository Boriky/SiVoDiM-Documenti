\section{Visione generale della strategia di verifica}
\subsection{Definizione obiettivi}
\subsubsection{Qualità di processo}
Per garantire la qualità del prodotto è necessario perseguire la qualità dei processi che lo definiscono. Per questo motivo si è deciso di utilizzare lo standard ISO\G/IEC\G 15504 denominato SPICE, che fornisce gli strumenti necessari per valutarne l'idoneità.\newline
Al fine di applicare correttamente questo modello si deve utilizzare il ciclo 
di \textit{Deming}\G, il quale definisce una metodologia di controllo per i 
processi durante il loro ciclo di vita, per migliorarne costantemente la 
qualità.

\subsubsection{Qualità di prodotto}
Per aumentare il valore commerciale di un prodotto software, e per garantire il 
corretto funzionamento dello stesso, è necessario fissare degli obiettivi 
qualitativi e verificare che questi vengano rispettati.\newline
Lo standard ISO\G/IEC\G 9126 è stato redatto allo scopo di definire questi 
obiettivi e delineare alcune metriche capaci di misurare il raggiungimento 
degli stessi.

\subsection{Procedure di controllo di qualità di processo}
La qualità dei processi è garantita dall'applicazione del principio PDCA\G. 
Grazie a questo processo si può garantire un continuo miglioramento della 
qualità dei processi, inclusa la verifica. Questo comporta quindi un 
miglioramento dei prodotti creati.\newline \newline
Per avere il controllo dei processi, e quindi della qualità, è necessario che:
\begin{itemize}
	\item	I processi siano pianificati dettagliatamente;
	\item	Vengano ripartite chiaramente le risorse nella pianificazione;
	\item	Ci sia controllo sui processi.
\end{itemize}
L'attuazione di tali punti è descritta dettagliatamente nel \textit{Piano di 
Progetto v1.0.0}. \newline
La qualità dei processi viene inoltre controllata tramite l'analisi costante della qualità del prodotto. Un processo da migliorare è indicato da un prodotto di bassa qualità. \newline
Per quantificare la qualità dei processi verranno utilizzate le metriche 
descritte nella sezione \textbf{(((Aggiungi sezione)))}.

\subsection{Procedure di controllo di qualità di prodotto}
Il controllo di qualità dei prodotti viene garantito da:
\begin{itemize}
	\item \textbf{Quality assurance}: è l'insieme di attività realizzate al 
	fine di garantire il raggiungimento degli obiettivi di qualità. Prevede 
	l'attuazione di tecniche di analisi statica e dinamica, descritte nella 
	\textbf{(((Aggiungi sezione)))};
	\item \textbf{Verifica}: processo che determina se il risultato di una fase 
	è corretto. La verifica viene eseguita costantemente durante 
	tutta la durata del progetto. I risultati delle attività di verifica 
	eseguiti nelle varie fasi di progetto riportate \textbf{(((Aggiungi 
	sezione)))};
	\item \textbf{Validazione}: la conferma oggettiva che il sistema risponde ai requisiti.
\end{itemize}

\subsection{Organizzazione}
L'organizzazione della strategia di verifica è basata sull'utilizzo di attività 
di controllo per ogni processo attuato. Per ognuno di questi viene verificata 
la qualità, ed eventualmente la qualità del prodotto ottenuto.
Ognuna delle fasi del progetto descritte nel \textit{Piano di Progetto v1.0.0} necessita di diverse attività di verifica a causa dei differenti output:
\begin{itemize}
	\item \textbf{Analisi:} in questa fase è necessario seguire i metodi di 
	verifica descritti nelle sezioni \textbf{(((Aggiungi sezione)))} sui 
	documenti prodotti. La realizzazione di tali attività di verifica sono 
	descritte nella \textbf{(((Aggiungi sezione)))} 4.1.1.
\end{itemize}
In ogni documento viene inoltre incluso il diario delle modifiche che permette di mantenere uno storico delle attività svolte e delle relative responsabilità.

\subsection{Pianificazione strategica e temporale}
Dato che l'obiettivo è di rispettare le scadenze fissate nel \textit{Piano di 
Progetto v1.0.0}, è necessario che l'attività di verifica sia ben organizzata. 
Quindi l'individuazione e la correzione di errori dovrà essere tempestiva, in 
modo da impedire che si diffondano.\newline
Ogni attività di redazione di documenti o di codifica deve essere preceduta da un analisi della struttura e dei contenuti. Questo allo scopo di evitare imprecisioni concettuali o tecniche, rendendo l'attività di verifica più semplice, richiedendo minori correzioni.
La metodologia da seguire per individuare e correggere eventuali errori è descritta nelle \textit{Norme di Progetto v1.0.0}.

\subsection{Responsabilità}
Per garantire che il processo di verifica sia efficace e sistematico, vengono 
attribuite le responsabilità al Responsabile di Progetto ed ai Verificatori. La 
suddivisione dei compiti e le modalità sono definite nelle \textit{Norme di 
Progetto v1.0.0}.

\subsection{Risorse}
Per assicurarsi che gli obiettivi vengano raggiunti sono necessarie delle 
risorse sia umane che tecnologiche. Coloro che detengono la responsabilità 
maggiore per le attività di verifica e validazione sono il Responsabile di 
Progetto e il Verificatore. I ruoli sono descritti nel dettaglio nelle Norme di 
Progetto. \newline
Per risorse tecniche e tecnologiche sono intesi tutti gli strumenti 
\textit{software} e \textit{hardware} che il gruppo intende utilizzare. 
Affinché il lavoro dei verificatori venga semplificato, sono stati impostati 
alcuni strumenti di controllo sistematico. Questi sono descritti in modo 
accurato nelle \textit{Norme di Progetto v1.0.0}.

\subsection{Tecniche di analisi}

\subsubsection{Analisi statica}
Per analisi statica si intende una tecnica di controllo che permette di effettuare la verifica di quanto prodotto individuando errori. Essa viene svolta in due modi complementari.

\paragraph{Walkthrough}
Viene svolta una lettura critica di tutto il materiale. Questa tecnica è utile 
nelle prime fasi di progetto, quando i membri del gruppo non hanno ancora una 
adeguata esperienza che permette verifiche più mirate. \newline
Grazie a questa tecnica, il Verificatore può stilare una lista di errori più frequenti, in modo da migliorare le attività future.\newline
Questa attività è onerosa e richiede l'intervento di più persone per essere efficace ed efficiente. In seguito alla lettura segue una fase di discussione con il fine di esaminare i difetti e proporre le correzioni. La fase finale consiste nello stilare un rapporto che elenchi le modifiche effettuate.

\paragraph{Inspection}
In questa tecnica viene eseguita un'analisi mirata delle parti del documento o 
del codice che sono ritenute maggiormente fonte di errore. La lista di 
controllo, che contiene queste sezioni critiche, è redatta anticipatamente, ed 
è frutto dell'esperienza dei verificatori grazie alla tecnica 
precedente.\newline
Questa strategia è più rapida del Walkthrough in quanto riduce il numero di 
parti da analizzare. Questo comporta che la tecnica possa essere eseguita 
solamente dai verificatori, che individuano e correggono eventuali errori e 
redigono il rapporto di verifica per tracciare il lavoro svolto.

\subsubsection{Analisi dinamica}
L'analisi dinamica viene applicata solamente alla produzione di codice e viene svolta durante l'esecuzione mediante l'uso di test utilizzati per verificarne il funzionamento. \newline
Per rendere questa attività utile e generare risultati attendibili, è 
necessario che i test siano ripetibili. Questo significa che il programma, da 
un determinato input, genera sempre lo stesso output. Test di questo tipo sono 
utili per determinare la correttezza ed evidenziare eventuali problemi in un 
\textit{software}. Devono quindi essere definiti:
\begin{itemize}
	\item \textbf{Ambiente:} consiste sia del sistema hardware che di quello software sui quali è pianificato lo sviluppo. Di questi è necessario specificare lo stato iniziale dal quale iniziare i test;
	\item \textbf{Specifica:} consiste nel definire gli input e i rispettivi output che sono attesi;
	\item \textbf{Procedure:} ossia la definizione di come i test vengono svolti, con quale ordine e come vengono analizzati i risultati.
\end{itemize}
Esistono 5 tipi diversi di test: test di unità, test di integrazione, test di sistema, test di regressione e test di accettazione.

\paragraph{Test di unità}

Consta nel verificare ogni singola unità del software con l'utilizzo di 
\textit{stub}\G, \textit{driver}\G\ e \textit{logger}\G.
Con unità è inteso il minimo quantitativo di software che sia utile verificare 
singolarmente, prodotto da un singolo programmatore. Con questi test si 
verifica il funzionamento corretto dei moduli. Questo per eliminare dal sistema 
possibili errori di implementazione.

\paragraph{Test di integrazione}

Consta nel verificare i componenti del sistema che vengono aggiunti. Questo per 
analizzare che la combinazione di due o più unità software funzionino come 
previsto.
Il fine di questo tipo di test è quello di individuare errori residui dalla 
realizzazione dei moduli, o comportamenti inaspettati da componenti software 
forniti da terze parti. Per effettuare questi test è necessario aggiungere 
delle componenti fittizie per sostituire quelle ancora non sviluppate, al fine 
di non influenzare l'esito dell'analisi.

\paragraph{Test di sistema}

Consta nel validare il prodotto software nel momento in cui si ritiene che sia 
giunto ad una versione definitiva. Questo test verifica che tutti i requisiti 
software stabiliti nell'\textit{Analisi dei Requisiti v1.0.0} vengano 
rispettati.

\paragraph{Test di regressione}

Consta nell'eseguire nuovamente i test sulle componenti software, quando 
subiscono modifiche. Questo per controllare che i cambiamenti non alterino il 
corretto funzionamento di queste componenti, o di altre che non sono state 
aggiornate.
Questa operazione viene aiutata dal tracciamento, che permette di individuare e ripetere i test di unità, integrazione e di sistema, che siano stati influenzati dalla modifica.

\paragraph{Test di accettazione}

Consiste nel collaudare il prodotto software in presenza del proponente. Se questo collaudo viene superato, si può procedere al rilascio ufficiale del prodotto sviluppato.

\subsection{Misure e metriche}

Il processo di verifica deve essere quantificabile per essere informativo. Quindi è necessario stabilire a priori delle metriche su cui basare le misurazioni del processo di verifica. Essendo le metriche di natura variabile, vengono definite due tipologie di intervalli:
\begin{itemize}
	\item \textbf{Accettazione:} sono valori che vengono richiesti affichè il prodotto sia accettato;
	\item \textbf{Ottimale:} sono valori entro cui è consigliabile che la 
	misurazione si collochi. Non sono intervalli vincolanti, ma consigliati. Se 
	tali valori si scostano, è necessaria una verifica approfondita.
\end{itemize}

\subsubsection{Metriche per i processi}
Come metriche per valutare i processi si è scelto di utilizzare degli indici che analizzino i costi e i tempi.

\paragraph{Schedule Variance (SV)}

Valuta se si è in linea, in anticipo o in ritardo rispetto alla pianificazione 
della \textit{baseline}\G. Questo è un indice di efficacia.\newline
Se SV\G>0 significa che il team sta producendo con maggior velocità 
rispetto alla pianificazione, viceversa se è negativo.\\\\
\textbf{Parametri utilizzati:}
\begin{itemize}
	\item Range di accettazione: [> -(Costo preventivo fase x 5\%)];
	\item Range ottimale: [>0].
\end{itemize}

\paragraph{Budget Variance (BV)}
Valuta nella data corrente la differenza tra la spesa attuale e quanto pianificato.\newline
Se BV\G>0 significa che il progetto sta consumando il budget più lentamente di 
quanto pianificato, viceversa se negativo.\\\\
\textbf{Parametri utilizzati:}

\begin{itemize}
	\item Range di accettazione: [> -(Costo preventivo fase x 10\%)];
	\item Range ottimale: [>0].
\end{itemize}

\subsubsection{Metriche per i documenti}
Come metrica per i documenti si è scelto di utilizzare un indice di 
leggibilità. L'indice utilizzato è specifico per la lingua italiana.

\paragraph{Gulpease\G (?)}
L'indice Gulpease\G è un indice di leggibilità tarato sulla lingua italiana. 
Viene preferito rispetto ad altri poiché utilizza la lunghezza in lettere 
anziché in sillabe, semplificandone il calcolo. Questo indice evidenzia la 
complessità dello stile del documento.\newline
L'indice è calcolato secondo la seguente formula: \newline \newline
{aggiungere formula} \newline \newline
I risultati sono compresi tra 0 e 100, con 100 la migliore leggibilità e 0 la peggiore. \textbf{Parametri utilizzati:}
\begin{itemize}
	\item Range di accettazione: [40-100];
	\item Range ottimale: [50-100].
\end{itemize}

\subsubsection{Metriche per il software}
\'E desiderabile, per poter raggiungere obiettivi di qualità del software, 
applicare le metriche ora descritte.

\paragraph{Numero di livelli di annidamento}

Rappresenta il numero di livelli di annidamento dei metodi, cioè l'annidamento di strutture.\newline
Un alto livello di questo indice può essere sintomo di alta complessità del codice o un basso livello di astrazione.\newline
\textbf{Parametri utilizzati:}
\begin{itemize}
	\item Range di accettazione: [1-6];
	\item Range ottimale: [1-3].
\end{itemize}

\paragraph{Numero di attributi per classe}

Rappresenta il numero di attributi contenuti in una classe. Un alto valore potrebbe indicare la necessità di suddividere la classe in più classi.
E quindi potrebbe essere indice di un errore di progettazione.\newline
\textbf{Parametri utilizzati:}
\begin{itemize}
	\item Range di accettazione: [0-16];
	\item Range ottimale: [3-8].
\end{itemize}

\paragraph{Numero di parametri per metodo}

Rappresenta il numero di parametri dei vari metodi. Un alto valore potrebbe 
indicare un metodo con funzionalità eccessivamente complesse. E quindi errori 
di progettazione.\newline
\textbf{Parametri utilizzati:}
\begin{itemize}
	\item Range di accettazione: [0-8];
	\item Range ottimale: [0-4].
\end{itemize}

\paragraph{Linee di codice per linee di commento}

Viene calcolato come il rapporto tra le linee di commento e le linee di codice. Valori ottimali di questo parametro indicano un codice più mantenibile. \newline
\textbf{Parametri utilizzati:}
\begin{itemize}
	\item Range di accettazione: [>0.25];
	\item Range ottimale: [>0.30].
\end{itemize}

\paragraph{Copertura di codice}

Indica la percentuale di istruzioni eseguite durante i test. Maggiore è questo parametro, minore sarà la probabilità di errori nel codice. Questo valore può essere abbassato dalla presenza di metodi semplici, come \textit{setter} o \textit{getter}. \newline
\textbf{Parametri utilizzati:}
\begin{itemize}
	\item Range di accettazione: [42\%-100\%];
	\item Range ottimale: [65\%-100\%].
\end{itemize}
\newpage
