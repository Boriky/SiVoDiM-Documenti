\section{Ordine del giorno}
Di seguito sono trascritti gli argomenti che il gruppo ha trattato durante la 
riunione con la Proponente \AZIENDA. Il \textit{meeting} ha avuto carattere 
informale.

\subsection{Precisazione sullo scopo del progetto}
Il capitolato è rimasto volutamente vago, per non imporre alcun limite sul tipo 
di applicazione da realizzare. L'unico requisito primario è l'utilizzo del 
motore di sintesi, con lo scopo ultimo di metterne in risalto le potenzialità. 
Tuttavia è fortemente desiderabile che l'applicazione si dimostri utile per 
qualche utilizzo reale, in linea con gli esempi suggeriti nel capitolato.

\subsection{Motivazioni e vantaggi dello sviluppo su dispositivi mobile}
La scelta è stata dettata dal desiderio del Proponente di entrare in tale 
settore. Le motivazioni date, durante la riunione, sono le seguenti:
\begin{itemize}
	\item Incentivare la diffusione dell'applicazione;
	\item Utilità nell'avere l'applicativo su un dispositivo \textit{mobile}, 
	il quale risulta facilmente trasportabile per eventuali dimostrazioni della 
	tecnologia FA-TTS\G.
\end{itemize}

\subsection{Problematiche riguardanti la connessione al servizio remoto}
Le problematiche emerse possono essere di vario genere:
\begin{itemize}
\item La connessione può cadere;
\item I tempi di risposta della connessione potrebbero non essere ottimali:
\begin{itemize}
\item[-]I file da trasferire sono codificati in formato WAV\G\ e pertanto 
bisogna anche tenere conto del loro peso;
\item[-]L'elaborazione effettuata dal \textit{server} causa un periodo di tempo 
d'attesa.
\end{itemize}
\end{itemize}
Per poter arginare il problema del ritardo si potrebbero adottare strategie di 
\textit{caching}\G , conoscendo con anticipo i possibili testi da elaborare.


\subsection{Integrazione del modulo di sintesi con il sistema}
Il modulo software deve essere capace di integrare il servizio remoto offerto 
con il sistema \textit{mobile} nel quale viene eseguita l'applicazione. Per 
risolvere la possibile assenza di connessione, che renderebbe impossibile lo 
sfruttamento del motore di sintesi, devono essere implementati meccanismi di 
\textit{fallback}\G, utilizzando le voci presenti nel sistema stesso. In questo 
modo è possibile effettuare la sintesi con le voci di sistema e consentire 
all'applicazione di funzionare correttamente. Tale applicazione deve permettere 
all'utente di aggiungere voci in modo dinamico e di poterle salvare con 
determinati parametri all'interno del sistema.

\subsection{Implementazione separata delle varie componenti}
L'applicazione deve essere formata da quattro parti:
\begin{itemize}
	\item Un modulo per la sintesi, a sé stante, che corrisponde 
	all'implementazione del motore FA-TTS\G\ di \AZIENDA;
	\item Un'applicazione per la configurazione, che deve essere in grado di 
	interagire con il modulo per la sintesi al fine di modificarne determinati 
	parametri. Ad esempio si vuole fornire la possibilità di aggiungere nuove 
	voci, assieme a nuovi \textit{preset}\G\ di effetti associabili alle voci;
	\item Una libreria\G\ che faciliti l'utilizzo delle funzionalità 
	aggiuntive, permettendo il riuso della libreria stessa;
	\item Un'applicazione innovativa che sfrutti le suddette componenti e che dimostri in modo chiaro le potenzialità offerte dal motore FA-TTS\G.
\end{itemize}
Malgrado sia preferibile l'approccio modulare appena descritto, è stato specificato che è possibile sviluppare un'unica applicazione contenente tutte le funzionalità sopra elencate. 

\subsection{Tempo di campionamento della voce}
Il processo di campionamento richiede circa 45 minuti per essere completato. Tale attesa è determinata dalla necessità di raccogliere ed elaborare un elevato numero di dati, tant'è che solitamente occorrono più di 100 frasi per ottenere un buon risultato. Il microfono incorporato negli \textit{smartphone}\G\ odierni è adatto per ottenere un campionamento di qualità soddisfacente. Considerando l'elevato tempo di attesa, è di estrema importanza  assistere al meglio l'utente durante il processo di registrazione.

\subsection{Strumenti di lavoro offerti da MIVOQ}
\AZIENDA\ fornisce il motore di sintesi assieme alle voci predefinite, materiale
che può essere scaricato direttamente dal sito dell’azienda. Su richiesta del 
gruppo sarà possibile avere accesso ad un sito interno di \AZIENDA\ per 
campionare la propria voce e poterla riutilizzare all'interno dell’applicazione 
da realizzare. Non viene dato alcun vincolo sugli strumenti di sviluppo per 
implementare le applicazioni richieste.

\newpage

