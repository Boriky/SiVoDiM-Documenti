\section{Ordine del giorno}


\subsection{Motivazioni e vantaggi dello sviluppo mobile}
La scelta è stata dettata dal desiderio del committente di entrare in tale settore. Occorre sfruttare la praticità dei dispositivi mobili e la loro facile trasportabilità. 


\subsection{Problematiche riguardanti la connessione al servizio remoto}
Le problematiche possono essere di vario genere:
\begin{itemize}
\item la connessione può cadere;
\item i tempi di risposta della connessione potrebbero non essere ottimali:
\begin{itemize}
\item[-]i file da trasferire sono codificati in formato WAV\G\ e pertanto 
bisogna tenere conto anche del loro peso;
\item[-]l'elaborazione richiede tempo.
\end{itemize}
\end{itemize}
Per poter arginare il problema del ritardo si potrebbero adottare strategie di 
\textit{caching}\G , conoscendo con anticipo i possibili testi da elaborare.


\subsection{Integrazione del modulo di sintesi con il sistema}
Il modulo software deve essere capace di integrare il servizio remoto offerto con il sistema mobile nel quale viene eseguita l'applicazione. Per risolvere la possibile assenza di connessione, che renderebbe impossibile lo sfruttamento del motore di sintesi, devono essere 
implementati meccanismi di \textit{fallback}\G\ che utilizzino le voci presenti nel sistema stesso. In questo modo sarà possibile effettuare la sintesi con le voci di 
sistema e consentire all'applicazione di funzionare correttamente. Tale applicazione dovrà consentire all'utente di aggiungere voci in maniera 
dinamica e di salvarle con i loro parametri all'interno del sistema.

\subsection{Implementazione separata delle varie componenti}
L'applicazione deve essere formata da 4 parti:
\begin{itemize}
	\item un modulo per la sintesi, a sé stante, che corrisponde 
	all'implementazione del motore FA-TTS\G\ di Mivoq;
	\item un'applicazione per la configurazione, che deve essere in grado di 
	interagire con il modulo per la sintesi al fine di modificarne determinati 
	parametri. Ad esempio si vuole fornire la possibilità di aggiungere nuove 
	voci, assieme a nuovi \textit{preset}\G\ di effetti associabili alle voci 
	date;
	\item una libreria che faciliti l'utilizzo delle funzionalità aggiuntive, 
	permettendo il riuso della libreria stessa;
	\item un'applicazione innovativa che dimostri in modo chiaro le 
	potenzialità offerte dal suddetto modulo di sintesi.
\end{itemize}
\'E quindi possibile sviluppare due applicazioni, oppure combinare le due funzionalità in 
un'unica applicazione. 

\subsection{Tempo di campionamento della voce}
Il processo di campionamento è così lungo 
a causa della necessità di raccogliere ed elaborare un elevato numero di dati, 
tant'è che solitamente occorrono più di 100 frasi per ottenere un buon 
risultato. La strumentazione degli \textit{smartphone}\G\ odierni è adatta ad ottenere un 
buon campionamento tuttavia sarà di estrema importanza  
assistere al meglio l'utente durante tale processo in modo da rendere più piacevole e meno stressante l'attesa. 


