\section{Requisiti}
Di seguito si riportano tutti i requisiti individuati, che derivano da casi d’uso, dal capitolato, dai \textit{meeting} con il Proponente e dalle esigenze del gruppo \GRUPPO. Sono divisi in tabelle separate a seconda della loro categoria. Di ogni requisito si specificano la tipologia, la priorità, la descrizione e ne viene indicata la provenienza.
I requisiti prodotti devono essere classificati a seconda del tipo e dell'importanza, rispettando la seguente notazione:

\begin{center}
	R[Importanza][Tipo][Codice]
\end{center}

\begin{itemize}
	\item\textbf{Importanza}: i valori che può assumere sono:
	
	\begin{itemize}
		\item[-] 0: requisito obbligatorio;
		\item[-] 1: requisito desiderabile;
		\item[-] 2: requisito opzionale.
	\end{itemize}
	
	\item\textbf{Tipo}: i valori che può assumere sono:
		\begin{itemize}
			\item[-] F: requisito funzionale;
			\item[-] P: requisito prestazionale;
			\item[-] Q: requisito di qualità;
			\item[-] V: requisito di vincolo.
		\end{itemize}
	
	\item\textbf{Codice}: è il codice gerarchico e univoco del vincolo espresso nella forma X.Y.Z dove X, Y e Z sono dei valori numerici.
\end{itemize}
Inoltre, ogni requisito deve contenere le seguenti informazioni:
	
\begin{itemize}
	\item\textbf{Descrizione}: descrizione del requisito con la minore ambiguità possibile;
	\item\textbf{Fonte}: la scelta può ricadere tra:
	\begin{itemize}
		\item[-] \textbf{Capitolato}: requisito ottenuto dalle specifiche del capitolato;
		\item[-] \textbf{Interno}: requisito elaborato dagli \textit{Analisti} nel corso di un'analisi più approfondita del problema;
		\item[-] \textbf{Caso d'uso}: requisito ottenuto da uno o più casi d'uso. Deve essere quindi specificato il codice del caso d'uso a cui ci si riferisce;
		\item[-] \textbf{Verbale}: requisito ottenuto da un incontro con il Proponente o da riunioni interne tra i membri del gruppo di lavoro \GRUPPO.
	\end{itemize}
\end{itemize}

\newpage

\subsection{Requisiti funzionali}
\begin{center}
\def\arraystretch{1.6}
\bgroup
\begin{longtable}{| p{2.5cm} | p{3cm} | p{5.25cm} | p{2cm} |}
\hline
\textbf{Requisito} & \textbf{Tipologia} & \textbf{Descrizione} & \textbf{Fonti}\\ \hline \hline  R0F1 & Funzionale \newline Obbligatorio & Il programma deve permettere la creazione di sceneggiati &  \hyperref[sec:UC1.1]{ UC1.1 }  \\ \hline  R0F1.1 & Funzionale \newline Obbligatorio & L'utente deve poter assegnare un titolo allo sceneggiato &  \hyperref[sec:UC1.1.6]{ UC1.1.6 }  \\ \hline  R0F1.1.1 & Funzionale \newline Obbligatorio & L'utente deve poter modificare il titolo ad uno sceneggiato &  \hyperref[sec:UC1.1.6]{ UC1.1.6 } \newline \hyperref[sec:UC1.8]{ UC1.8 }  \\ \hline  R0F1.2 & Funzionale \newline Obbligatorio & L'utente deve poter creare un nuovo personaggio per lo sceneggiato &  \hyperref[sec:UC1.1.1]{ UC1.1.1 }  \\ \hline  R0F1.2.1 & Funzionale \newline Obbligatorio & L'utente deve poter assegnare un nome al personaggio creato &  \hyperref[sec:UC1.1.1.1]{ UC1.1.1.1 }  \\ \hline  R0F1.2.1.1 & Funzionale \newline Obbligatorio & L'utente deve poter modificare il nome associato ad un personaggio &  \hyperref[sec:UC1.1.1.1]{ UC1.1.1.1 }  \\ \hline  R0F1.2.2 & Funzionale \newline Obbligatorio & L'utente deve poter assegnare un \textit{avatar}\G\ al personaggio creato &  \hyperref[sec:UC1.1.1.3]{ UC1.1.1.3 }  \\ \hline  R0F1.2.2.1 & Funzionale \newline Obbligatorio & L'utente deve poter aprire un'immagine &  \hyperref[sec:UC1.1.1.3]{ UC1.1.1.3 } \newline \hyperref[sec:UC1.7.1]{ UC1.7.1 }  \\ \hline  R1F1.2.2.1.1 & Funzionale \newline Desiderabile & L'utente deve poter navigare nel \textit{filesystem}\G\ per aprire un'immagine &  \hyperref[sec:UC1.7.1]{ UC1.7.1 }  \\ \hline  R1F1.2.3 & Funzionale \newline Desiderabile & L'utente deve poter cambiare l'\textit{avatar}\G\ al personaggio creato &  \hyperref[sec:UC1.1.1.3]{ UC1.1.1.3 }  \\ \hline  R0F1.2.4 & Funzionale \newline Obbligatorio & L'utente deve poter assegnare una voce al personaggio creato &  \hyperref[sec:UC1.1.1.2]{ UC1.1.1.2 } \newline \hyperref[sec:UC1.1.1.2.2]{ UC1.1.1.2.2 }  \\ \hline  R0F1.2.4.1 & Funzionale \newline Obbligatorio & L'utente deve poter visualizzare le voci disponibili  &  \hyperref[sec:UC1.1.1.2.1]{ UC1.1.1.2.1 }  \\ \hline  R0F1.2.5 & Funzionale \newline Obbligatorio & L'utente deve poter modificare la voce di un personaggio &  \hyperref[sec:UC1.1.1.2.2]{ UC1.1.1.2.2 }  \\ \hline  R0F1.2.5.1 & Funzionale \newline Obbligatorio & L'utente deve poter modificare i parametri che definiscono una voce &  \hyperref[sec:UC1.1.1.5]{ UC1.1.1.5 }  \\ \hline  R0F1.2.5.1.1 & Funzionale \newline Obbligatorio & L'utente deve poter modificare il nome associato ad una voce &  \hyperref[sec:UC1.1.1.5]{ UC1.1.1.5 }  \\ \hline  R0F1.2.6 & Funzionale \newline Obbligatorio & L'utente deve poter scrivere una battuta &  \hyperref[sec:UC1.1.2.3]{ UC1.1.2.3 }  \\ \hline  R0F1.2.6.1 & Funzionale \newline Obbligatorio & L'utente deve poter associare un personaggio ad ogni battuta &  \hyperref[sec:UC1.1.2.3.1]{ UC1.1.2.3.1 } \newline \hyperref[sec:UC1.1.2.3.2]{ UC1.1.2.3.2 }  \\ \hline  R0F1.2.6.2 & Funzionale \newline Obbligatorio & L'utente deve poter modificare il personaggio associato ad una battuta &  \hyperref[sec:UC1.1.1]{ UC1.1.1 }  \\ \hline  R0F1.2.6.3 & Funzionale \newline Obbligatorio & L'utente deve poter cancellare una battuta &  \hyperref[sec:UC1.1.2.5]{ UC1.1.2.5 }  \\ \hline  R0F1.2.6.4 & Funzionale \newline Obbligatorio & L'utente deve poter associare un sentimento ad ogni battuta  &  \hyperref[sec:UC1.1.2.6]{ UC1.1.2.6 }  \\ \hline  R0F1.2.6.4.1 & Funzionale \newline Obbligatorio & L'utente deve poter visualizzare i sentimenti disponibili &  \hyperref[sec:UC1.1.3.2]{ UC1.1.3.2 }  \\ \hline  R0F1.2.6.5 & Funzionale \newline Obbligatorio & L'utente deve poter modificare il sentimento associato ad una battuta &  \hyperref[sec:UC1.1.2.4.3]{ UC1.1.2.4.3 }  \\ \hline  R0F1.2.6.6 & Funzionale \newline Obbligatorio & L'utente deve poter cambiare il personaggio associato ad ogni battuta &  \hyperref[sec:UC1.1.2.4.4]{ UC1.1.2.4.4 }  \\ \hline  R0F1.2.6.7 & Funzionale \newline Obbligatorio & L'utente deve poter modificare il testo di ogni battuta &  \hyperref[sec:UC1.1.2.4.2]{ UC1.1.2.4.2 }  \\ \hline  R2F1.2.6.8 & Funzionale \newline Opzionale & L'utente deve poter ascoltare ogni battuta singolarmente &  \hyperref[sec:UC1.1.2.6.2]{ UC1.1.2.6.2 }  \\ \hline  R0F1.2.6.9 & Funzionale \newline Obbligatorio & L'utente deve poter selezionare una battuta &  \hyperref[sec:UC1.2.4.1]{ UC1.2.4.1 }  \\ \hline  R0F1.2.7 & Funzionale \newline Obbligatorio & L'utente deve poter visualizzare l'elenco dei personaggi a disposizione &  \hyperref[sec:UC1.1.2.3.1]{ UC1.1.2.3.1 }  \\ \hline  R0F1.2.8 & Funzionale \newline Obbligatorio & L'utente deve poter creare una nuova voce &  \hyperref[sec:UC1.1.1.4]{ UC1.1.1.4 }  \\ \hline  R0F1.2.8.1 & Funzionale \newline Obbligatorio & Al momento della creazione di una nuova voce, l'utente deve poter selezionare la lingua &  \hyperref[sec:UC1.1.1.4.1]{ UC1.1.1.4.1 }  \\ \hline  R0F1.2.8.2 & Funzionale \newline Obbligatorio & Al momento della creazione di una nuova voce, l'utente deve poter selezionare il sesso &  \hyperref[sec:UC1.1.1.4.2]{ UC1.1.1.4.2 }  \\ \hline  R0F1.2.8.3 & Funzionale \newline Obbligatorio & Al momento della creazione di una nuova voce, l'utente può modificare i parametri di una voce &  \hyperref[sec:UC1.1.1.4.3]{ UC1.1.1.4.3 }  \\ \hline  R0F1.2.8.4 & Funzionale \newline Obbligatorio & Al momento della creazione di una nuova voce, l'utente deve poter assegnare un nome alla stessa &  \hyperref[sec:Capitolato]{ Capitolato } \newline \hyperref[sec:UC1.1.1.4.4]{ UC1.1.1.4.4 }  \\ \hline  R0F1.3 & Funzionale \newline Obbligatorio & L'utente deve poter cancellare un personaggio &  \hyperref[sec:UC1.1.1]{ UC1.1.1 }  \\ \hline  R1F1.4 & Funzionale \newline Desiderabile & L'utente deve poter condividere l'audio dello sceneggiato &  \hyperref[sec:UC1.10]{ UC1.10 }  \\ \hline  R1F1.4.1 & Funzionale \newline Desiderabile & L'utente deve poter navigare fra le opzioni di condivisione &  \hyperref[sec:UC1.2.1]{ UC1.2.1 }  \\ \hline  R1F1.4.2 & Funzionale \newline Desiderabile & L'applicazione deve reperire gli strumenti di condivisione disponibili nel sistema &  \hyperref[sec:UC1.2.3]{ UC1.2.3 }  \\ \hline  R1F1.4.3 & Funzionale \newline Desiderabile & L'utente deve poter selezionare lo strumento di condivisione che preferisce &  \hyperref[sec:UC1.2.2]{ UC1.2.2 }  \\ \hline  R0F1.6 & Funzionale \newline Obbligatorio & L'utente deve poter esportare lo sceneggiato in un \textit{file} audio &  \hyperref[sec:UC1.3]{ UC1.3 }  \\ \hline  R1F1.6.1 & Funzionale \newline Desiderabile & L'utente deve poter selezionare il formato audio di esportazione &  \hyperref[sec:UC1.3.1]{ UC1.3.1 }  \\ \hline  R0F1.6.1.2 & Funzionale \newline Obbligatorio & Deve essere possibile esportare lo sceneggiato nel formato wav\G\ &  \hyperref[sec:UC1.3]{ UC1.3 }  \\ \hline  R0F1.6.2 & Funzionale \newline Obbligatorio & L'utente deve poter indicare il nome di salvataggio del \textit{file} audio &  \hyperref[sec:UC1.3.2]{ UC1.3.2 }  \\ \hline  R0F1.7 & Funzionale \newline Obbligatorio & L'utente deve poter salvare lo sceneggiato &  \hyperref[sec:UC1.5]{ UC1.5 }  \\ \hline  R0F1.7.1 & Funzionale \newline Obbligatorio & L'utente deve poter scegliere con che nome salvare lo sceneggiato &  \hyperref[sec:UC1.5]{ UC1.5 }  \\ \hline  R2F1.7.3 & Funzionale \newline Opzionale & L'utente deve poter salvare il \textit{file} video esportato &  \hyperref[sec:UC1.13]{ UC1.13 }  \\ \hline  R1F1.8.3 & Funzionale \newline Desiderabile & All'avvio dell'applicazione deve essere aperto l'ultimo sceneggiato creato &  \hyperref[sec:UC1.9]{ UC1.9 }  \\ \hline  R0F1.9 & Funzionale \newline Obbligatorio & L'utente deve poter aprire uno sceneggiato salvato in precedenza &  \hyperref[sec:UC1.7.2]{ UC1.7.2 }  \\ \hline  R0F1.9.1 & Funzionale \newline Obbligatorio & Deve essere disponibile una lista contenente tutti gli sceneggiati salvati  &  \hyperref[sec:UC1.7]{ UC1.7 }  \\ \hline  R1F1.10 & Funzionale \newline Desiderabile & L'utente deve poter ascoltare una parte selezionata dello sceneggiato &  \hyperref[sec:UC1.1.2.7]{ UC1.1.2.7 } \newline \hyperref[sec:UC1.8.2]{ UC1.8.2 }  \\ \hline  R0F1.11 & Funzionale \newline Obbligatorio & L'utente può creare un nuovo capitolo &  \hyperref[sec:UC1.1.2]{ UC1.1.2 }  \\ \hline  R0F1.11.1 & Funzionale \newline Obbligatorio & L'utente deve poter modificare un capitolo &  \hyperref[sec:UC1.8.1]{ UC1.8.1 }  \\ \hline  R0F1.11.1.2 & Funzionale \newline Obbligatorio & L'utente deve poter modificare il titolo di un capitolo &  \hyperref[sec:UC1.8.1.4]{ UC1.8.1.4 }  \\ \hline  R1F1.11.1.4 & Funzionale \newline Desiderabile & L'utente deve poter modificare lo sfondo assegnato ad un capitolo &  \hyperref[sec:UC1.8.1.3]{ UC1.8.1.3 }  \\ \hline  R1F1.11.1.8 & Funzionale \newline Desiderabile & L'utente deve poter modificare l'immagine di sfondo &  \hyperref[sec:UC1.8.1.3]{ UC1.8.1.3 }  \\ \hline  R2F1.11.1.9 & Funzionale \newline Opzionale & L'utente deve poter modificare la traccia audio di sfondo di un capitolo &  \hyperref[sec:UC1.8.1.3]{ UC1.8.1.3 }  \\ \hline  R0F1.11.1.11 & Funzionale \newline Obbligatorio & L'utente deve poter selezionare un singolo capitolo &  \hyperref[sec:UC1.8.1.1]{ UC1.8.1.1 }  \\ \hline  R0F1.11.2 & Funzionale \newline Obbligatorio & L'utente deve poter assegnare un titolo al capitolo &  \hyperref[sec:UC1.1.2.1]{ UC1.1.2.1 }  \\ \hline  R1F1.11.3 & Funzionale \newline Desiderabile & L'utente deve avere a disposizione un'anteprima del capitolo &  \hyperref[sec:UC1.8.1.2]{ UC1.8.1.2 }  \\ \hline  R1F1.11.4 & Funzionale \newline Desiderabile & L'utente deve poter inserire uno sfondo da associare ad un capitolo &  \hyperref[sec:UC1.1.2.2]{ UC1.1.2.2 }  \\ \hline  R1F1.11.4.1 & Funzionale \newline Desiderabile & L'utente deve poter caricare un'immagine di sfondo &  \hyperref[sec:UC1.1.2.2.1]{ UC1.1.2.2.1 }  \\ \hline  R1F1.11.4.2 & Funzionale \newline Desiderabile & L'utente deve poter inserire un sottofondo sonoro &  \hyperref[sec:UC1.1.2.2.2]{ UC1.1.2.2.2 }  \\ \hline  R1F1.11.5 & Funzionale \newline Desiderabile & L'utente deve poter inserire un suono tra le battute &  \hyperref[sec:UC1.1.2.8]{ UC1.1.2.8 }  \\ \hline  R2F1.11.6 & Funzionale \newline Opzionale & L'utente deve poter caricare un suono da inserire tra le battute &  \hyperref[sec:UC1.1.2.8]{ UC1.1.2.8 }  \\ \hline  R0F1.11.7 & Funzionale \newline Obbligatorio & L'utente deve poter togliere un suono precedentemente inserito &  \hyperref[sec:UC1.8.1.5]{ UC1.8.1.5 }  \\ \hline  R1F1.12 & Funzionale \newline Desiderabile & L'utente deve poter esportare lo sceneggiato in un video &  \hyperref[sec:UC1.4]{ UC1.4 }  \\ \hline  R2F1.12.1 & Funzionale \newline Opzionale & L'utente deve poter scegliere il formato video di esportazione &  \hyperref[sec:UC1.3.1]{ UC1.3.1 }  \\ \hline  R1F1.13 & Funzionale \newline Desiderabile & L'utente deve poter cambiare l'ordine dei capitoli &  \hyperref[sec:UC1.1.4]{ UC1.1.4 }  \\ \hline  R0F2 & Funzionale \newline Obbligatorio & L'utente deve poter creare delle configurazioni di effetti personali &  \hyperref[sec:UC2.1]{ UC2.1 }  \\ \hline  R0F2.1 & Funzionale \newline Obbligatorio & L'utente deve poter navigare fra le configurazioni di effetti disponibili &  \hyperref[sec:UC2.1.1]{ UC2.1.1 }  \\ \hline  R0F2.2 & Funzionale \newline Obbligatorio & Il modulo di sistema deve reperire gli effetti disponibili &  \hyperref[sec:UC3.2]{ UC3.2 }  \\ \hline  R0F2.3 & Funzionale \newline Obbligatorio & L'utente deve poter impostare i parametri di ogni configurazione di effetti &  \hyperref[sec:UC2.1.2]{ UC2.1.2 }  \\ \hline  R0F2.4 & Funzionale \newline Obbligatorio & L'utente deve poter testare le configurazioni di effetti ascoltando la sintesi di un testo di prova &  \hyperref[sec:UC2.4]{ UC2.4 }  \\ \hline  R0F2.5 & Funzionale \newline Obbligatorio & L'utente deve poter modificare i parametri di una configurazione di effetti &  \hyperref[sec:UC2.5]{ UC2.5 }  \\ \hline  R0F3 & Funzionale \newline Obbligatorio & L'utente deve poter salvare le configurazioni di effetti create &  \hyperref[sec:UC2.1]{ UC2.1 }  \\ \hline  R0F3.1 & Funzionale \newline Obbligatorio & L'utente deve poter decidere il nome con cui salvare una configurazione di effetti &  \hyperref[sec:UC2.1.3]{ UC2.1.3 }  \\ \hline  R2F4 & Funzionale \newline Opzionale & L'utente deve poter selezionare una voce per personalizzare il servizio TTS\G\ di sistema &  \hyperref[sec:UC2.2]{ UC2.2 }  \\ \hline  R0F4.1 & Funzionale \newline Obbligatorio & L'utente deve poter navigare fra le voci di sistema disponibili &  \hyperref[sec:UC2.2]{ UC2.2 } \newline \hyperref[sec:UC2.2.1]{ UC2.2.1 }  \\ \hline  R0F5 & Funzionale \newline Obbligatorio & Il modulo di sistema deve poter reperire le voci disponibili nel sistema &  \hyperref[sec:UC3.1]{ UC3.1 }  \\ \hline  R0F6 & Funzionale \newline Obbligatorio & Il modulo di sistema deve poter reperire le voci offerte dal server di \AZIENDA &  \hyperref[sec:UC3.1]{ UC3.1 }  \\ \hline  R0F7 & Funzionale \newline Obbligatorio & Il modulo di sistema deve poter comunicare con il server di \AZIENDA &  \hyperref[sec:UC3.3]{ UC3.3 }  \\ \hline  R0F7.1 & Funzionale \newline Obbligatorio & Il modulo di sistema deve essere in grado di creare una corretta richiesta HTTP\G &  \hyperref[sec:UC3.3.1]{ UC3.3.1 }  \\ \hline  R0F7.2 & Funzionale \newline Obbligatorio & Il modulo di sistema deve essere in grado di ricevere le risposte delle richieste effettuate dal server di \AZIENDA &  \hyperref[sec:UC3.3.3]{ UC3.3.3 }  \\ \hline  R0F9 & Funzionale \newline Obbligatorio & Il modulo di sistema deve ottenere dal server di \AZIENDA\ la lista degli effetti disponibili &  \hyperref[sec:UC3.2]{ UC3.2 }  \\ \hline  R0F10 & Funzionale \newline Obbligatorio & Il modulo di sistema deve essere in grado di formulare una richiesta HTTP\G\ per la ricezione degli effetti &  \hyperref[sec:UC3.2]{ UC3.2 }  \\ \hline  R1F12 & Funzionale \newline Desiderabile & Il modulo di sistema deve integrarsi con il servizio TTS\G\ del sistema, per mettere a disposizione nuove voci alle applicazioni installate &  \hyperref[sec:Capitolato]{ Capitolato }  \\ \hline  R1F14 & Funzionale \newline Desiderabile & Il modulo di sistema deve poter convertire una battuta in SSML\G  &  \hyperref[sec:UC1.1.2.3.4]{ UC1.1.2.3.4 }  \\ \hline  R1F15 & Funzionale \newline Desiderabile & L'utente deve poter condividere un video &  \hyperref[sec:UC1.11]{ UC1.11 }  \\ \hline  R1F15.1 & Funzionale \newline Desiderabile & L'utente deve poter scegliere se condividere audio o video &  \hyperref[sec:UC1.2.4]{ UC1.2.4 }  \\ \hline  R0F16 & Funzionale \newline Obbligatorio & L'utente deve poter modificare uno sceneggiato già creato &  \hyperref[sec:UC1.8]{ UC1.8 }  \\ \hline  R1F16.2 & Funzionale \newline Desiderabile & L'applicazione deve fornire un'anteprima dello sceneggiato su richiesta dell'utente &  \hyperref[sec:UC1.8.2]{ UC1.8.2 }  \\ \hline  R2F17 & Funzionale \newline Opzionale & L'utente deve avere la possibilità di campionare la propria voce &  \hyperref[sec:UC2.3]{ UC2.3 }  \\ \hline  R2F17.1 & Funzionale \newline Opzionale & L'utente deve poter effettuare la \textit{login}\G\ al sito di \AZIENDA &  \hyperref[sec:UC2.3.4]{ UC2.3.4 }  \\ \hline  R2F17.2 & Funzionale \newline Opzionale & L'utente deve poter visualizzare le frasi da leggere nella fase di campionamento &  \hyperref[sec:UC2.3.1]{ UC2.3.1 }  \\ \hline  R2F17.3 & Funzionale \newline Opzionale & L'utente deve potersi registrare durante la lettura delle frasi necessaria per il campionamento &  \hyperref[sec:UC2.3.2]{ UC2.3.2 }  \\ \hline  R2F17.4 & Funzionale \newline Opzionale & L'applicazione deve essere in grado di inviare al server di \AZIENDA\ le registrazioni necessarie per il campionamento &  \hyperref[sec:UC2.3.3]{ UC2.3.3 }  \\ \hline  R2F17.5 & Funzionale \newline Opzionale & L'utente deve poter interrompere il processo di campionamento senza perdere i dati già salvati &  \hyperref[sec:UC2.3]{ UC2.3 }  \\ \hline  R2F17.6 & Funzionale \newline Opzionale & L'utente deve poter riprendere il processo di campionamento in qualsiasi momento  &  \hyperref[sec:UC2.3]{ UC2.3 }  \\ \hline  R2F18 & Funzionale \newline Opzionale & L'utente deve potersi registrare al sito di \AZIENDA &  \hyperref[sec:UC2.3]{ UC2.3 }  \\ \hline  R0F19 & Funzionale \newline Obbligatorio & L'utente deve poter aprire un audio &  \hyperref[sec:UC1.7]{ UC1.7 }  \\ \hline  R0F19.1 & Funzionale \newline Obbligatorio & L'utente deve poter navigare nel \textit{filesystem}\G\ per aprire un file audio &  \hyperref[sec:UC1.7.0.1]{ UC1.7.0.1 }  \\ \hline \caption{Requisiti funzionali}
\end{longtable}
\egroup
\end{center} \newpage \subsection{Requisiti qualitativi}
\begin{center}
\def\arraystretch{1.6}
\bgroup
\begin{longtable}{| p{2.5cm} | p{3cm} | p{5.25cm} | p{2cm} |}
\hline
\textbf{Requisito} & \textbf{Tipologia} & \textbf{Descrizione} & \textbf{Fonti}\\ \hline \hline  R0Q1.5 & Di qualità \newline Obbligatorio & Devono essere offerti meccanismi di annullamento delle operazioni svolte dall'utente &  \hyperref[sec:Accessibilità ]{ Accessibilità  }  \\ \hline  R1Q1.6.3 & Di qualità \newline Desiderabile & Nel caso la condivisione fallisse deve essere fornita come opzione all'utente l'esportazione e il salvataggio in locale del \textit{file} audio &  \hyperref[sec:UC1.2]{ UC1.2 } \newline \hyperref[sec:UC1.3]{ UC1.3 }  \\ \hline  R1Q1.7.2 & Di qualità  \newline Desiderabile & Nel caso il salvataggio non andasse a buon fine deve essere fornito all'utente un messaggio di errore &  \hyperref[sec:UC1.5]{ UC1.5 } \newline \hyperref[sec:UC1.5.3]{ UC1.5.3 }  \\ \hline  R1Q1.8 & Di qualità  \newline Desiderabile & Il sistema deve effettuare salvataggi automatici ogni minuto &  \hyperref[sec:UC1.9]{ UC1.9 }  \\ \hline  R1Q1.8.2 & Di qualità \newline Desiderabile & Deve essere possibile recuperare la sessione corrente in caso di \textit{crash}\G\ del programma &  \hyperref[sec:UC1.9]{ UC1.9 }  \\ \hline  R1Q7.3 & Di qualità  \newline Desiderabile & In caso di connessione assente l'utente deve poter visualizzare un opportuno messaggio di errore &  \hyperref[sec:UC3.3.4]{ UC3.3.4 }  \\ \hline  R1Q7.4 & Di qualità  \newline Desiderabile & In caso la connessione venisse meno a processo avviato, l'utente deve poter visualizzare un opportuno messaggio di errore &  \hyperref[sec:UC3.3]{ UC3.3 } \newline \hyperref[sec:UC3.3.4]{ UC3.3.4 }  \\ \hline  R2Q8 & Di qualità  \newline Opzionale & In caso di assenza di connessione l'utente deve poter utilizzare l'applicazione con le voci di sistema &  \hyperref[sec:Capitolato]{ Capitolato }  \\ \hline  R1Q11 & Di qualità  \newline Desiderabile & Il modulo di sistema deve essere in grado di mantenere nella memoria l'audio ricevuto dal server \AZIENDA  &  \hyperref[sec:UC3.4]{ UC3.4 }  \\ \hline  R1Q20 & Di qualità  \newline Desiderabile & L'applicazione deve utilizzare meccanismi di \textit{caching}\G\ per ridurre il tempo di attesa dovuto alla connessione &  \hyperref[sec:Capitolato]{ Capitolato }  \\ \hline  R0Q22 & Di qualità \newline Obbligatorio & La conversione in mp3\G\ avviene solo dopo l'assemblamento di tutte le parti in formato wav\G\ &  \hyperref[sec:Capitolato]{ Capitolato }  \\ \hline  R1Q23 & Di qualità \newline Desiderabile & Deve essere garantita l'esportazione in mp3\G\ per risparmiare spazio nel dispositivo  &  \\ \hline \caption{Requisiti di qualità}
\end{longtable}
\egroup
\end{center} \newpage \subsection{Requisiti prestazionali}
\begin{center}
\def\arraystretch{1.6}
\bgroup
\begin{longtable}{| p{2.5cm} | p{3cm} | p{5.25cm} | p{2cm} |}
\hline
\textbf{Requisito} & \textbf{Tipologia} & \textbf{Descrizione} & \textbf{Fonti}\\ \hline \hline  R0P20.1 & Prestazionale \newline Obbligatorio & Ogni volta che una battuta viene creata o modificata il modulo di sistema deve richiedere l'audio al server e salvarlo nella memoria locale  &  \hyperref[sec:Capitolato]{ Capitolato }  \\ \hline \caption{Requisiti prestazionali}
\end{longtable}
\egroup
\end{center} \newpage \subsection{Vincoli}
\begin{center}
\def\arraystretch{1.6}
\bgroup
\begin{longtable}{| p{2.5cm} | p{3cm} | p{5.25cm} | p{2cm} |}
\hline
\textbf{Requisito} & \textbf{Tipologia} & \textbf{Descrizione} & \textbf{Fonti}\\ \hline \hline  R1V1.6.1.1 & Vincolo \newline Desiderabile & Deve essere possibile esportare nel formato mp3\G\ &  \hyperref[sec:UC1.3]{ UC1.3 }  \\ \hline  R0V1.8.1 & Vincolo \newline Obbligatorio & L'applicazione deve salvare i \textit{file} in una cartella di default &  \hyperref[sec:UC1.12]{ UC1.12 }  \\ \hline  R0V13 & Vincolo \newline Obbligatorio & Il modulo di sistema deve integrarsi con il sistema Android\G &  \hyperref[sec:Capitolato]{ Capitolato }  \\ \hline  R0V21 & Vincolo \newline Obbligatorio & Il modulo di sistema deve utilizzare il linguaggio SSML\G &  \hyperref[sec:Verbale 18/03/2016]{ Verbale 18/03/2016 }  \\ \hline  R1V24 & Vincolo \newline Desiderabile & Il sistema deve essere compatibile con versioni di Android\G\ a partire dalla 4.0 &  \\ \hline  R0V25 & Vincolo \newline Obbligatorio & Il progetto deve essere pubblicato su GitHub\G &  \\ \hline  R0V26 & Vincolo \newline Obbligatorio & Il sistema deve permettere l’interrogazione alle API\G\ di \AZIENDA &  \\ \hline \caption{Vincoli}
\end{longtable}
\egroup
\end{center} \newpage \subsection{Tracciamento Fonti-Requisiti}
\begin{center}
\def\arraystretch{1.6}
\bgroup
\begin{longtable}{| p{10.25cm} | p{2.5cm} | }
\hline
\textbf{Fonte} & \textbf{Requisito} \\ \hline \hline  \hyperref[sec:Accessibilità ]{ Accessibilità   } &  R0Q1.5  \\ \hline  \hyperref[sec:Capitolato]{ Capitolato  } &  R2Q8  \newline R1F12  \newline R0V13  \newline R0F1.2.8.4  \newline R1Q20  \newline R0P20.1  \newline R0Q22  \\ \hline  \hyperref[sec:UC1.1]{ UC1.1 Creazione sceneggiato } &  R0F1  \\ \hline  \hyperref[sec:UC1.1.1]{ UC1.1.1 Creazione personaggio } &  R0F1.2  \newline R0F1.2.6.2  \newline R0F1.3  \\ \hline  \hyperref[sec:UC1.1.1.1]{ UC1.1.1.1 Assegnazione nome } &  R0F1.2.1  \newline R0F1.2.1.1  \\ \hline  \hyperref[sec:UC1.1.1.2]{ UC1.1.1.2 Assegnazione voce } &  R0F1.2.4  \\ \hline  \hyperref[sec:UC1.1.1.2.1]{ UC1.1.1.2.1 Navigazione fra le voci disponibili } &  R0F1.2.4.1  \\ \hline  \hyperref[sec:UC1.1.1.2.2]{ UC1.1.1.2.2 Selezione voce } &  R0F1.2.4  \newline R0F1.2.5  \\ \hline  \hyperref[sec:UC1.1.1.3]{ UC1.1.1.3 Assegnazione \textit{avatar}\G } &  R0F1.2.2  \newline R1F1.2.3  \newline R0F1.2.2.1  \\ \hline  \hyperref[sec:UC1.1.1.4]{ UC1.1.1.4 Creazione nuova voce } &  R0F1.2.8  \\ \hline  \hyperref[sec:UC1.1.1.4.1]{ UC1.1.1.4.1 Selezione lingua } &  R0F1.2.8.1  \\ \hline  \hyperref[sec:UC1.1.1.4.2]{ UC1.1.1.4.2 Selezione sesso } &  R0F1.2.8.2  \\ \hline  \hyperref[sec:UC1.1.1.4.3]{ UC1.1.1.4.3 Modifica parametri di effetti } &  R0F1.2.8.3  \\ \hline  \hyperref[sec:UC1.1.1.4.4]{ UC1.1.1.4.4 Assegnazione nome } &  R0F1.2.8.4  \\ \hline  \hyperref[sec:UC1.1.1.5]{ UC1.1.1.5 Modifica voce esistente } &  R0F1.2.5.1  \newline R0F1.2.5.1.1  \\ \hline  \hyperref[sec:UC1.1.2]{ UC1.1.2 Creazione capitolo } &  R0F1.11  \\ \hline  \hyperref[sec:UC1.1.2.1]{ UC1.1.2.1 Assegnazione titolo } &  R0F1.11.2  \\ \hline  \hyperref[sec:UC1.1.2.2]{ UC1.1.2.2 Creazione sfondo } &  R1F1.11.4  \\ \hline  \hyperref[sec:UC1.1.2.2.1]{ UC1.1.2.2.1 Caricamento immagine di sfondo } &  R1F1.11.4.1  \\ \hline  \hyperref[sec:UC1.1.2.2.2]{ UC1.1.2.2.2 Caricamento sottofondo sonoro } &  R1F1.11.4.2  \\ \hline  \hyperref[sec:UC1.1.2.3]{ UC1.1.2.3 Scrittura battute } &  R0F1.2.6  \\ \hline  \hyperref[sec:UC1.1.2.3.1]{ UC1.1.2.3.1 Navigazione tra i personaggi } &  R0F1.2.6.1  \newline R0F1.2.7  \\ \hline  \hyperref[sec:UC1.1.2.3.2]{ UC1.1.2.3.2 Scelta personaggio } &  R0F1.2.6.1  \\ \hline  \hyperref[sec:UC1.1.2.3.4]{ UC1.1.2.3.4 Conversione in SSML\G  } &  R1F14  \\ \hline  \hyperref[sec:UC1.1.2.4.2]{ UC1.1.2.4.2 Cambio testo } &  R0F1.2.6.7  \\ \hline  \hyperref[sec:UC1.1.2.4.3]{ UC1.1.2.4.3 Cambio sentimento } &  R0F1.2.6.5  \\ \hline  \hyperref[sec:UC1.1.2.4.4]{ UC1.1.2.4.4 Cambio personaggio } &  R0F1.2.6.6  \\ \hline  \hyperref[sec:UC1.1.2.5]{ UC1.1.2.5 Cancellazione battuta } &  R0F1.2.6.3  \\ \hline  \hyperref[sec:UC1.1.2.6]{ UC1.1.2.6 Associazione battuta-effetto } &  R0F1.2.6.4  \\ \hline  \hyperref[sec:UC1.1.2.6.2]{ UC1.1.2.6.2 Anteprima battuta } &  R2F1.2.6.8  \\ \hline  \hyperref[sec:UC1.1.2.7]{ UC1.1.2.7 Anteprima scena } &  R1F1.10  \\ \hline  \hyperref[sec:UC1.1.2.8]{ UC1.1.2.8 Inserimento suono } &  R1F1.11.5  \newline R2F1.11.6  \\ \hline  \hyperref[sec:UC1.1.3.2]{ UC1.1.3.2 Navigazione tra i sentimenti } &  R0F1.2.6.4.1  \\ \hline  \hyperref[sec:UC1.1.4]{ UC1.1.4 Riordinamento capitoli } &  R1F1.13  \\ \hline  \hyperref[sec:UC1.1.6]{ UC1.1.6 Assegnazione titolo } &  R0F1.1  \newline R0F1.1.1  \\ \hline  \hyperref[sec:UC1.10]{ UC1.10 Condivisione audio } &  R1F1.4  \\ \hline  \hyperref[sec:UC1.11]{ UC1.11 Condivisione video } &  R1F15  \\ \hline  \hyperref[sec:UC1.12]{ UC1.12 Salvataggio audio } &  R0V1.8.1  \\ \hline  \hyperref[sec:UC1.13]{ UC1.13 Salvataggio video  } &  R2F1.7.3  \\ \hline  \hyperref[sec:UC1.2]{ UC1.2 Condivisione } &  R1Q1.6.3  \\ \hline  \hyperref[sec:UC1.2.1]{ UC1.2.1 Navigazione tra gli strumenti di condivisione } &  R1F1.4.1  \\ \hline  \hyperref[sec:UC1.2.2]{ UC1.2.2 Selezione strumento di condivisione } &  R1F1.4.3  \\ \hline  \hyperref[sec:UC1.2.3]{ UC1.2.3 Reperimento strumenti di condivisione } &  R1F1.4.2  \\ \hline  \hyperref[sec:UC1.2.4]{ UC1.2.4 Scelta se condividere audio o video } &  R1F15.1  \\ \hline  \hyperref[sec:UC1.2.4.1]{ UC1.2.4.1 Selezione battuta } &  R0F1.2.6.9  \\ \hline  \hyperref[sec:UC1.3]{ UC1.3 Esportazione audio finale } &  R0F1.6  \newline R1V1.6.1.1  \newline R0F1.6.1.2  \newline R1Q1.6.3  \\ \hline  \hyperref[sec:UC1.3.1]{ UC1.3.1 Selezione formato } &  R1F1.6.1  \newline R2F1.12.1  \\ \hline  \hyperref[sec:UC1.3.2]{ UC1.3.2 Inserimento nome \textit{file} } &  R0F1.6.2  \\ \hline  \hyperref[sec:UC1.4]{ UC1.4 Esportazione video finale } &  R1F1.12  \\ \hline  \hyperref[sec:UC1.5]{ UC1.5 Salvataggio \textit{file} } &  R0F1.7  \newline R0F1.7.1  \newline R1Q1.7.2  \\ \hline  \hyperref[sec:UC1.5.3]{ UC1.5.3 Visualizzazione messaggio di errore } &  R1Q1.7.2  \\ \hline  \hyperref[sec:UC1.7]{ UC1.7 Apertura \textit{file} } &  R0F1.9.1  \newline R0F19  \\ \hline  \hyperref[sec:UC1.7.0.1]{ UC1.7.0.1 Navigazione \textit{filesystem}\G } &  R0F19.1  \\ \hline  \hyperref[sec:UC1.7.1]{ UC1.7.1 Apertura immagine } &  R0F1.2.2.1  \newline R1F1.2.2.1.1  \\ \hline  \hyperref[sec:UC1.7.2]{ UC1.7.2 Apertura sceneggiato } &  R0F1.9  \\ \hline  \hyperref[sec:UC1.8]{ UC1.8 Modifica sceneggiato } &  R0F1.1.1  \newline R0F16  \\ \hline  \hyperref[sec:UC1.8.1]{ UC1.8.1 Modifica capitolo } &  R0F1.11.1  \\ \hline  \hyperref[sec:UC1.8.1.1]{ UC1.8.1.1 Selezione capitolo } &  R0F1.11.1.11  \\ \hline  \hyperref[sec:UC1.8.1.2]{ UC1.8.1.2 Anteprima capitolo } &  R1F1.11.3  \\ \hline  \hyperref[sec:UC1.8.1.3]{ UC1.8.1.3 Modifica sfondo } &  R1F1.11.1.4  \newline R1F1.11.1.8  \newline R2F1.11.1.9  \\ \hline  \hyperref[sec:UC1.8.1.4]{ UC1.8.1.4 Modifica titolo capitolo } &  R0F1.11.1.2  \\ \hline  \hyperref[sec:UC1.8.1.5]{ UC1.8.1.5 Rimozione suono } &  R0F1.11.7  \\ \hline  \hyperref[sec:UC1.8.2]{ UC1.8.2 Anteprima sceneggiato } &  R1F1.10  \newline R1F16.2  \\ \hline  \hyperref[sec:UC1.9]{ UC1.9 Salvataggio di emergenza } &  R1Q1.8  \newline R1Q1.8.2  \newline R  \newline R1F1.8.3  \\ \hline  \hyperref[sec:UC2.1]{ UC2.1 Creazione \textit{preset}\G } &  R0F2  \newline R0F3  \\ \hline  \hyperref[sec:UC2.1.1]{ UC2.1.1 Navigazione tra effetti disponibili } &  R0F2.1  \\ \hline  \hyperref[sec:UC2.1.2]{ UC2.1.2 Impostazione parametri } &  R0F2.3  \\ \hline  \hyperref[sec:UC2.1.3]{ UC2.1.3 Salvataggio \textit{preset}\G } &  R0F3.1  \\ \hline  \hyperref[sec:UC2.2]{ UC2.2 Selezione voce per il sistema } &  R2F4  \newline R0F4.1  \\ \hline  \hyperref[sec:UC2.2.1]{ UC2.2.1 Navigazione fra le voci disponibili } &  R0F4.1  \\ \hline  \hyperref[sec:UC2.3]{ UC2.3 Campionamento voce } &  R2F17  \newline R2F18  \newline R2F17.5  \newline R2F17.6  \\ \hline  \hyperref[sec:UC2.3.1]{ UC2.3.1 Visualizzazione frasi da leggere } &  R2F17.2  \\ \hline  \hyperref[sec:UC2.3.2]{ UC2.3.2 Registrazione lettura frasi } &  R2F17.3  \\ \hline  \hyperref[sec:UC2.3.3]{ UC2.3.3 Invio audio } &  R2F17.4  \\ \hline  \hyperref[sec:UC2.3.4]{ UC2.3.4 \textit{Login}\G\ \AZIENDA } &  R2F17.1  \\ \hline  \hyperref[sec:UC2.4]{ UC2.4 Test \textit{preset}\G } &  R0F2.4  \\ \hline  \hyperref[sec:UC2.5]{ UC2.5 Modifica \textit{preset}\G } &  R0F2.5  \\ \hline  \hyperref[sec:UC3.1]{ UC3.1 Reperimento voci disponibili } &  R0F5  \newline R0F6  \\ \hline  \hyperref[sec:UC3.2]{ UC3.2 Reperimento effetti disponibili } &  R0F2.2  \newline R0F9  \newline R0F10  \\ \hline  \hyperref[sec:UC3.3]{ UC3.3 Comunicazione server } &  R0F7  \newline R1Q7.4  \\ \hline  \hyperref[sec:UC3.3.1]{ UC3.3.1 Creazione richiesta HTTP\G } &  R0F7.1  \\ \hline  \hyperref[sec:UC3.3.3]{ UC3.3.3 Ricezione risposta HTTP\G } &  R0F7.2  \\ \hline  \hyperref[sec:UC3.3.4]{ UC3.3.4 Visualizzazione messaggio di errore } &  R1Q7.3  \newline R1Q7.4  \\ \hline  \hyperref[sec:UC3.4]{ UC3.4 Ricezione audio dal server } &  R1Q11  \\ \hline  \hyperref[sec:Verbale 18/03/2016]{ Verbale 18/03/2016  } &  R0V21  \\ \hline \caption{Tracciamento fonti-requisiti}
\end{longtable}
\egroup
\end{center}