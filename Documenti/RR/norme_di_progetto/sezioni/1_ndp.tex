\section{Introduzione}

\subsection{Scopo del documento}
Questo documento definisce le norme che i membri del gruppo \GRUPPO\ si impegnano a rispettare per un corretto svolgimento del progetto SiVoDiM: Sintesi Vocale per Dispositivi Mobili. Ogni componente del gruppo è tenuta a leggere tale documento e seguire le norme con lo scopo di raggiungere il miglior punto di incontro tra efficienza ed efficacia. In questo modo viene garantita l'uniformità del materiale prodotto e vengono facilitate le operazioni di verifica. In particolare, verranno specificate le seguenti norme:

\begin{itemize}
	\item	Interazioni tra i membri del gruppo;
	\item	Comunicazione verso l'esterno;
	\item	Stesura dei documenti e convenzioni tipografiche;
	\item	Organizzazione dell'ambiente di lavoro;
	\item	Modalità di lavoro durante le fasi del progetto;
	\item	Stesura del codice.
\end{itemize}

\subsection{Scopo del progetto}
Lo scopo del progetto risiede nello sviluppo di un'applicazione utile a dimostrare efficacemente
le potenzialità del motore di sintesi vocale FA-TTS\G, realizzato dall'azienda \AZIENDA\ e messo a disposizione del gruppo di lavoro. Si devono realizzare due applicazioni per sistemi Android\G:
\begin{itemize}
	\item \textbf{Applicazione di configurazione}: deve permettere all'utente di interfacciarsi direttamente con il sistema operativo per configurare, salvare e modificare le voci ereditate dal motore di sintesi FA-TTS di MIVOQ;
	\item \textbf{Applicazione per la creazione di sceneggiati}: permette la creazione e il salvataggio di racconti e sceneggiati, che possono essere esportati in formato audio attraverso l'utilizzo del motore FA-TTS.
\end{itemize}
Entrambe le applicazioni devono interfacciarsi con due moduli di basso livello:
\begin{itemize}
	\item \textbf{Modulo di sistema}: permette di interfacciarsi tramite connessione di rete al motore FA-TTS;
	\item \textbf{Libreria}: una libreria contenente tutte le funzionalità offerte dal motore FA-TTS, utile nell'ottica di un riuso futuro del \textit{software}.
\end{itemize} 
Lo sviluppo di tutte e quattro le suddette componenti è a carico del gruppo Stark Labs.

\subsection{Glossario}
Al fine di aumentare la comprensione del testo ed evitare eventuali ambiguità, viene fornito un glossario (\textit{Glossario v1.0.0}) contenente le definizioni degli acronimi e dei termini tecnici utilizzati nei documenti. Ogni vocabolo che ha un riferimento contenuto nel glossario è contrassegnato dal pedice “\G “.

\subsection{Riferimenti}

\subsubsection{Informativi}
\begin{itemize}
	\item \textit{Glossario v1.0.0};
	\item Capitolato C1 – Actorbase: a NoSQL DB based on the Actor model\\
	\url{http://www.math.unipd.it/~tullio/IS-1/2015/Progetto/C1.pdf};
	\item Capitolato C2 – CLIPS: Communication \& Localization with Indoor Positioning Systems\\
	\url{http://www.math.unipd.it/~tullio/IS-1/2015/Progetto/C2.pdf};
	\item Capitolato C3 – UMAP: un motore per l’analisi predittiva in ambiente Internet of Things\\
	\url{http://www.math.unipd.it/~tullio/IS-1/2015/Progetto/C3.pdf};
	\item Capitolato C4 – MaaS: MongoDB as an admin Service\\
	\url{http://www.math.unipd.it/~tullio/IS-1/2015/Progetto/C4.pdf};
	\item Capitolato C5 – Quizzipedia: software per la gestione di questionari\\
	\url{http://www.math.unipd.it/~tullio/IS-1/2015/Progetto/C5.pdf};
	\item Capitolato C6 – SiVoDiM: Sintesi Vocale per Dispositivi Mobili\\
	\url{http://www.math.unipd.it/~tullio/IS-1/2015/Progetto/C6.pdf};
	\item AS/NZS ISO/IEC 12207:1997 \\
	\url{http://www.math.unipd.it/~tullio/IS-1/2009/Approfondimenti/ISO\_12207-1995.pdf};
	\item SI/ISO 31-0
	\url{https://en.wikipedia.org/wiki/ISO_31-0};
	\item Guida all'utilizzo di Teamwork\G\\
	\url{http://support.teamwork.com/projects/start/getting-started};
	\item Guida all'utilizzo di GitHub\G\\
	\url{https://guides.github.com};
	\item Guida all'utilizzo di Astah\G\\
	\url{http://astah.net/tutorials};
	\item Guida all'utilizzo di Microsft Project 2016\G \\
	\url{https://youtu.be/_eD2u8bxecs} \\
   \def\UrlBreaks{\do\/\do-}
    \url{https://support.office.com/it-it/article/Formattazione-del-diagramma-di-una-visualizzazione-di-Gantt-7473acdc-4abe-4b2f-8361-546efa9dce06#top}.
\end{itemize}
\newpage