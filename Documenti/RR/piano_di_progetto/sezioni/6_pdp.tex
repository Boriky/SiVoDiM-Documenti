\section{Consuntivo a finire}

Di seguito viene riportato un resoconto delle ore effettivamente impiegate dai 
vari componenti in ogni fase.
In base alla differenza tra le ore pianificate e quelle reali, si ottiene un bilancio:
\begin{itemize}
	\item \textbf{Positivo}: se le ore pianificate sono inferiori a quelle 
	realmente utilizzate;
	\item \textbf{Negativo}: se le ore pianificate sono superiori a quelle 
	realmente utilizzate;
	\item \textbf{In pari}: se le ore pianificate coincidono con quelle 
	realmente utilizzate;	
\end{itemize}

\subsection{Analisi dei Requisiti}
La tabella sottostante mostra la differenza tra le ore pianificate e le ore effettivamente 
impiegate. Inoltre viene indicata la differenza di costo che 
queste variazioni comportano rispetto a quanto preventivato.
Un costo positivo indica un passivo di bilancio. Un costo negativo indica un 
risparmio sul costo preventivato. Un costo pari a 0 (zero) rispetta quanto 
preventivato.
\begin{center}
	\def\arraystretch{1.6}
	\bgroup
	\begin{longtable}{| p{4cm} | c | c | c | c |}
		\hline
		\textbf{Ruolo} & \textbf{Ore pianificate} & \textbf{Ore effettive} & 
		\textbf{Differenza} & \textbf{Costo (€)}\\ 
		
		\hline \hline  
		
		\textbf{Project Manager} & {19} & {20} & {+1} & {+30} \\ 
		\hline 
		
		\textbf{Amministratore} & {24} & {27} & {+3} & {+60} \\ 
		\hline 
		
		\textbf{Analista} & {48} & {45} & {-3} & {-75} \\ 
		\hline 
		
		\textbf{Progettista} & {} & {} & {} & {} \\ 
		\hline 
		
		\textbf{Programmatore} & {} & {} & {} & {} \\ 
		\hline 
		
		\textbf{Verificatore} & {40} & {40} & {0} & {0} \\ 
		\hline 
		
		\textbf{Totale} & \textbf{131} & \textbf{132} & \textbf{+1} & 
		\textbf{+15} \\ 
		\hline 
		
		
		\hline 
		
		\caption{Consuntivo a finire - Analisi dei Requisiti}
	\end{longtable}
	\egroup
\end{center}


\subsubsection{Conclusione}
La percentuale di variazione tra le ore preventivate nella fase di 
progettazione e le ore effettive di lavoro è di circa 5,34\%.
Per lo svolgimento della fase di analisi il \textit{team} \GRUPPO\ ha impiegato 
un'ora in più rispetto a quelle preventivate, comportando un aumento di 15€ sul 
preventivo totale. Questo costo non è imputato al Committente.
