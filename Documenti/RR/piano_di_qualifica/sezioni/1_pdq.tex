\section{Introduzione}

\subsection{Scopo del documento}
Questo documento contiene le strategie che il gruppo di lavoro 
ha deciso di adottare per raggiungere gli obiettivi qualitativi nel progetto \PROGETTO. 
All'interno di tale ottica, è necessario un continuo processo di verifica, 
con lo scopo di correggere tempestivamente e senza spreco di risorse le anomalie riscontrate sulle attività svolte.


\subsection{Scopo del progetto}
Lo scopo del progetto risiede nello sviluppo di un'applicazione utile a dimostrare efficacemente
le potenzialità del motore di sintesi vocale FA-TTS\G\ realizzato dall'azienda MIVOQ s.r.l. e messo a disposizione del gruppo di lavoro. Si devono realizzare due applicazioni per sistemi Android\G:
\begin{itemize}
	\item \textbf{Applicazione di configurazione}: deve permettere all'utente di interfacciarsi direttamente con il sistema operativo per configurare, salvare e modificare le voci ereditate dal motore di sintesi FA-TTS di MIVOQ;
	\item \textbf{Applicazione per la creazione di sceneggiati}: permette la creazione e il salvataggio di racconti e sceneggiati, che possono essere esportati in formato audio attraverso l'utilizzo del motore FA-TTS.
\end{itemize}
Entrambe le applicazioni devono interfacciarsi con due moduli di basso livello:
\begin{itemize}
	\item \textbf{Modulo di sistema}: permette di interfacciarsi tramite connessione di rete al motore FA-TTS;
	\item \textbf{Libreria}: una libreria contenente tutte le funzionalità offerte dal motore FA-TTS, utile nell'ottica di un riuso futuro del \textit{software}.
\end{itemize} 
Lo sviluppo di tutte e quattro le suddette componenti è a carico del gruppo Stark Labs.

\subsection{Glossario}
Al fine di aumentare la comprensione del testo ed evitare eventuali ambiguità, 
viene fornito un glossario (\textit{Glossario v1.0.0}) contenente le 
definizioni degli acronimi e dei termini tecnici utilizzati nel documento. Ogni 
vocabolo contenuto nel glossario è contrassegnato dal pedice “\G “.

\subsection{Riferimenti}

\subsubsection{Normativi}
\begin{itemize}
\item \textit{Norme di Progetto v1.0.0};
\item \textbf{Capitolato C6} – SiVoDiM: Sintesi Vocale per Dispositivi Mobili.
\end{itemize}

\subsubsection{Informativi}
\begin{itemize}
\item \textit{Glossario v1.0.0};
\item \textit{Piano di progetto v1.0.0};
\item \textbf{Slide dell’insegnamento Ingegneria del Software modulo A}:
\begin{itemize}
\item Ingegneria dei requisiti;
\item Diagrammi dei casi d'uso\\
\url{http://www.math.unipd.it/~tullio/IS-1/2015/}.
\end{itemize}
\item Software Engineering - Ian Sommerville - 9th Edition 2010:
\begin{itemize}
\item Chapter 24: Quality management;
\item Chapter 25: Configuration management;
\item Chapter 26: Process improvement.
\end{itemize} 
\item SWEBOK - Version 3 (2004): capitolo 11 - Software Quality;
\item IEEE 830-1998: \url{ 
https://en.wikipedia.org/wiki/Software_requirements_specification};
\item Indice Gulpease: \url{http://it.wikipedia.org/wiki/Indice_Gulpease}.
\end{itemize}
\newpage
