\section{Visione generale della strategia di verifica}
\subsection{Definizione obiettivi}
Di seguito vengono descritti gli obiettivi di qualità relativi al prodotto commissionato e ai processi necessari per un sua corretta realizzazione.
\subsubsection{Qualità di processo}
Per garantire la qualità del prodotto si deve perseguire la qualità dei processi che lo definiscono. Per tale ragione si è deciso di utilizzare lo standard ISO\G/IEC\G\ 15504 noto come SPICE, che fornisce gli strumenti necessari per valutare l'idoneità dei processi. Per applicare correttamente il suddetto modello si deve far uso del ciclo di \textit{Deming}\G, che definisce una metodologia di controllo specifica per i processi nel corso del loro ciclo di vita, con l'intento di migliorarne costantemente la qualità.

\subsubsection{Qualità di prodotto}
Per aumentare il valore commerciale di un prodotto \textit{software}, e per garantire il 
corretto funzionamento dello stesso, è necessario fissare degli obiettivi 
qualitativi e verificare che questi vengano rispettati. Lo standard ISO\G/IEC\G\ 9126 è stato redatto allo scopo di definire questi 
obiettivi e delineare alcune metriche capaci di misurare il raggiungimento 
degli stessi.

\subsection{Procedure di controllo di qualità di processo}
La qualità dei processi è garantita dalla corretta applicazione del metodo PDCA\G. 
Grazie a questo strumento si può garantire un continuo miglioramento della 
qualità dei processi, inclusa la verifica, con un conseguente 
miglioramento dei prodotti creati. 
Per avere il controllo dei processi e della qualità è necessario che:
\begin{itemize}
	\item	I processi siano pianificati dettagliatamente;
	\item	Vengano ripartite chiaramente le risorse nella pianificazione;
	\item	Ci sia controllo sui processi.
\end{itemize}
L'attuazione di tali punti è descritta dettagliatamente nel \textit{Piano di 
Progetto v1.0.0}. La qualità dei processi viene controllata tramite l'analisi costante della qualità del prodotto.  Un prodotto di bassa qualità è indice di un processo che deve essere migliorato. Per quantificare la qualità dei processi vengono utilizzate le metriche 
descritte nella \hyperref[cap:sezione 2.9 Misure e metriche]{sezione 2.9 Misure e metriche}.

\subsection{Procedure di controllo della qualità del prodotto}
Si garantisce il controllo della qualità del prodotto attraverso il rispetto dei seguenti punti:
\begin{itemize}
	\item \textbf{Quality assurance}: è l'insieme di attività svolte al 
	fine di garantire il raggiungimento degli obiettivi di qualità. Prevede 
	l'attuazione di tecniche di analisi statica e dinamica, descritte nella \hyperref[cap:sezione 2.8.1 Analisi statica]{sezione 2.8.1 Analisi statica};
	\item \textbf{Verifica}: processo che determina se il risultato di una fase 
	è corretto. La verifica viene eseguita costantemente durante 
	l'intera durata del progetto. I risultati delle attività di verifica 
	eseguiti nelle varie fasi di progetto sono riportate nella sezione \hyperref[cap:Sezione 5 Resoconto delle attività di verifica]{5 Resoconto delle attività di verifica};
	\item \textbf{Validazione}: la conferma oggettiva che il sistema risponda ai requisiti.
\end{itemize}

\subsection{Organizzazione}
L'organizzazione della strategia di verifica è basata sull'utilizzo di attività 
di controllo per ogni processo attuato. Per ognuno di questi viene verificata 
la qualità ed eventualmente si verifica anche la qualità del prodotto ottenuto.
Ognuna delle fasi del progetto descritte nel \textit{Piano di Progetto v1.0.0} necessita di diverse attività di verifica dipendenti da differenti \textit{output}:
\begin{itemize}
	\item \textbf{Analisi:} in questa fase è necessario seguire i metodi di 
	verifica descritti nella sezione \hyperref[cap:sezione 2.8 Tecniche di analisi]{2.8 Tecniche di analisi} sui 
	documenti prodotti. La realizzazione di tali attività di verifica sono 
	descritte nella sezione \hyperref[cap: sezione 4.1 Standard ISO/IEC 15504]{4.1 Standard ISO/IEC 15504}.
\end{itemize}
In ogni documento viene inoltre incluso il registro delle modifiche che permette di mantenere uno storico delle attività svolte e delle relative responsabilità.

\subsection{Pianificazione strategica e temporale}
Dato che l'obiettivo è di rispettare le scadenze fissate nel \textit{Piano di 
Progetto v1.0.0}, è necessario che l'attività di verifica sia ben organizzata. 
Pertanto l'individuazione e la correzione di errori dovrà essere tempestiva, in 
modo da impedire che si diffondano. Ogni attività di redazione di documenti o di codifica deve essere preceduta da un'analisi della struttura e dei contenuti. Questo allo scopo di evitare imprecisioni concettuali o tecniche e rendendo l'attività di verifica più semplice, con conseguente riduzione del numero di correzioni richieste.
La metodologia da seguire per individuare e correggere eventuali errori è descritta nelle \textit{Norme di Progetto v1.0.0}.

\subsection{Responsabilità}
Per garantire che il processo di verifica sia efficace e sistematico, vengono 
attribuite tali responsabilità a \textit{Verificatore} e \textit{Responsabile di Progetto}. La 
suddivisione dei compiti e le modalità sono definite nelle \textit{Norme di 
Progetto v1.0.0}.

\subsection{Risorse}
Per assicurarsi che gli obiettivi vengano raggiunti sono necessarie risorse umane e tecnologiche. Coloro che detengono la maggiore responsabilità per le attività di verifica e validazione sono \textit{Verificatore} e \textit{Responsabile di 
Progetto}. I ruoli sono descritti nel dettaglio nelle \textit{Norme di 
Progetto v1.0.0}. Per risorse tecniche e tecnologiche sono intesi tutti gli strumenti 
\textit{software} e \textit{hardware} che il gruppo intende utilizzare. 
Affinché il lavoro dei \textit{Verificatori} venga semplificato, sono stati impostati 
alcuni strumenti di controllo sistematico. Questi sono descritti in modo 
accurato nelle \textit{Norme di Progetto v1.0.0}.

\subsection{Tecniche di analisi}
\label{cap:sezione 2.8 Tecniche di analisi}

\subsubsection{Analisi statica}
\label{cap:sezione 2.8.1 Analisi statica}
Per analisi statica si intende una tecnica di controllo che permette di effettuare la verifica di quanto prodotto individuando eventuali errori. Essa viene svolta in due modi complementari.

\paragraph{Walkthrough}
\label{cap:sezione 2.8.2 Walkthrough}
Viene svolta una lettura critica di tutto il materiale. Questa tecnica è utile 
nelle prime fasi di progetto, quando i membri del gruppo non hanno ancora un'adeguata esperienza che permette verifiche più mirate.
Grazie a questa tecnica, il \textit{Verificatore} può stilare una lista degli errori più frequenti, in modo da migliorare le attività di analisi future.
Questa attività è onerosa e richiede l'intervento di più persone per essere efficace ed efficiente. In seguito alla lettura segue una fase di discussione con il fine di esaminare i difetti e proporre le correzioni. La fase finale consiste nello stilare un rapporto che elenchi le modifiche effettuate.

\paragraph{Inspection}
In questa tecnica viene eseguita un'analisi mirata delle parti del documento o 
del codice che sono ritenute maggiormente fonte di errore. La lista di 
controllo, che contiene queste sezioni critiche, è redatta anticipatamente ed 
è frutto dell'esperienza dei \textit{Verificatori} in seguito all'applicazione del Walkthrough.
Questa strategia è più rapida del Walkthrough in quanto riduce il numero di 
parti da analizzare. Per tale ragione l'Inspection può essere eseguita 
solamente dai \textit{Verificatori}, che individuano e correggono eventuali errori e 
redigono il rapporto di verifica per tracciare il lavoro svolto.

\subsubsection{Analisi dinamica}
L'analisi dinamica viene applicata solamente alla produzione di codice e viene effettuata durante l'esecuzione mediante l'uso di test utilizzati per verificarne il funzionamento. Per rendere questa attività utile e generare risultati attendibili, è 
necessario che i test siano ripetibili. Questo significa che il programma da 
un determinato \textit{input} generi sempre lo stesso \textit{output}. Test di questo tipo sono 
utili per determinare la correttezza ed evidenziare eventuali problemi in un 
\textit{software}. Devono quindi essere definiti:
\begin{itemize}
	\item \textbf{Ambiente:} consiste sia del sistema \textit{hardware} che di quello \textit{software} sui quali è pianificato lo sviluppo. Di questi è necessario specificare lo stato iniziale dal quale iniziare i test;
	\item \textbf{Specifica:} consiste nel definire gli \textit{input} e i rispettivi \textit{output} attesi;
	\item \textbf{Procedure:} consiste nella definizione di come i test vengono svolti, con quale ordine e come vengono analizzati i risultati.
\end{itemize}
Esistono cinque tipi diversi di test: test di unità, test di integrazione, test di sistema, test di regressione e test di accettazione.

\paragraph{Test di unità}

Consta nel verificare ogni singola unità del \textit{software} con l'utilizzo di 
\textit{stub}\G, \textit{driver}\G\ e \textit{logger}\G.
Con unità è inteso il minimo quantitativo di \textit{software}, che sia utile verificare 
singolarmente, prodotto da un singolo programmatore. Con questi test si 
verifica il funzionamento corretto dei moduli per eliminare dal sistema 
possibili errori di implementazione.

\paragraph{Test di integrazione}

Consta nel verificare i componenti del sistema che vengono aggiunti, al fine di 
analizzare che la combinazione di due o più unità \textit{software} funzionino come 
previsto.
L'obiettivo di questo test è di individuare errori residui dalla 
realizzazione dei moduli o comportamenti inaspettati da componenti \textit{software} 
forniti da terze parti. Per effettuare questi test è necessario aggiungere 
delle componenti fittizie per sostituire quelle ancora non sviluppate, al fine 
di non influenzare l'esito dell'analisi.

\paragraph{Test di sistema}

Consta nel validare il prodotto \textit{software} nel momento in cui si ritiene che sia 
giunto ad una versione definitiva. Questo test verifica che tutti i requisiti 
software stabiliti nell'\textit{Analisi dei Requisiti v1.0.0} vengano 
rispettati.

\paragraph{Test di regressione}

Consta nell'eseguire nuovamente i test sulle componenti \textit{software} in seguito a nuove modifiche, al fine di controllare che i cambiamenti non alterino il 
corretto funzionamento di queste componenti o di altre che non sono state 
aggiornate.
Questa operazione viene facilitata dal tracciamento, che permette di individuare e ripetere i test di unità, integrazione e di sistema che sono stati influenzati dalla modifica.

\paragraph{Test di accettazione}

Consta nel collaudare il prodotto \textit{software} in presenza del Proponente. Se questo collaudo viene superato, si può procedere al rilascio ufficiale del prodotto sviluppato.

\subsection{Misure e metriche}
\label{cap:sezione 2.9 Misure e metriche}

Il processo di verifica deve essere quantificabile per risultare informativo. Pertanto, è necessario stabilire a priori delle metriche su cui basare le misurazioni del processo di verifica. Essendo le metriche di natura variabile, vengono definite due tipologie di intervalli:
\begin{itemize}
	\item \textbf{Accettazione:} valori che vengono richiesti affichè il prodotto sia accettato;
	\item \textbf{Ottimale:} valori entro cui è consigliabile che la 
	misurazione si collochi.
\end{itemize}
Non sono intervalli vincolanti, ma consigliati. Se 
tali valori si discostano, è necessaria una verifica approfondita.

\subsubsection{Metriche per i processi}
\label{cap: sezione 2.8.1 Metriche per i processi}
Come metriche per valutare i processi si è scelto di utilizzare degli indici che analizzino i costi e i tempi.

\paragraph{Schedule Variance (SV)}

Valuta se si è in linea, in anticipo o in ritardo rispetto alla pianificazione 
della \textit{baseline}\G. Questo è un indice di efficacia. Se SV>0 significa 
che il team sta producendo con maggior velocità 
rispetto alla pianificazione, viceversa se è negativo.\\\\
\textbf{Parametri utilizzati:}
\begin{itemize}
	\item Range di accettazione: [> -(Costo preventivo fase x 5\%)];
	\item Range ottimale: [>0].
\end{itemize}

\paragraph{Budget Variance (BV)}
Valuta nella data corrente la differenza tra la spesa attuale e il costo 
pianificato. Se BV>0 significa che il progetto sta consumando il budget più 
lentamente di 
quanto pianificato, viceversa se negativo.\\\\
\textbf{Parametri utilizzati:}

\begin{itemize}
	\item Range di accettazione: [> -(Costo preventivo fase x 10\%)];
	\item Range ottimale: [>0].
\end{itemize}

\subsubsection{Metriche per i documenti}
\label{cap:sezione 2.9.2 Metriche per i documenti}
Come metrica per i documenti si è scelto di utilizzare un indice di 
leggibilità. L'indice utilizzato è specifico per la lingua italiana.

\paragraph{Gulpease\G}
L'indice Gulpease\G\ è un indice di leggibilità tarato sulla lingua italiana. 
Viene preferito rispetto ad altri poiché utilizza la lunghezza in lettere 
anziché in sillabe, semplificandone il calcolo. Questo indice evidenzia la 
complessità dello stile del documento.\newline
L'indice è calcolato secondo la seguente formula:
\begin{center}
$ 89+ \dfrac{300*(numero\ frasi)-10*(numero\ lettere)}{numero\ parole} $
\end{center}
I risultati sono compresi tra 0 e 100, con 100 la migliore leggibilità e 0 la peggiore. \newline \newline \textbf{Parametri utilizzati:}
\begin{itemize}
	\item Range di accettazione: [40-100];
	\item Range ottimale: [50-100].
\end{itemize}

\subsubsection{Metriche per il software}
È desiderabile, per poter raggiungere gli obiettivi di qualità del \textit{software}, 
applicare le metriche a seguire.

\paragraph{Numero di livelli di annidamento}

Tale indice rappresenta il numero di livelli di annidamento dei metodi, ossia l'annidamento delle strutture. Un alto livello di questo indice può essere sintomo di un elevata complessità del codice o di un basso livello di astrazione.\newline \newline
\textbf{Parametri utilizzati:}
\begin{itemize}
	\item Range di accettazione: [1-6];
	\item Range ottimale: [1-3].
\end{itemize}

\paragraph{Numero di attributi per classe}

Tale indice rappresenta il numero di attributi contenuti in una classe. Un valore elevato può indicare la necessità di suddividere la classe in più classi. Pertanto, tale valore potrebbe essere sintomo di errori progettuali. \newline \newline
\textbf{Parametri utilizzati:}
\begin{itemize}
	\item Range di accettazione: [0-16];
	\item Range ottimale: [3-8].
\end{itemize}

\paragraph{Numero di parametri per metodo}

Tale indice rappresenta il numero di parametri contenuti nei metodi. Un valore elevato potrebbe indicare un metodo con funzionalità eccessivamente complesse, sintomo di errori progettuali.\newline \newline
\textbf{Parametri utilizzati:}
\begin{itemize}
	\item Range di accettazione: [0-8];
	\item Range ottimale: [0-4].
\end{itemize}

\paragraph{Linee di codice per linee di commento}

Tale indice viene calcolato come il rapporto tra le linee di commento e le linee di codice. Valori ottimali di questo parametro indicano un codice più manutenibile. \newline \newline
\textbf{Parametri utilizzati:}
\begin{itemize}
	\item Range di accettazione: [>0.25];
	\item Range ottimale: [>0.30].
\end{itemize}

\paragraph{Copertura di codice}

Tale indice indica la percentuale di istruzioni eseguite durante i test. Maggiore è questo parametro, minore sarà la probabilità di errori nel codice. Questo valore può essere abbassato dalla presenza di metodi semplici, come \textit{setter} o \textit{getter}. \newline \newline
\textbf{Parametri utilizzati:}
\begin{itemize}
	\item Range di accettazione: [42\%-100\%];
	\item Range ottimale: [65\%-100\%].
\end{itemize}
\newpage
