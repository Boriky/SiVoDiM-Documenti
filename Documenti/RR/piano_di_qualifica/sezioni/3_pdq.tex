\section{Gestione amministrativa della revisione}

\subsection{Comunicazione e risoluzione di anomalie}
\label{cap: sezione 3.1 Comunicazione e risoluzione di anomalie}

Un'anomalia viene identificata da almeno uno dei seguenti casi:

\begin{itemize}
	\item Violazione delle norme tipografiche di un documento;
	\item Violazione dei range di accettazione da parte degli indici di misurazione, descritti nella sezione \hyperref[cap:sezione 2.9 Misure e metriche]{2.9 Misure e metriche};
	\item Incongruenze tra il codice e il \textit{design} del prodotto;
	\item Incongruenze delle funzionalità del prodotto con funzionalità 
	indicate nel documento \textit{Analisi dei Requisiti v1.0.0}.
\end{itemize}
Se un \textit{Verificatore} individua un'anomalia, dovrà aprire un nuovo \textit{task} 
nell'apposita sezione Task di \textit{Teamwork}\G, come descritto nelle 
\textit{Norme di Progetto v1.0.0}.

\newpage