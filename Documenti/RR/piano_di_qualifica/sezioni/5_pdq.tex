\section{Resoconto delle attività di verifica}
\label{cap:Sezione 5 Resoconto delle attività di verifica}

\subsection{Riassunto delle attività di verifica}

\subsubsection{Revisione dei Requisiti}

Nel periodo antecedente alla consegna della documentazione sono state eseguite le attività di verifica dei processi e dei documenti redatti. Ogni documento è stato controllato tramite la tecnica del \textit{walkthrough} descritta nella sezione \hyperref[cap:sezione 2.8.2 Walkthrough]{2.8.2 Walkthrough}. Si sono riscontrati diversi errori che sono stati segnalati in modo informale ai redattori. Il gruppo è consapevole di non aver seguito la procedura descritta nella sezione \hyperref[cap: sezione 3.1 Comunicazione e risoluzione di anomalie]{3.1 Comunicazione e risoluzione di anomalie}. Pertanto si impegna a seguirla dalla prossima fase: Analisi di Dettaglio. Si precisa inoltre che nel corso di questa attività il gruppo non si è occupato di stilare la lista di controllo necessaria per la tecnica dell'\textit{inspection}. Infine, dopo la correzione dei documenti, si è provveduto a calcolare le metriche descritte nella sezione \hyperref[cap:sezione 2.9 Misure e metriche]{2.9 Misure e metriche}. \\
Il tracciamento dei requisiti è avvenuto grazie all'utilizzo di un \textit{software} realizzato appositamente dagli \textit{Amministratori} del gruppo. \\
I processi sono stati verificati utilizzando le metriche di processo descritte nella sezione \hyperref[cap: sezione 2.8.1 Metriche per i processi]{2.8.1 Metriche per i processi}. Sono quindi stati riportati i valori di BV e SV relativi a tutti i processi di ciascuna fase.

\subsection{Dettaglio delle verifiche tramite analisi}

\subsubsection{Documenti}
Riportiamo in questa sezione i valori degli indici Gulpease calcolati per ogni documento. Un documento è valido se e solo se rispetta le metriche descritte nella sezione \hyperref[cap:sezione 2.9.2 Metriche per i documenti]{2.9.2 Metriche per i documenti}.

\begin{table}[ht]
\centering
\begin{tabular}{|c|c|c|}
	\hline
	\textbf{Documento} & \textbf{Valore} & \textbf{Indice} \\ \hline
	\textit{Piano di Qualifica v1.0.0} & 54 & Superato \\ \hline
	\textit{Piano di Progetto v1.0.0} & 58 & Superato \\ \hline
	\textit{Norme di Progetto v1.0.0} & 56 & Superato \\ \hline
	\textit{Analisi dei Requisiti v1.0.0} & 60 & Superato \\ \hline
	\textit{Studio di Fattibilità v1.0.0} & 54 & Superato \\ \hline
	\textit{Glossario v1.0.0} & 50 & Superato \\ \hline
\end{tabular}
\caption{Indici Gulpease dei documenti}
\end{table}
La tabella mostra come i documenti rientrino tutti nel \textit{range} ottimale precedentemente definito. Rispettano pertanto il livello di leggibilità minimo desiderato.
\newpage

\subsubsection{Processi}

Qui vengono riportati i valori di SV e BV descritti nella sezione \hyperref[cap: 
sezione 2.8.1 Metriche per i processi]{2.8.1 Metriche per i processi}, 
per le attività nella fase di \textbf{Analisi dei Requisiti} che hanno portato alla stesura dei documenti.

\begin{table}[ht]
	\centering
	\begin{tabular}{|c|c|c|}
		\hline
		\textbf{Attività} & \textbf{SV} & \textbf{BV} \\ \hline
		
		\textit{Piano di Qualifica v1.0.0} & -30 € & -30 € \\ \hline
		\textit{Piano di Progetto v1.0.0} & 0 € & -30 € \\ \hline
		\textit{Norme di Progetto v1.0.0} & 0 € & -30 € \\ \hline
		\textit{Analisi dei Requisiti v1.0.0} & +50 € & +25 € \\ \hline
		\textit{Studio di Fattibilità v1.0.0} & +30 € & 0 € \\ \hline
		\textit{Glossario v1.0.0} & 0 € & 0 € \\ \hline
	\end{tabular}
	\caption{Metriche dei processi}
\end{table}

Complessivamente la fase di \textbf{Analisi} ha:

\begin{itemize}
	\item SV : +50€;
	\item BV : -65€.
\end{itemize}

Da questi indici si può notare che:

\begin{itemize}
	\item Grazie a una pianificazione accurata delle scadenze, le attività 
	sono state svolte nei tempi pianificati, alcune volte terminando in 
	anticipo rispetto alle date di fine prefissate;
	\item Il costo complessivo si è di poco discostato da quanto pianificato nel 
	\textit{Piano di Progetto v1.0.0}. La causa di questa variazione dei costi è imputabile alla poco esperienza del gruppo nel redigere tale tipologia di documenti.
\end{itemize}

\subsubsection{Esito delle revisioni}

Durante lo sviluppo del progetto vi saranno quattro revisioni da parte del Committente. Il Committente segnalerà problematiche che verranno riscontrate valutando globalmente l'andamento del progetto e di ogni singolo documento. Grazie a queste informazioni sarà possibile correggere gli errori commessi, al fine di procedere lungo delle attività lavorative che risultino corrette e verificate.