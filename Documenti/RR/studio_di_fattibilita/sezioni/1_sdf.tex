\section{Introduzione}

\subsection{Scopo del documento}
Questo documento ha lo scopo di chiarire le motivazioni che hanno guidato il gruppo verso la scelta del capitolato C6 (SiVoDiM: Sintesi Vocale per Dispositivi Mobili). Inoltre, verranno indicate le analisi e le considerazioni che hanno portato ad escludere i capitolati rimanenti.

\subsection{Capitolato scelto}
È stato scelto di realizzare il capitolato C6, proposto dall'azienda MIVOQ s.r.l. e commissionato dal professore Tullio Vardanega.

\subsubsection{Scopo del progetto}
Lo scopo del progetto risiede nello sviluppo di un'applicazione utile a dimostrare efficacemente
le potenzialità del motore di sintesi vocale FA-TTS\G, realizzato dall'azienda \AZIENDA\ e messo a disposizione del gruppo di lavoro. Si devono realizzare due applicazioni per sistemi Android\G:
\begin{itemize}
	\item \textbf{Applicazione di configurazione}: deve permettere all'utente di interfacciarsi direttamente con il sistema operativo per configurare, salvare e modificare le voci ereditate dal motore di sintesi FA-TTS di MIVOQ;
	\item \textbf{Applicazione per la creazione di sceneggiati}: permette la creazione e il salvataggio di racconti e sceneggiati, che possono essere esportati in formato audio attraverso l'utilizzo del motore FA-TTS.
\end{itemize}
Entrambe le applicazioni devono interfacciarsi con due moduli di basso livello:
\begin{itemize}
	\item \textbf{Modulo di sistema}: permette di interfacciarsi tramite connessione di rete al motore FA-TTS;
	\item \textbf{Libreria}: una libreria contenente tutte le funzionalità offerte dal motore FA-TTS, utile nell'ottica di un riuso futuro del \textit{software}.
\end{itemize} 
Lo sviluppo di tutte e quattro le suddette componenti è a carico del gruppo Stark Labs.

\subsection{Glossario}
Al fine di aumentare la comprensione del testo ed evitare eventuali ambiguità, 
viene fornito un glossario (\textit{Glossario v1.0.0}) contenente le 
definizioni degli acronimi e dei termini tecnici utilizzati nel documento. Ogni 
vocabolo contenuto nel glossario è contrassegnato dal pedice “\G “.

\subsection{Riferimenti}

\subsubsection{Normativi}
\begin{itemize}
\item \textit{Norme di Progetto v1.0}.
\end{itemize}

\subsubsection{Informativi}
\begin{itemize}
\item \textit{Glossario v1.0.0};
\item Capitolato C1 – Actorbase: a NoSQL DB based on the Actor model\\
\url{http://www.math.unipd.it/~tullio/IS-1/2015/Progetto/C1.pdf};
\item Capitolato C2 – CLIPS: Communication \& Localization with Indoor 
Positioning Systems \\ 
\url{http://www.math.unipd.it/~tullio/IS-1/2015/Progetto/C2.pdf};
\item Capitolato C3 – UMAP: un motore per l'analisi predittiva in ambiente Internet of Things\\ \url{http://www.math.unipd.it/~tullio/IS-1/2015/Progetto/C3.pdf};
\item Capitolato C4 – MaaS: MongoDB as an admin Service\\ 
\url{http://www.math.unipd.it/~tullio/IS-1/2015/Progetto/C4.pdf};
\item Capitolato C5 – Quizzipedia: software per la gestione di questionari\\
\url{http://www.math.unipd.it/~tullio/IS-1/2015/Progetto/C5.pdf};
\item Capitolato C6 – SiVoDiM: Sintesi Vocale per Dispositivi Mobili\\ 
\url{http://www.math.unipd.it/~tullio/IS-1/2015/Progetto/C6.pdf}.

\end{itemize}

\newpage


