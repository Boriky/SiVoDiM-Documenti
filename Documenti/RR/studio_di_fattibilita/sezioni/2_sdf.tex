\section{Capitolato C6 - SiVoDiM}
\subsection{Descrizione}
Si è deciso di sviluppare un'applicazione \textit{mobile} nell'ambito del \textit{Text-to-Speech}\G, seguendo le direttive date dal capitolato C6 proposto da MIVOQ s.r.l.. Il lavoro del gruppo deve soddisfare i seguenti requisiti obbligatori:
\begin{itemize}
\item Realizzazione di un'applicazione che utilizzi il motore di sintesi 
\textit{open-source}\G\ Flexible and Adaptive Text-To-Speech\G\, su almeno una 
piattaforma mobile a scelta fra Android\G, iOS\G\ e Windows Phone\G;
\item Implementazione di meccanismi atti a risolvere le problematiche legate all’utilizzo del motore di sintesi, disponibile come servizio remoto;
\item Implementazione di un'interfaccia di configurazione dei servizi TTS\G\, offerti dal motore di sintesi;
\item Documentazione dell'applicazione che includa: descrizione del caso d'uso, analisi dei requisiti e descrizione tecnica.
\end{itemize}

\subsection{Studio del dominio}
Il dominio del capitolato riguarda la sintesi vocale, ossia la tecnologia che 
permette di convertire un qualsiasi file di testo in un file sonoro. Questa 
tecnologia viene utilizzata da lungo tempo come strumento di assistenza per 
utenti con difficoltà di lettura e di apprendimento; un esempio tipico è lo 
\textit{screen reader\G}, che identifica ed interpreta il testo mostrato sullo 
schermo di un computer, presentandolo tipicamente sotto forma di audio tramite 
sintesi vocale. Tra le altre applicazioni in cui viene utilizzata la tecnologia 
TTS\G\, troviamo i sistemi di navigazione GPS\G, la comunicazione di 
informazioni in aeroporti o stazioni ferroviarie, e la creazione di dialoghi in 
prodotti videoludici\G. Nonostante il notevole sviluppo e utilizzo di tali 
motori di sintesi, ad oggi non sono ancora state realizzate applicazioni in 
grado di utilizzare in modo efficace effetti sulle voci sintetiche, 
problematica chiave che il motore Flexible and Adaptive Text-To-Speech\G\ è in 
grado di affrontare. Dovrà essere preso in considerazione il mondo \textit{mobile} e dei sistemi operativi di riferimento: Android\G, iOS\G\ e Windows Phone\G. La creazione di un'applicazione multipiattaforma\G\ non è un requisito obbligatorio del progetto e, pertanto, si è scelto Android come il sistema operativo da cui partire con lo sviluppo.

\subsubsection{Dominio applicativo}
Il capitolato in esame vuole mostrare le potenzialità dell'uso di un motore di sintesi con effetti sonori integrati. Tali caratteristiche devono essere implementate in un'applicazione \textit{mobile} per favorire la diffusione di questa tecnologia e per metterla efficacemente in risalto.

\subsubsection{Dominio tecnologico}
Come si evince dal dominio applicativo, sarà richiesto ai membri del team lo studio dei seguenti ambiti tecnologici:
\begin{itemize}
\item Tecnologia TTS\G;
\item Tecnologia FA-TTS\G;
\item Interfacce del sistema Android\G\ messe a disposizione per lo sviluppo di nuovi applicativi;
\item Gestione di un servizio remoto;
\item Conoscenze finalizzate all'utilizzo di \textit{framework}\G\ per la 
programmazione\G.
\end{itemize}

\subsection{Valutazione del capitolato}
\subsubsection{Potenziali criticità}
\begin{itemize}
\item Text-To-Speech (TTS\G): è richiesta una buona conoscenza delle tecnologie coinvolte nell'utilizzo di sistemi di sintesi vocale (TTS). La formazione del gruppo in merito a tale ambito di studio è scarsa e sarà pertanto necessario un approfondimento da realizzarsi prima della fase di analisi dei requisiti \textit{software};
\item Flexible and Adaptive Text-To-Speech (FA-TTS\G): è richiesto uno studio 
del motore di sintesi \textit{open-source}\G\ "Flexible and Adaptive 
Text-To-Speech" sviluppato e messo a disposizione  sotto forma di applicazione 
\textit{web} dall'azienda proponente. È richiesto infatti che il motore 
FA-TTS possa integrarsi con i sistemi di sintesi vocale preesistenti nelle 
piattaforme \textit{mobile} di riferimento;
\item Sviluppo di un'applicazione \textit{mobile} a contenuto innovativo: l'azienda proponente richiede lo sviluppo di un'applicazione per dispositivi mobili che utilizzi il motore di sintesi FA-TTS. L'esperienza e le conoscenze del gruppo relative allo sviluppo su piattaforme \textit{mobile} sono nulle;
\item Trovare metodologie e tecniche efficaci per assistere l'utente durante il 
tempo di attesa relativamente lungo dovuto al campionamento della propria voce.
\end{itemize}

\subsubsection{Analisi di mercato}
La sintesi vocale è una tecnologia che già nel 2013 si è affermata positivamente nel mercato e che ha raggiunto un alto livello di applicabilità nei più disparati ambiti. Come sottolineato nel capitolato, la sintesi vocale si è diffusa rapidamente grazie ad applicazioni come le voci guida dei navigatori satellitari, gli annunci dei mezzi di trasporto pubblico, centralini telefonici, lettori di messaggi e, più in generale, assistenti vocali, in particolar modo in tutti quei casi in cui è ridotto o assente l'uso della vista. Un'applicazione che permetta la creazione di veri e propri dialoghi con voci personalizzate e distinte, e che ne permetta una facile condivisione, è senz'altro un prodotto unico e innovativo. Il punto forte di questo \textit{software} è portare a un notevole risparmio di tempo, e in alcuni casi di denaro, per quegli utenti che hanno necessità di realizzare sceneggiati propri.

\subsection{Valutazione finale}
Il capitolato presenta i seguenti punti che il \textit{team} ha valutato positivi:
\begin{itemize}
\item Interesse del gruppo nei confronti del capitolato, il cui scopo sta nell'applicare una tecnologia giovane e con svariati ambiti a cui può essere applicata. Inoltre, è stata colta positivamente la richiesta di sviluppare un'applicazione, opzionalmente multipiattaforma\G, su dispositivi \textit{mobile};
\item Esperienza e conoscenze tecniche acquisite a fine progetto che sono state ritenute formative per possibili progetti lavorativi futuri;
\item Salvo eventuali accordi, l'applicazione realizzata resterà di proprietà del gruppo di lavoro;
\item Tema libero del progetto, che lascia spazio alla creatività del gruppo.
\end{itemize}
Analogamente, sono stati riscontrati i seguenti aspetti negativi:
\begin{itemize}
\item Tema libero del progetto, che per sua natura fornisce poche informazioni e aumenta i potenziali errori vista la scarsa esperienza del \textit{team} nel settore tecnologico di riferimento;
\item Difficoltà iniziali nel preventivare il tempo e la quantità di lavoro a 
causa delle poche conoscenze possedute nelle tecnologie richieste e 
possibili ambiti d'uso, in aggiunta alla necessità di trovare un'applicazione 
del TTS\G\, con caso d'uso di facile comprensione;
\item Azienda giovane rispetto alle altre proponenti.
\end{itemize}

\newpage